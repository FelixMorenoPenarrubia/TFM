\todowrite{Write Chapter 2}

\section{Basic Definitions}

Basic definition of constructive separation as in \cite{ConstructiveSeparations}.

\begin{definition}
	\label{def:refuter}
	For a function $f \colon {0, 1}^* \to {0, 1}$ and an algorithm $\A$, 
	a $\PTIME$-refuter for $f$ against $\A$ is a deterministic polynomial time
	algorithm $R$ that, given input $1^n$, prints a string $x \in \{0, 1\}^n$,
	such that for infinitely many $n$, $A(x) \neq f(x)$.
	
	A $\BPP$-refuter for $f$ against $\A$ is a randomized polynomial time algorithm
	$R$ that, given input $1^n$, prints a string $x \in \{0, 1\}^n$, such that 
	for infinitely many $n$, $A(x) \neq f(x)$ with probability at least $2/3$.

	A $\ZPP$-refuter for $f$ against $\A$ is a randomized polynomial time algorithm $R$
	that, given input $1^n$, prints $x \in \{0, 1\}^n \cup \{\blank\}$, such that for 
	infinitely many $n$, either $x = \blank$ or $A(x) \neq f(x)$, and $x \neq \blank$ with
	probability at least $2/3$. 
\end{definition}

\begin{definition}
	\label{def:constructiveseparation}
	For $\calD \in \{\PTIME, \BPP, \ZPP\}$ and a class of algorithms $\mathcal{C}$, we say there is a 
	$\calD$-constructive separation of $f \not\in \mathcal{C}$, if for every algorithm $\A$ computable
	in $\mathcal{C}$, there is a refuter for $f$ against $A$ that is computable in $\mathcal{D}$. 
\end{definition}

\section{Main Results of \cite{ConstructiveSeparations}}

Briefly state the results of \cite{ConstructiveSeparations}, pointing to further discussion in the next chapters. 

\subsection{Conjectured Separations Can Be Made Constructive}

Results about constructive separations for conjectured uniform separations.

\subsection{Constructive Separations for Known Results Imply Breakthroughs}

\subsubsection{Results}

State the results we will discuss in this thesis.

\begin{restatable}[Theorem 1.6 from \cite{ConstructiveSeparations}]{theorem}{thmcsrefuterqa}\label{thm:csrefuterqa} 
	Let $\eps$ be a function of $n$ satisfying $\eps \leq 1/(\log n)^{\omega(1)}$, and $1/\eps$ is a positive integer
     computable in $\poly(1/\eps)$ time given $n$ in binary.
	\begin{itemize}
		\item If there is a polylogtime-uniform-$\AC^0$-constructive separation of $\MAJ_{n,\eps}$ from randomized
         query algorithms $A$ using $o(1/\eps^2)$ queries and $\poly(1/\eps)$ time, then $\NP \ne \PTIME$.
		\item If there is a polylogtime-uniform-$\NC^1$-constructive separation of $\MAJ_{n,\eps}$ from randomized
         query algorithms $A$ using $o(1/\eps^2)$ queries and $\poly(1/\eps)$ time, then $\PSPACE \ne \PTIME$.
	\end{itemize}
\end{restatable}

\begin{restatable}[Theorem 3.4 from \cite{ConstructiveSeparations}]{theorem}{thmcsrefutertm}\label{thm:csrefutertm} 
The following hold:
	\begin{itemize}
	    \item If there is a $\PTIME^{\NP}$-constructive separation of $\PAL$
		from nondeterministic $O(n^{1.1})$ time one-tape Turing machines, then
		$\E^{\NP} \not \subset \SIZE[2^{\delta n}]$ for some constant $\delta >0$.
	    
        \item If there is a $\PTIME$-constructive separation of $\PAL$ 
		from nondeterministic $O(n^{1.1})$ time one-tape Turing machines, then
		 $\E \not \subset \SIZE[2^{\delta n}]$ for some constant $\delta >0$.
        
        \item If there is a $\LOGSPACE$-constructive separation of $\PAL$ 
		from nondeterministic $O(n^{1.1})$ time one-tape Turing machines, then
		 $\PSPACE \not \subset \SIZE[2^{\delta n}]$ for some constant $\delta >0$.
	\end{itemize}
\end{restatable}


\subsubsection{Lemmas About Circuit Complexity of Refuters}

%Explain and prove ``binary-to-unary'' lemmas, which are used in many of the proofs 
%and will be useful for next chapters. 

\todo{REVIEW THIS - ATSERIAS}

\begin{lemma}
	\label{lem:circuitbinarytounary}
	Let $T(n), s(n) \geq n$ be functions and assume $\DTIME(T(2^n)) \subseteq \SIZE(s(n))$. 
	Then, for every algorithm $R$ running in time $T(n)$ and outputting $n$ bits,
	the output of $R(1^n)$ can be computed by a circuit of size $s(2 \lceil \log n \rceil)$ 
	(which takes as input an index $0 \leq i < n$ and outputs the $i$-th bit of $R(1^n)$).
\end{lemma}
\begin{proof}
Consider the function $f_R(m, i)$ which, given $m$ and $i$ in binary, outputs the $i$-th bit of $R(1^m)$. The inputs to $f_R$ can 
be encoded in $2 \lceil \log m \rceil$ bits so that the input size is the same for each input $(m, i)$ with the same $m$.
$f_R \in \DTIME(T(2^N))$, so $f_R$ has circuits of size $s(N)$, where $N = 2 \lceil \log m \rceil$ is the input size. 
We get the desired result by considering, for each $n$, the circuit for $f_R$ with input size $2 \lceil \log n \rceil$
which we modify by fixing the first $\lceil \log n \rceil$ input 
bits to the value of $n$. 
\end{proof}

\section{Additional Aspects of Refuters}

Explain additional aspects such as list refuters, black-boxness/gray-boxness/explicit constructions, certifiability, ``minimum refuting size vs. minimum counterexample size'', etc.
Mention results of \cite{Atserias06} and \cite{Bogdanov10}. 

