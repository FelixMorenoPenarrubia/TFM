
In this chapter, we will study constructive separations of the language of palindromes
$$\PAL = \{ww^R : w \in \{0, 1\}^*\} \cup \{wbw^R : w \in \{0, 1\}^*, b \in \{0, 1\}\}$$
from one-tape Turing machines running in time $o(n^2)$. In \cite{ConstructiveSeparations}
it was proved that constructive separations against nondeterministic machines yielded
breakthrough results in circuit lower bounds (recall \cref{thm:csrefutertm}). In this chapter
we will explore this setting more deeply, and explain when we can get circuit lower bounds, 
when we can actually get constructive separations, and even when we can prove that no 
constructive separations can exist. 

For simplicity, through the chapter we will often only consider the language 
of \emph{even} palindromes $\{ww^R : w \in \{0, 1\}^*\}$ in the proofs. 
Recall that in \cref{def:refuter} we only require
that the refuter works
for infinitely many values of $n$, so in particular a refuter that only works for
even palindromes and ignores the odd case is valid. However, the arguments that
we make for the even case in this chapter can always be modified slightly to 
work for the odd case too. We omit the explanations of which particular changes 
would be necessary for each of our results for brevity, since they are rarely
illuminating. 

%and, as we have said, it is often not necessary to consider the 
%odd case to reach the desired conclusion. 


\section{Lower Bounds Against 1-TMs for $\PAL$}
\label{sec:lbtm}

In this section, we prove the $\Omega(n^2)$ lower bound for 1-TMs computing $\PAL$,
due to Hennie \cite{Hennie65}. The proof of this fact lends itself to a simple
\emph{probabilistic} constructive separation that we also discuss. 

\subsection{Crossing Sequences and Lower Bounds}

\todowrite{explain indexing wrt to how input is written in the tape in the introduction}

\begin{definition}
Let $M$ be a 1-TM, and let $E = ((q_1, p_1), (q_2, p_2), \dots, (q_t, p_t))$, 
with $(q_i, p_i) \in Q \times \mathbb{Z}$, be a sequence describing an execution of
$M$, that is, $E$ is the sequence of states and positions of the tape head of $M$ when
it is executed on some input and (possibly) with some given non-deterministic choices.
The \emph{crossing sequence at position $i$}, denoted by $\CS_i(E)$,
is the sequence of states for 
which the tape head crosses from position $i$ to position $i+1$ or vice versa; that is, the
sequence of states $q_j$
such that $p_{j-1} = i$ and $p_{j} = i+1$ or $p_{j-1} = i+1$ and $p_j = i$. 

If $M$ is a deterministic 1-TM and $x \in \{0, 1\}^*$, we denote by $\CS_{M,i}(x)$ 
the crossing sequence at position $i$ of the execution of $M$ on input $x$. 

If $M$ is a non-deterministic 1-TM, $x \in \{0, 1\}^*$ and $b \in \{0, 1\}^*$ is a 
sequence of non-deterministic choices for the execution of $M$ with input $x$,
we denote by $\CS_{M,i}(x, b)$ the crossing sequence at position $i$ of the execution of $M$ on
input $x$ with non-deterministic choices $b$.

%If $M$ is a non-deterministic 1-TM and $x \in \{0, 1\}^*$ is a word accepted by $M$, we
%will denote by $\CS_i(x)$ the crossing sequence at position $i$ of the lexicographically smallest
%accepting execution of $M$ on $x$. 
\end{definition}


When the 1-TM $M$ is clear from context, we will usually write $\CS_i(x)$ instead of $\CS_{M,i}(x)$.
Also, for a non-deterministic 1-TM $M$ and an accepted word $x$, 
we will usually write just $\CS_i(x)$, omitting the description of 
the non-deterministic choices, to refer to the crossing sequence at position $i$ on input $x$
for \emph{some} accepting execution of $x$. That is, for each accepted word $x$, we arbitrarily choose
some accepting execution (e.g. the lexicographically smallest one), and use $\CS_i(x)$ to refer to the
crossing sequences of that execution. This is because, for the purposes of proving lower bounds for $\PAL$,
the fact that there might be multiple executions (and therefore multiple crossing sequences) for each accepted
word does not hurt, and we can prove the lower bounds only assuming that each accepted word has \emph{at least}
one accepting execution.

Intuitively, crossing sequences are a way to quantify the way a machine ``carries information'' from one
side of the boundary to the other one via its internal states. Note that, in a crossing sequence at position
$i \geq 1$, each odd crossing is from left to right
and each even crossing is from right to left. 

\begin{lemma}
    \label{lem:pastingcs}
    Let $a = a_1\ldots a_i a_{i+1} \ldots a_n$ and $b = b_1 \ldots b_i b_{i+1} \ldots b_m$ 
    be two words accepted by a 1-TM $M$ so that $\CS_i(a) = \CS_i(b)$.
    Then the word  $c = a_1\ldots a_i b_{i+1} \ldots b_m$ is also accepted by $M$.
\end{lemma}
\begin{proof}
Let $E^a = ((q^a_1, p^a_1), \ldots (q^a_{t^a}, p^a_{t^a}))$ be the accepting execution for $a$ in $M$ and let
$E^b = ((q^b_1, p^b_1), \ldots (q^a_{t^b}, p'_{t^b}))$ be the accepting execution for $b$. 
Let $j^a_1, j^a_2, \ldots, j^a_{|\CS_i(a)|}$ and $j^b_1, j^b_2, \ldots, j^b_{|\CS_i(a)|}$ be the indices
corresponding to the states in $\CS_i(a)$ in $E^a$ and $E^b$, respectively.

We can define a new execution $E^c$ by reproducing the execution $E^a$ up to index $j^a_1$, then the execution
$E^b$ between indices $j^b_1$ and $j^b_2$, and so on:
\begin{multline*}
E^c = ((q^a_1, p^a_1), \ldots, (q^a_{j^a_1-1}, i), (q^b_{j^b_1}, i+1), (q^b_{j^b_1+1}, p^b_{j^b_1+1}), \ldots, \\
(q^b_{j^b_2-1}, i+1), (q^a_{j^a_2}, i), (q^a_{j^a_2+1}, p^a_{j^a_2+1}), \ldots ).
\end{multline*}
It is not difficult to see that $E^c$ is a 
valid accepting execution for $M$ on input $c$, since the machine behaves as if it were reading input $a$ to the left
of index $i$ and as if it were reading input $b$ to the right of index $i$. 

\end{proof}

\begin{corollary}
    \label{cor:crosscollision}
    Let $M$ be a 1-TM that accepts two palindromes $w_1 w_2 w_2^R w_1^R$ and
    $w_3 w_4 w_4^R w_3^R$ with $|w_1| = |w_3| > 0$, $|w_2| = |w_4|$, and $w_1 \neq w_3$. 
    If the $\CS_{|w_1|}(w_1 w_2 w_2^R w_1^R) = \CS_{|w_1|}(w_3 w_4 w_4^R w_3^R)$, 
    then $M$ also accepts a non-palidrome of length $2|w_1| + 2|w_2|$.
\end{corollary}
\begin{proof}
By \cref{lem:pastingcs}, the non-palindrome $w_1 w_4 w_4^R w_3^R$ is accepted. 
\end{proof}

Now we are ready to prove the $\Omega(n^2)$ lower bound for $\PAL$. We will prove a quantitative
bound, which will be useful for our randomized constructive separation:

\todopolish{check that formulas are correctly updated everywhere}
\begin{theorem}
\label{thm:palindromebound}
Let $M$ be a non-deterministic 1-TM running in time $T(n)$ with $s$ states
that rejects every non-palindrome.
Then, for every even $n$, the number of palindromes of length $n$ accepted by $M$ is
bounded by $2^{\frac{(4 \log s) T(n)}{n} + \frac{n+6}{4}}$.
\end{theorem}
\begin{proof}
Let $P_n$ be the set of palindromes of length $n$ accepted by $M$. 

Let $$\ell_i = \frac{1}{|P_n|} \sum_{p \in P_n} |\CS_i(p)|$$ be the average length of the
crossing sequence at position $i$ on the tape for all palindromes in $P_n$.
By Markov's inequality, there are at least $\frac{1}{2}|P_n|$ palindromes $p$
from $P_n$ for which $|C_i(p) \leq 2\ell_i$.
Now, fix $i \in [1, n/2]$ and note that:

\begin{itemize}
    \item There are $1 + s + \dots + s^{2\ell_i} < 2s^{2\ell_i}$ crossing sequences
    at position $i$
    of length at most $2\ell_i$, where $s$ is the number of states in $M$.
    \item There can be at most $2^{n/2-i}$ palindromes in $P_n$ with the same crossing sequence
    at position $i$,
    since in the decomposition $p = w_1 w_2 w_2^R w_1^R$ with $|w_1| = i$,
    the prefix
    $w_1$ must be the same for all such palindromes because of \cref{cor:crosscollision}
    and the assumption that $M$ rejects every non-palindrome. 
\end{itemize}

Therefore:

$$
\frac{1}{2}|P_n| < 2 s^{2\ell_i} \cdot 2^{n/2-i}.
$$

Rearranging: 

\begin{equation}
\ell_i > \frac{\log |P_n| + i - n/2 - 2}{2 \log s}.
\label{previneqpal}
\end{equation}

For each palindrome $p \in P_n$, we have that $\sum_{i=1}^{n/2} \CS_i(p) \leq T(n)$, so

$$
\sum_{i=1}^{n/2} \ell_i = \frac{1}{|P_n|} \sum_{i=1}^{n/2} \sum_{p \in P_n} |\CS_i(p)| = \frac{1}{|P_n|} \sum_{p \in P_n} \sum_{i=1}^{n/2} |\CS_i(p)|  \leq T(n)
$$

Hence, summing over all $i$ in \eqref{previneqpal}:

$$
T(n) > \sum_{i=1/}^{n/2} \frac{\log |P_n| + i - n/2 - 2}{2 \log s} = \frac{1}{2\log s} \left( \frac{n}{2} \log |P_n| - \frac{n^2}{8} -\frac{3n}{4} \right) 
$$

%So:

%$$
%\frac{(4 \log s) (T(n)+n/2)}{n} + \frac{n+6}{4} > \log |P_n|
%$$

And therefore:

$$
2^{\frac{(4 \log s) T(n)}{n} + \frac{n+6}{4}} > |P_n|
$$

As desired. 

    
\end{proof}

\begin{corollary}
\label{cor:palindromebound}
Every non-deterministic 1-TM recognizing $\PAL$ must run in $\Omega(n^2)$ time.
\end{corollary}
\begin{proof}
Such a machine $M$ rejects every non-palindrome, so we can apply \cref{thm:palindromebound} to get that
$2^{\frac{(4 \log s) T(n)}{n} + \frac{n+6}{4}} \geq 2^{n/2}$, so $T(n) \geq \frac{n(n-6)}{16 \log s} = \Omega(n^2)$.
%where $s$ is the number of states of the machine.
\end{proof}



\subsection{Randomized Constructive Separation}

We prove the following: 

\begin{theorem}
    \begin{itemize}
    \item There is a $\ZPP^\NP$-constructive separation of $\PAL$ from nondeterministic 1-TMs 
    running in time $o(n^2)$. 
    \item There is a $\BPP$-constructive separation of $\PAL$ from nondeterministic 1-TMs 
    running in time $o(n^2)$ which
    are promised to accept only palindromes. 
    \end{itemize}
\end{theorem}

Let $M$ be the non-deterministic Turing machine running in time $o(n^2)$ we are trying to refute, 
and let $M(x, b)$ be the result of the execution of $M$ on input $x$ and non-deterministic choices $b$.
Our refuter $R$ does the following on input $1^n$:

\begin{itemize}
	\item Uses the $\NP$ oracle to determine the truth of the statement 
	$$\exists x \in \{0, 1\}^n \exists b \in \{0, 1\}^{\leq n^2} : x \not\in \PAL \wedge M(x, b) \text{ accepts}.$$ 
	\begin{itemize}
	\item If it is true, it performs a standard search-to-decision 
	reduction with the $\NP$ oracle to determine such an $x$, and outputs $x$.
	\item Otherwise, it chooses a uniformly random palindrome of length $n$ and outputs it. 
    %(If $n$ is odd, it outputs anything and halts).
	\end{itemize}
\end{itemize}

In the first case, it is clear that $R$ outputs a counterexample for $M$. 
In the second case, we have that $M$ does not accept any non-palindrome of length $n$, so
by \cref{thm:palindromebound} it also rejects a fraction which goes to $1$ when $n \rightarrow \infty$
of palindromes of length $n$, 
so the probability of outputting a counterexample will be greater than $2/3$ for sufficiently large $n$.

Thus, we have a $\BPP^\NP$-constructive separation. In fact, it can be made $\ZPP^\NP$-constructive,
since we can verify the output to be a counterexample.
%but we will not insist in this distinction.
Also note that the refuter as stated (without performing verification of the output)
uses randomness and the $\NP$ oracle in a completely ``disjoint'' way:
in the case where the machine $M$ accepts a non-palindrome, it proceeds in a deterministic,
``$\PTIME^\NP$'' way, while in the case where the machine $M$ only accepts palindromes
it just outputs a random palindrome without using the $\NP$ oracle.
This differentiated behavior is of interest because, as we will see in the following sections,
there are different consequences for constructive separations from 1-TMs which are promised
to accept only palindromes and for constructive
separations from 1-TMs which are promised to accept all palindromes. 



\section{Constructive Separations Imply Circuit Lower Bounds}

Recall that in \cite{ConstructiveSeparations}, the following is proved:

\thmcsrefutertm*

We will reproduce the proof argument here, but with some modifications to suit our
purposes better. First, we prove the result with more generality, with the circuit lower 
bound we obtain being parametrized by the time complexity of the refuter, and not just
stated in two discrete cases for $\PTIME$ and $\PTIME^\NP$ (we will not discuss results about space complexity). 
This more granular statement will be 
useful to compare the result with the refuters we obtain in \cref{sec:refuteragainstweakertm}.

Second, we emphasize in the statement that it is enough to have refuters against 1-TMs which
are promised to accept only palindromes, that is, that reject for every non-palindrome. 
This turns out to be an important aspect, because as we will see in \cref{sec:cryptotm}, no
$\PTIME$-constructive separation is possible even against deterministic 1-TMs running in time
$O(n^{1+\eps})$ assuming the existence of a cryptographic primitive. Thus, the second and third 
bullet points of \cref{thm:csrefutertm}, as stated in \cite{ConstructiveSeparations}, are 
vacuous under this cryptographic assumption. By introducing this strengthening, we can still
hope to find such constructive separations even if we believe that the cryptographic primitive exists. 

\begin{theorem}
    \label{thm:refutertm}
    Let $\{T_i(n)\}_{i \in \mathcal{I}}$ be a family of functions, let $\mathcal{O}$ be any language
    and let $\mathcal{C}^\mathcal{O}$ denote the class of algorithms
    running in time $T_i(n)$ for some $i \in \mathcal{I}$ with oracle access to $\mathcal{O}$. 
    Let $f(n)$ be a non-decreasing function with $f(n) = \Omega(\log n)$ and $f(2n) = O(f(n))$.
    If there is a $\mathcal{C}^\mathcal{O}$-constructive separation of $\PAL$ from
    1-NTMs promised to accept only palindromes running in time $O(n \cdot f(n))$, then 
    $$\bigcup_{i\in \mathcal{I}} \DTIME[T_i(2^{n/2})]^{\mathcal{O}} \not\subset \SIZE[f(2^{n/2})^\delta]$$
    for some universal constant $\delta > 0$.  
\end{theorem}
\begin{proof}
Assume that $\bigcup_{i\in \mathcal{I}} \DTIME[T_i(2^{n/2})]^{\mathcal{O}} \subseteq \SIZE[f(2^{n/2})^\delta]$. 
By \cref{lem:circuitbinarytounary}, the output of any refuter $R \in \mathcal{C}^\mathcal{O}$ has circuit complexity
$O(f(2^{\bitlength n})^\delta) = O(f(2n)^\delta) = O(f(n)^\delta)$. We will construct a 1-NTM $M$ running in time $O(n \cdot f(n))$
that works correctly on the output of any such refuter, proving the contrapositive of the desired statement. 
First, $M$ guesses a circuit $C$ of size $s = O(f(n)^\delta)$ and writes its description around the beginning of the tape.
Then, $M$ verifies that the circuit generates the input $x$ by moving the circuit and a counter through the tape
and checking that $C(i) = x_i$. Finally, $M$ checks that $C(i) = C(n-i+1)$ for all $1 \leq i \leq n/2$, verifying that
the input is indeed a palindrome. Moving the circuit through the input can be done in time $O(n \cdot (\log n + s))$ and each
of the $O(n)$ evaluations of the circuit can be done in time $O(s^{O(1)})$, so if $\delta$ is small enough the total
time complexity is $O(n \cdot f(n))$. 
\end{proof}

\begin{corollary}
The following hold:
\begin{itemize}
    \item A $\PTIME^{\NP}$-constructive separation of $\PAL$ from nondeterministic $O(n^{1.1})$ time
    one-tape Turing machines promised to accept only palindromes implies $\E^{\NP} \not \subset \SIZE[2^{\delta n}]$ for some constant $\delta >0$.
	    
    \item A $\PTIME$-constructive separation of $\PAL$ from nondeterministic $O(n^{1.1})$ time
    one-tape Turing machines promised to accept only palindromes implies $\E \not \subset \SIZE[2^{\delta n}]$ for some constant $\delta >0$.
\end{itemize}
\end{corollary}
\begin{proof}
Take $\mathcal{I} = \mathbb{N}$, $T_i(n) = n^i$, $f(n) = n^{0.1}$ and $\mathcal{O} = \SAT$ or $\mathcal{O} = \emptyset$ (respectively) in \cref{thm:refutertm}.
\end{proof}

Interestingly, those lower bounds for $\E^{\NP}$ and $\E$ are precisely the ones that are sufficient
for derandomization of
algorithms:

\begin{theorem}[\cite{NW94, IW97}] 
    \label{thm:nisanwigderson}
    The following hold:
    \begin{itemize}
    \item If $\E \not\subset \SIZE[2^{\delta n}]$, then there exists a pseudorandom generator $G : \{0, 1\}^s \to \{0, 1\}^n$,
    with $n = 2^{\Omega(s)}$ running in time $2^{O(s)}$ which fools circuits of size $n^3$.
    In particular, this implies $\BPP = \PTIME$. 
    \item If $\E^{\NP} \not\subset \SIZE[2^{\delta n}]$, then there exists a pseudorandom generator $G : \{0, 1\}^s \to \{0, 1\}^n$,
    with $n = 2^{\Omega(s)}$ running in time $2^{O(s)}$ with an $\NP$ oracle which fools circuits of size $n^3$.
    In particular, this implies $\BPP \subseteq \PTIME^\NP$. 
    \end{itemize}
\end{theorem}

Therefore, deterministic constructive separations of $\PAL$ from $O(n^{1+\eps})$ time 1-NTMs imply breakthrough results in
derandomization. At the same time, such deterministic constructive separations could be obtained by derandomizing the
probabilistic constructive separation of the previous section. The difference here between the two kinds of derandomization
is that in the second case we would need to have pseudorandom generators against \emph{nondeterministic} machines, which 
is a stronger requirement than the generators against deterministic circuits provided by \cref{thm:nisanwigderson}.

On the other hand, our generator only needs to fool nondeterministic one-tape machines running in $O(n^{1+\eps})$ time, 
which is a much weaker model than general non-deterministic polynomial-size circuits. In this setting, a pseudorandom
generator with seed length $\tilde{O}(\sqrt{T})$ against \emph{deterministic} 1-TMs running in time $O(T)$ is known 
\cite{Impagliazzo94}. Also, while the generators of \cref{thm:nisanwigderson} and \cite{Impagliazzo94} fool all 
$n^3$-size circuits and $O(T)$-time 1-DTMs respectively, our generator can be \emph{targeted}, that is, it can depend
on which TM $M$ we are trying to fool. And it does not need to be a full pseudorandom generator in the sense of 
indistinguishability from randomness; it can be a ``hitting set generator'' \cite{Andreev98} which generates at least
one element outside the square-root-sized set defined by \cref{thm:palindromebound}. All in all, it is not clear
that the requierements to derandomize our constructive separation are much stronger than the results of 
\cref{thm:nisanwigderson} by achieving it. 

\begin{question}
    What are the relationships between constructive separations of $\PAL$ from 1-NTMs, (targeted) derandomization/hitting sets
    of $O(n^{1+\eps})$ 1-NTMs, and derandomization of deterministic machines/circuits? 
    Can we find some sort of equivalence between them, or evidence than one of them is harder than the others? 
\end{question}




\section{Strongly Constructive Separations Do Not Exist Under Cryptographic Assumptions}
\label{sec:cryptotm}

In this section, we will see that $\BPP$-constructive separations of $\PAL$ against $O(n^{1+\eps})$ time 1-TMs do not exist if we assume
the existence of a certain cryptographic primitive. To show this, we will use an idea from \cite{Oliveira24}, where it is used 
to show that lower bounds for 1-TMs recognizing palindromes can not be proved in certain weak theories of arithmetic. Here we 
will reproduce their argument, adapting it to our setting.

The cryptographic primitives we will be using are \emph{keyless collision-resistant hash functions}.

\begin{definition}[Keyless Collision-Resistant Hash Function] A keyless collision-resistant hash
function with hash value length $m = m(n) < n$ is a polynomial-time computable
function $h \colon \{0, 1\}^* \to \{0, 1\}^*$ such that for every $x$ with $|x| = n$ we have $|h(x)| = m(n)$
and for every uniform probabilistic polynomial-time adversary
$\A$ and every $n$, we have
$$\Prob{\A(1^n)  \text{ outputs } \langle x_1, x_2 \rangle \text{ such that } x_1 \neq x_2, |x_1| = |x_2| = n \text{ and } h(x_1) = h(x_2)} \leq \eps(n)$$ 
where $\eps(n)$ is a \emph{negligible} function (that is, $\eps(n) = 1/n^{\omega(1)}$).
\end{definition}

The idea is that, if we have a TM that hashes the left half and the reverse of the right half of a palindrome and accepts if they are equal, a
refuter against this TM needs to find collisions in the hash function, which is supposed to be difficult. The main question here is whether
we can implement this in a one-tape TM running in time $O(n^{1+\eps})$, only assuming the existence of polynomial-time hash functions $h$.

The answer is yes: we can do it using a technique from cryptography known as the \emph{Merkle-Damgård construction} \cite{Merkle90, Damgard90}.

\begin{theorem}
\label{thm:hashonetape}
Suppose there exist keyless collision-resistant hash functions. Then, for all $0 < \delta, \eps < 1$, there exists a keyless collision resistant hash
function $H = H_{\delta, \epsilon}$ with hash value length $m(n) < n^\delta$ which can be computed in time $O(n^{1+\eps})$ in a one-tape
TM.
\end{theorem}

\begin{proof}
Let $h$ be a keyless collision-resistant hash function, computable in time $O(n^k)$ in a 1-TM for some $k$.
Without loss of generality, we can assume that $h$ has hash value length $n-1$ (we can always arbitrarily
pad the output values mantaining collision resistance). Given an input size $n$, 
let $m = m(n) = \min\{\lfloor n^\delta \rfloor,  \lfloor n^{\eps/k} \rfloor \}$ and define the following sequence of functions:

\todowrite{consistent notation for concatenation, also explain in introduction}

\begin{itemize}
    \item[] $H_{-1} \colon \{0, 1\}^{m-1} \to \{0, 1\}^{m-1}$: $H_{-1}(x) = x$
    \item[] $H_0 \colon \{0, 1\}^m \to \{0, 1\}^{m-1}$: $H_0(x) = h(x)$.
    \item[] $H_1 \colon \{0, 1\}^{m+1} \to \{0, 1\}^{m-1}$: $H_1(x \concat b) = h(H_0(x) \concat b)$.
    \item[] $\dots$
    \item[] $H_i \colon \{0, 1\}^{m+i} \to \{0, 1\}^{m-1}$: $H_i(x \concat b) = h(H_{i-1}(x) \concat b)$.
    \item[] $\dots$
    \item[] $H_{n-m} \colon \{0, 1\}^{n} \to \{0, 1\}^{m-1}$: $H_{n-m}(x \concat b) = h(H_{n-m-1}(x) \concat b)$.
\end{itemize}
 

Take $H = H_{n-m}$. We have to prove that $H$ is a collision-resistant hash function, and that it can be
computed in time $O(n^{1+\eps})$ in a 1-TM. 

We first prove that $H$ is a collision-resistant hash function.
Let $\A_H$ be a polynomial-time adversary outputting collisions for $H$. Consider the following algorithm
$\A_h$: on input $1^m$, run $\A(1^n)$ for all $n$ so that $m(n) = m$
(there are polynomially many such values of $n$). 
If $\A_H$ outputs a valid collision $\langle x_1, x_2 \rangle$ for $H$ for some such value of $n$,
do the following: compute the smallest value $i \in [0, n-m]$ such that $x_1[1 \ldots m+i] \neq x_2[1 \ldots m+i]$
but $H_i(x_1[1 \ldots m+i]) = H_i(x_2[1 \ldots m+i])$ and output
$\langle H_{i-1}(x_1[1 \ldots m+i-1]) || x_1[m+i], H_{i-1}(x_2[1 \ldots m+i-1]) || x_2[m+i] \rangle$.
We have that $h(H_{i-1}(x_1[1 \ldots m+i-1]) || x_1[m+i]) = h(H_{i-1}(x_2[1 \ldots m+i-1]) || x_2[m+i])$,
and by the assumption that $i$ was the smallest value with the specified property, at least one of
$H_{i-1}(x_1[1 \ldots m+i-1]) \neq H_{i-1}(x_2[1 \ldots m+i-1])$ or $x_1[m+i] \neq x_2[m+i]$
hold. In any case, $\A_h(1^m)$ outputs a valid collision of length $m$ for $h$. Hence:

$$
\Prob{\A_H(1^n)  \text{ outputs a collision for } H } \leq \Prob{\A_h(1^{m(n)}) \text{ outputs a collision for } h} \leq \eps_{\A_h}(m(n)),
$$

where $\eps_{\A_h}(n)$ is a negligible function since $h$ is collision-resistant, 
and therefore $\eps_{\A_h}(m(n))$ is also negligible and we get that the probability of success
of any adversary is bounded by a negligible function, as desired. 

Now we describe how to compute the function in a one-tape TM in time $O(n^{1+\eps})$:

First, we compute $n = |x|$, the length of the input. We iterate through the input from left to right,
keeping a counter near the head in a ``separate track'' from the input 
(using a larger tape alphabet).  
The counter has at most $\log n$ bits and so updating it and moving it to the right takes
$O(\log n)$ time, for a total complexity of $O(n \log n)$. 

Second, we compute $m$ and we put a mark on the $m$-th character of $x$ on the tape.
Computing $m$ given $n$ in binary can be done in $\polylog(n)$ time, since
$\eps$ and $\delta$ are fixed rational numbers, and mantaining a counter until the $m$-th character can
be done in $O(m \log m) = O(n)$, for a total complexity of $O(n)$ for this part. 

Third, we compute the hash value in a process with $n-m+1$ iterations. The $i$-th iteration ($0 \leq i \leq n-m$) is 
as follows:

\begin{enumerate}
    \item At the beginning, the tape contains $H_{i-1}(x[1 \ldots m-1+i])$, followed by the $m+i$-th 
    charcater of $x$ with a mark on it, followed by the remaining characters of $x$.
    %(For convenience, we let $H_{-1}$ be the identity function, so that this is also true for $i = 0$).
    \item We scan the tape, find the marked character, move the mark one cell to the right and copy the value 
    $H_{i-1}(x[1 \ldots m-1+i]) \concat x[m+i]$ into a separate track. 
    \item We compute $h(H_{i-1}(x[1 \ldots m-1+i]) \concat x[m+i]) = H_i(x[1 \ldots m+i])$ in a the separate
    track, cleaning it after the execution.
    \item We copy the value of $H_i(x[1 \ldots m+i])$ back into the main track, so that the tape now contains
    $H_{i}(x[1 \ldots m+i])$, followed by the $m+i+1$-th character with a mark, followed by the rest of characters
    of $x$.
\end{enumerate}

After $n-m+1$ iterations, we have the value of $H_{n-m}(x)$ in the tape, as desired. Copying the values from one
track to the other can be done in $O(m)$ time; the most expensive part of one iteration is computing $h$, which 
is done in $O(m^k)$ time. Therefore, the total time complexity is $O(n \cdot m^k) = O(n^{1+\eps})$. 



\end{proof}

\begin{theorem}
    \label{thm:nobppseparation}
    Assuming the existence of keyless collision-resistant hash functions, there does not exist a 
    $\BPP$-constructive separation of $\PAL$ against 1-DTMs running in time $O(n^{1+\eps})$ for any $\eps > 0$.
\end{theorem}

\begin{proof}
Construct a machine $M$ that computes the hash function $H = H_{\eps, \eps}$ of \cref{thm:hashonetape} on the first
half on the input and on the reverse of the second half of the input, and accepts if they are equal. Note that dividing the 
input into two halves can be done in $O(n \log n)$ time by moving a counter, 
and that the computation of $H$ can be done on the reverse of
the second half without needing to explicitly write the reverse of the second half of the tape. Comparing the 
two hashes can be done in $O(mn) = O(n^{1+\eps})$, so the machine $M$ runs in time $O(n^{1+\eps})$. 

Any counterexample against $\PAL$ for $M$ must be of the form $x_1 x_2^R$ or $x_1 b x_2^R$,
where $x_1 \neq x_2$ and $\langle x_1, x_2 \rangle$ is a collision for $H$.
Thus, any polynomial-time refuter $R$ that outputs counterexamples against $M$ can be transformed into an 
efficient adversary that outputs collisions for $H$, contradicting that $H$ is collision-resistant. 
\end{proof}

Note that the machine $M$ we constructed in \cref{thm:nobppseparation} accepts all palindromes (and therefore fails only on non-palindromes), so
the result also applies to $\BPP$-constructive separations against 1-TMs promised to fail only on non-palindromes.
This offers an interesting contrast with \cref{thm:refutertm}: refuting 1-NTMs promised to fail only on palindromes proves
circuit lower bounds, while refuting 1-TMs promised to fail only on non-palindromes proves non-existence of cryptographic
primitives, which is an ``upper bound''.




\section{Constructive Separations Against Weaker 1-TMs}
\label{sec:refuteragainstweakertm}

One can wonder whether the results of \cref{thm:refutertm} are ``tight'', in the sense of requiring the minimal possible assumptions
on the hypotheses (such as the resources used by the one-tape TMs being refuted) in order to obtain a breakthrough result via a constructive
separation. It would be interesting to find refuters when each of the hypotheses are slightly weakened, in order to establish a ``threshold''
result which sets a dividing line between the situations where refuters can be found and the situations where proving the existence of
refuters is as hard as major open problems.

In this section, we show that indeed there exist such refuters for certain weakenings of the hypotheses of \cref{thm:refutertm}.

\subsection{Refuters Against $o(n \log n)$-time 1-TMs}

The most natural weakening of the hypotheses of \cref{thm:refutertm} is to consider deterministic 1-TMs rather than non-deterministic ones,
since they are a more natural model of computation, and to consider deterministic refuters for them. The results of the above 
section show that we can not hope to find such refuters against $O(n^{1+\eps})$-time machines, but we can wonder what happens with, say,
$O(n)$-time machines. 

It turns out that even nondeterministic $O(n)$-time (even $o(n \log n)$-time) one-tape machines are very limited in computational power,
which allows us to obtain simple deterministic refuters for nondeterministic machines. 

\begin{theorem}[\cite{Hartmanis68}]
Any language recognized by a 1-NTM running in $o(n \log n)$ time is regular.
\end{theorem}

\begin{theorem}
\label{thm:regularrefuter}
There is a $\PTIME$-constructive separation of $\PAL$ against $o(n \log n)$-time 1-NTMs. 
\end{theorem}

As usual, for simplicity we will focus only on refuting for even $n$. To construct the refuter, we first show that
``the first half'' of the counterexamples are generated by finite automata:

\todowrite{Write in introduction notation used for automata}

\begin{lemma}
\label{lem:regularlang}
Given $L \in \REG$ over any alphabet $\Sigma$, the following languages are also regular:
\begin{enumerate}
    \item $L_1 := \{w : ww^R \in L\}$.
    \item $L_2 := \{w : \exists \tilde{w} \in \Sigma^*, |w| = |\tilde{w}|, w \neq \tilde{w}, w\tilde{w}^R \in L\}$.
\end{enumerate}
\end{lemma}
\begin{proof}
Let $A(L) = (Q^0, q^0_0, \delta^0, S^0)$ be the DFA for $L$.

We construct the DFA $A(L_1) = (Q^1, q^1_0, \delta^1, S^1)$ for $L_1$ by setting
\begin{gather*}
Q^1 = Q^0 \times 2^{Q^0}, \\ q_0^1 = (q_0^0, S^0), \\ S^1 = \{(q, S) : q \in S\}, \\
\delta^1((q, S), x) = (\delta^0(q, x), \{q' : \delta^0(q', x) \in S\}).
\end{gather*}

Informally, we are simulating the DFA $A(L)$, but each time a character is read we are also ``moving'' the set of accepting states
in the inverse direction of the transitions. 

We can easily verify by induction that the following property is satisfied: $A(L_1)$ is in state $(q, S)$ after reading word $w$ if and only if
$A(L)$ is in state $q$ after reading $w$ and $S$ is the set of states $s$ that satisfy that, if $A(L)$ reads $w^R$ starting at state $s$, then it
ends in an accepting state in $S^0$. As a corollary, $A(L_1)$ accepts $w$ if and only if $A(L)$ accepts $ww^R$, as desired. 

We construct the NFA $A(L_2) = (Q^2, q^2_0, \delta^2, S^2)$ for $L_2$ by setting
\begin{gather*}
Q^2 = Q^0 \times 2^{Q^0} \times \{0, 1\}, \\ q_0^2 = (q_0^0, S^0, 0), \\ S^2 = \{(q, S, 1) : q \in S\}, \\
\delta^2((q, S, 0), x) = \{(\delta^0(q, x), \{q' : \delta^0(q', x) \in S\}, 0), (\delta^0(q, x), \{q' : \exists y \neq x \, \delta^0(q', y) \in S\}, 1)\}, \\
\delta^2((q, S, 1), x) = \{(\delta^0(q, x), \{q' : \exists y \, \delta^0(q', y) \in S\}, 1)\}.
\end{gather*}

Informally, we simulate the DFA $A(L)$ as before, but this time the word $\tilde{w}$ we are reading ``backwards'' (by moving the set of 
accepting states) needs to be different from the word $w$ we are reading forwards, so we divide the states in two stages depending on 
whether there is already at least one different character in the partial words $w$ and $\tilde{w}$ or not. 

The invariant property that is satisfied in this case is:

\begin{itemize}
    \item  $A(L_2)$ can reach state $(q, S, 0)$ after reading word $w$ if and only if
    $A(L)$ is in state $q$ after reading $w$ and $S$ is the set of states $s$ that satisfy that, if $A(L)$ reads $w^R$ starting at state $s$,
    then it ends in an accepting state in $S^0$, and
    \item $A(L_2)$ can reach state $(q, S, 1)$ after reading word $w$ if and only if
    $A(L)$ is in state $q$ after reading $w$ and $S$ is the set of states $s$ that satisfy that, 
    there exists $\tilde{w} \in \Sigma^*, |w| = |\tilde{w}|, w \neq \tilde{w}$ so that if $A(L)$ reads $\tilde{w}^R$ starting at state $s$,
    then it ends in an accepting state in $S^0$.
\end{itemize}

This implies that there is an accepting execution for $w$ in $A(L_2)$ if and only if there exists
$\tilde{w} \in \Sigma^*, |w| = |\tilde{w}|, w \neq \tilde{w}$ so that $A(L)$ accepts $w\tilde{w}^R$, as desired. 


\end{proof}


\begin{proof}[Proof of \cref{thm:regularrefuter}]
Let $L \in \REG$ be the language recognized by the 1-NTM. By \cref{lem:regularlang}, the languages 
$L_1 := \{w : ww^R \not\in L\}$ and $L_2 := \{w : \exists \tilde{w} \in \Sigma^*, |w| = |\tilde{w}|, w \neq \tilde{w}, w\tilde{w}^R \in L\}$
are also regular. We assume we have access to the automata $A_{L_i}$ recognizing those languages. Note that these automata are of size $O(1)$.

%On input $1^n$ with $n$ even, our refuter does the following:

%\begin{enumerate}
%\item Search for a string $w \in L_1$ or a string $w \in L_2$ of size $n/2$. This can be done in time
%$O(n \cdot \text{size of automaton}) = O(n)$ time by constructing the product automaton for $L_i \cap \{0, 1\}^{n/2}$. 
%    \begin{enumerate}
%        \item If a valid $w \in L_1$ is found, output $ww^R$. 
%        \item If a valid $w \in L_2$ is found, find the corresponding extension $w\tilde{w}^R$ and output it. 
%        In order to find the extension, find an accepting word with prefix $w$ in the product automaton for $L \cap \{0, 1\}^n$. 
%        Note that it is possible to find $ww^R$ as the accepting word, and then it would not be a valid counterexample.
%        In this case, we search for a second accepting word with prefix $w$ different from $ww^R$. This be done in $O(n)$ time
%        in total.   
%        \item Otherwise, halt without outputting anything.
%    \end{enumerate}
%\end{enumerate}

Our refuter is described in \cref{alg:regularrefuter}.

\begin{algorithm}
\caption{Refuter of \cref{thm:regularrefuter}}\label{alg:regularrefuter}
\begin{algorithmic}
\Input $1^n$
\Require $n$ even.
\Hardcode $A_{L_1} \gets$ DFA corresponding to $L_1$.
\Hardcode $A_{L_2} \gets$ DFA corresponding to $L_2$.
\Program
\State $w \gets \Call{FindWord}{A_{L_1}, n/2}$ \Comment{$\Call{FindWord}{A, \ell}$ finds a word of length $\ell$ accepted by automaton $A$.}
\If{Found $w$}
    \Output $ww^R$
    \Halt
\EndIf
\State $w \gets \Call{FindWord}{A_{L_2}, n/2}$
\If{Found $w$}
    \State $u \gets \Call{ExtendWord}{A_{L_2}, w, ww^R, n}$ \Comment{$\Call{ExtendWord}{A, p, v, \ell}$ finds a word of length $\ell$, with prefix $p$ and different from word $v$, accepted by automaton $A$.}
    \Output $u$
    \Halt
\EndIf
\Halt
\end{algorithmic}
\end{algorithm}

\textsc{FindWord} can be implemented in time $O(n)$ by constructing the product automaton for the language $L(A) \cap \{0, 1\}^\ell$. 
Similarly, \textsc{ExtendWord} can be implemented in time $O(n)$ by constructing the corresponding product automaton and finding a 
second accepting path if the first found one results in the undesired word (the graph of the product automaton is acyclic, so finding
a second path can be done simply by e.g. a DFS traversal). 

Thus, the refuter runs in polynomial (in fact, linear) time and that it outputs counterexamples for infinitely many values of $n$
(for any even value of $n$ for which there is a counterexample, it finds one, and for big enough $n$ there are always counterexamples by 
\cref{cor:palindromebound}).
\end{proof}

Note that the refuter can be made black-box by learning the regular language recognized by the TM using black-box queries. 
In fact, if we iterate over all automata of size (say) $\log \log n$ and output the counterexample found for each of the
automata, our algorithm still runs in polynomial time, so we have an explicit obstruction against all $o(n \log n)$ 1-TMs. 

\subsection{Refuters Against $O(n \log n)$-time 1-TMs}

One-tape Turing machines running in $n \log n$ time have much more power than finite automata. And non-deterministic machines running in $n \log n$ time
are strictly more powerful than deterministic machines: for example, the language of non-palindromes can be recognized in $O(n \log n)$
non-deterministic time by guessing the index of the differing character and moving the $O(\log n)$ bits of the index around the tape,
while $\Omega(n^2)$ is required for deterministic machines to recognize this language as it is the complement of a language requiring $\Omega(n^2)$
time. 

Still, surprisingly, we can find a refuter against palindromes for these machines:

\begin{theorem}
    \label{thm:refuternlogn}

    \begin{itemize}
        \item There exists a $\PTIME^\NP$-constructive separation of $\PAL$ against one-tape nondeterministic Turing machines running in time $O(n \log n)$.
        \item There exists a $\PTIME$-constructive separation of $\PAL$ against one-tape deterministic Turing machines running in time $O(n \log n)$.
    \end{itemize}
\end{theorem}


We say that a crossing sequence is \emph{short} if it has length at most $4K \log n$. Our refuter works by repeatedly generating palindromes
and running the machine on them until either a non-accepting palindrome or a collision among short crossing sequences is found. The algorithmic
procedure is described in detail in \cref{alg:refuternlogn}; we proceed to explain how it works. 

\begin{algorithm}
    \caption{Refuter of \cref{thm:refuternlogn}}\label{alg:refuternlogn}
    \begin{algorithmic}
    \Input $1^n$
\Require $n$ even.
\Hardcode $M$ 1-TM to be refuted. 
\Hardcode $K \gets$ constant such that $M$ runs in $K n \log n$ steps.
\Hardcode $s \gets$ number of states of $M$. 
\Program
\State $L \gets 4K \log n$
\State \texttt{MarkedPrefixes} $\gets \{\}$
\State \texttt{SequenceByPrefix} $\gets []$
\RepeatTimes{$4 \cdot s^{L+1}$}
    \State $w \gets \Call{ChoosePalindrome}{n, \texttt{MarkedPrefixes}}$
    %\State Run $M$ on input $ww^R$
    \CustomBlock{\textbf{run} $M$ on input $ww^R$, and set:}
        \State $\texttt{Accepted} \gets$ result (whether $M$ accepts or not).
        \State $\CS \gets$ crossing sequences of the execution. 
    \EndCustomBlock
    \If{$\texttt{Accepted}$} 
        \For{$i \gets 1\ldots n/2$}
            \State $p \gets w[1\ldots i]$
            \If{$|\CS_i| \leq L$ and $p \not\in \texttt{MarkedPrefixes}$}
                \For{$\tilde{p} \in \texttt{MarkedPrefixes}$}
                    \If{$|\tilde{p}| = i$ and $\texttt{SequenceByPrefix}[\tilde{p}] = \CS_i$}
                        \State $v \gets w(i\ldots n/2]$
                        \Output $\tilde{p}vw^R$
                        \Halt
                    \EndIf
                \EndFor
                \State $\texttt{MarkedPrefixes} \gets \texttt{MarkedPrefixes} \cup \{p\}$
                \State $\texttt{SequenceByPrefix}[p] \gets \CS_i$
            \EndIf
        \EndFor
    \Else
        \Output $ww^R$
        \Halt
    \EndIf
\EndRepeatTimes
\Halt
    \end{algorithmic}
\end{algorithm}
\begin{algorithm}
    \caption{Procedure $\textsc{ChoosePalindrome}$ for \cref{alg:refuternlogn}}\label{alg:choosepalindrome}
    \begin{algorithmic}
    \Function{ChoosePalindrome}{n, $S$}
    \State $\texttt{Weight} \gets []$ \Comment{Associative array of integers. If $\texttt{Weight}$ is accessed for a key that has not been defined yet, it is initialized with default value $0$.}
    \State $\texttt{VerticesByLength} \gets \Call{Array}{n/2}$ \Comment{Array of sets.}
    \For{$v \in S$}
        \State $\texttt{VerticesByLength}[|v|] \gets \texttt{VerticesByLength}[|v|] \cup \{v\}$
    \EndFor
    \For{$\ell \gets n/2 \ldots 1$}
        \For {$v \in \texttt{VerticesByLength}[\ell]$}
            \If{$v \in S$}
                \State $\texttt{Weight}[v] \gets \texttt{Weight}[v]+1$
            \EndIf
            \State $p \gets \Call{Pop}{v}$
            \If{$\texttt{Weight}[p] = 0$}
                \State $\texttt{Weight}[p] \gets \texttt{Weight}[v]$
                \State $\texttt{VerticesByLength}[\ell-1] \gets \texttt{VerticesByLength}[\ell-1] \cup \{p\}$
            \Else
                \State $\texttt{Weight}[p] \gets \min\{\texttt{Weight}[p], \texttt{Weight}[v]\}$
            \EndIf
        \EndFor
    \EndFor
    \State $w \gets \emptyword$
    \RepeatTimes{$n/2$}
        \If{$\texttt{Weight}[w \concat 0] \leq \texttt{Weight}[w \concat 1]$}
            \State $w \gets w \concat 0$
        \Else
            \State $w \gets w \concat 1$
        \EndIf
    \EndRepeatTimes
    \State \Return $w$
\EndFunction
    \end{algorithmic}
\end{algorithm}

\todowrite{Explain empty word notation in introduction}

The program starts by initializing $L$ to the maximum length of short sequences (this will simply act as a shorthand
for the expression $4K \log n$ later, it is not a variable), and by creating two empty containers, 
$\texttt{MarkedPrefixes}$ and $\texttt{SequenceByPrefix}$. $\texttt{MarkedPrefixes}$ will be a set
containing prefixes of palindromes accepted by $M$ for which the $M$ generated a short crossing
sequence at the right end of the prefix in an accepting execution. $\texttt{SequenceByPrefix}$ will be
an associative array mapping each prefix in $\texttt{MarkedPrefixes}$ to the corresponding 
short crossing sequence.

Next, we have the main loop, in which in each iteration we choose a palindrome $ww^R$ and run $M$ on it.  
We choose the palindrome using the procedure \textsc{ChoosePalindrome}, described in \cref{alg:choosepalindrome}. 
This procedure returns a word $w$ of length $n/2$, and the corresponding palindrome will be $ww^R$. 
What this procedure does is to choose a $w$ minimizing the number of prefixes of $w$ which are
in \texttt{MarkedPrefixes}. It is easier to visualize it by thinking of a binary tree of height $n/2$ whose
vertices correspond to all binary words of length up to $n/2$, the root being the empty word and the two children
of each vertex corresponding to the word of the parent concatenated with each of the two characters in the alphabet.

In this setting, \texttt{MarkedPrefixes} would correspond to marked vertices in the tree, and we want to find a path
from the root to a leaf visiting the minimum number of marked vertices. We can compute the minimum number of marked
vertices in such a path by an easy recursive formula: 
$$\texttt{Weight}[v] = \texttt{Weight}[\text{child}_1(v)] + \texttt{Weight}[\text{child}_2(v)] + [1 \text{if } v \text{ is marked}],$$
and once we have computed it for all vertices, find a path with minimum weight by repeatedly descending
to the child with smaller weight. However, we can not explore the whole tree since it is of exponential size.
\textsc{ChoosePalindrome} does this in a ``bottom-up'' way so that only vertices with weight greater than $0$ are
explored. See the pseudocode description at \cref{alg:choosepalindrome} for details.

After choosing the palindrome and running it, if $M$ does not accept we can output it as a counterexample.
If $M$ does accept it, we analyze the crossing sequences at the first half of the input. If there is a prefix
with a short crossing sequence that we have not marked yet, we iterate over all other marked prefixes of the same
size. If we had marked a different prefix with the same short crossing sequence, we can replace the current prefix
by that one and we obtain a non-palindrome that is accepted, which we can output as a counterexample. If not, we
add it to the \texttt{MarkedPrefixes} and \texttt{SequenceByPrefix} containers.

We claim that, for all big enough $n$, we will always find a counterexample repeating this $4 \cdot s^{L+1}$ times,
and that the whole algorithm always runs in polynomial time. Let us prove some simple observations first:



\begin{lemma}
\label{lem:averageseqlength}
    Let $M$ be a 1-TM running in time $K n \log n$, and let $w$ be an accepted word of even length $n$. Then, for at least 
    $n/4$ of the indices $1 \leq i \leq n/2$, $|\CS_i(x)| \leq 4K \log n$. 
\end{lemma}
\begin{proof}
\begin{multline*}
K n \log n \geq \sum_{i=1}^{n/2} |\CS_i(x)| \geq 4K \log n \cdot \left(n/2 - |\{i : |\CS_i(x)| \leq 4K \log n\}|\right) \\ \implies  |\{i : |\CS_i(x)| \leq 4K \log n\}| \geq n/4.
\end{multline*}
\end{proof}

\begin{lemma}
\label{lem:choosepalindromeshortseq}
Let $M$ be a 1-TM. For $n$ large enough (depending on $M$), for every call to \textsc{ChoosePalindrome}
during the execution of the refuter
of \cref{alg:refuternlogn} against $M$, the word $w$ returned by \textsc{ChoosePalindrome} satisfies the following:
there exist at least $n/8$
indices $1 \leq i \leq n/2$ so that $|\CS_i(ww^R)| \leq 4K log n$ and the prefix $w[1 \ldots i]$ does not
already belong to \texttt{MarkedPrefixes}. (In other words, at least $n/8$ prefixes are added to \texttt{MarkedPrefixes}
in each iteration, if no counterexample is found).
\end{lemma}
\begin{proof}
There are $4 \cdot s^{L+1}$ iterations and in each iteration at most $n/2$ prefixes are added to \texttt{MarkedPrefixes}.
Therefore, at any time, there are at most $2n \cdot s^{L+1}$ prefixes in \texttt{MarkedPrefixes}. We claim that the 
word returned by $\textsc{ChoosePalindrome}(n, S)$ has at most $1+ \log |S|$  
prefixes in $S$, which for our calls with $S = \texttt{MarkedPrefixes}$ is at most 
$1 + \log(2n \cdot s^{L+1}) = O(\log n)$.

To prove so, consider the following greedy algorithm to find a path from the root of the tree to a leaf
passing through not too many marked vertices 
which could have been an alternative implementation for 
\textsc{ChoosePalindrome}: at each vertex, descend to the child with the least amount of marked vertices 
in its subtree. It is clear that after each step the number of marked vertices in the current subtree is halved,
so after $1 + \log |S|$ steps there are no marked vertices in the subtree. Therefore, there exists a path with
at most $1 + \log |S|$ marked vertices, and since \textsc{ChoosePalindrome} chooses the path with the minimum number of
marked vertices, its result must have at most that many marked vertices.

To finish, by \cref{lem:averageseqlength} the result $w$ has at least $n/4$ indices $0 \leq i \leq n/2$ with 
$|\CS_i(ww^R)| \leq 4K \log n$, so at least $n/4 - O(\log n) > n/8$ (for big enough $n$) of them are not in 
\texttt{MarkedPrefixes}. 
\end{proof}

\begin{lemma}
\label{lem:numiterations}
Let $M$ be a 1-TM. For $n$ large enough (depending on $M$), the execution of the refuter
of \cref{alg:refuternlogn} against $M$ outputs a word $x$ such that $M(x) \neq \PAL(x)$.
\end{lemma}
\begin{proof}
It is clear that whenever the refuter outputs a word, it is a correct counterexample. Therefore,
we have to prove that for $n$ large enough the refuter always outputs a counterexample. 
Suppose not, that for some $n$ large enough to apply \cref{lem:choosepalindromeshortseq}
the refuter does not output anything after the $4 \cdot s^{L+1}$ iterations.
Then, by \cref{lem:choosepalindromeshortseq}, \texttt{MarkedPrefixes} has at least $n/8 \cdot 4 \cdot s^{L+1}$
elements at the end of the execution. 
That means that for some $1 \leq i \leq n/2$, there must be at least $s^{L+1}$ of length $i$. 
There are $1 + s + \ldots + s^L < s^{L+1}$ crossing sequences of length at most $L$, so there are two 
prefixes of length $i$ with the same short crossing sequence. But that would have been detected 
during the execution of the program and a counterexample would have been printed, contradiction. 
\end{proof}

\begin{proof}[Proof of \cref{thm:refuternlogn}]
The correctness of the refuter is established by \cref{lem:numiterations}. We now need to prove that it runs in time 
polynomial in $n$.  The number of iterations is $4 \cdot s^{L+1} = 4 \cdot n^{4K \log s}$, polynomial in $n$.
Inside each iteration, $\textsc{ChoosePalindrome}(n, S)$ runs in $O(n \cdot |S|)$, which is polynomial (by the previous
argument that \texttt{MarkedPrefixes} always has an at most polynomial number of elements), 
running $M$ is $O(n \log n)$ (with possibly a call to an $\NP$ oracle),
and the subsequent iteration
over $i \gets 1 \ldots n/2$ and all elements of \texttt{MarkedPrefixes} is also polynomial.

Therefore, the refuter runs in polynomial time and we have a $\PTIME$-constructive (or $\PTIME^\NP$-constructive) 
separation. 
\end{proof}

\subsection{Additional Aspects and Consequences of the Refuters}

We can modify our refuter to work against machines that run in more than $n \log n$ time; we just have to modify the
definition of ``short sequence'' to ones that are shorter than $4T(n)/n$, where $T(n)$ is the runtime of the machine.
Our refuter then no longer runs in polynomial time in that case, though: 

\begin{theorem}
    For each 1-NTM $M$ running in time $O(n \cdot f(n))$ with $f(n) = o(n)$, there is a refuter of $\PAL$ against $M$ running in time 
    $n \cdot 2^{O(f(n))}$ with a $\NP$ oracle.
\end{theorem}

\begin{corollary}
    $\DTIME[2^{n+O(f(2^n))}] \not\subset \SIZE[f(2^{n/2})^\delta]$.
\end{corollary}

Note that by taking $f(n) = (\log n)^k$, the $\DTIME$-class bound we get is similar to the bound implicitly obtained
by the classic diagonalization used to show $\Sigma^{\PTIME}_4 \not\subset \SIZE(n^k)$ \cite{Kannan82}. This refuter
is too weak to give new circuit lower bounds, but on in some sense it seems to be ``close''. For example, if we could 
refute $O(n \polylog(n))$ 1-NTMs still using polynomial time, then we would prove $\E^\NP \not\subset \SIZE(n^k)$, 
which is not known. We now have a refuter against all $O(n^{1+o(1)})$ 1-TMs using subexponential ($2^{n^o(1)}$) time;
if we could refute $O(n^{1+\eps})$ in that time, we would get $\DTIME[2^{2^{o(n)}}] \not\subset \SIZE[2^{\delta n}]$.   

Therefore, it seems difficult to improve this refuter. However, there is some evidence that it does admit improvements.
First, for the ease of exposition here we have explained a version of the refuter which is white-box on the machine 
being refuted (indeed, it is a constructive separation in the sense of \cite{ConstructiveSeparations} which has access
to uncomputable information about the machine $M$, such as the constant $K$ in its runtime), but it can be improved 
to be black-box (with oracle access to $M$). Here we briefly explain which aspects need to be modified:

\begin{itemize}
    \item The refuter depends on the values of $s$ and $K$, which can not be obtained by black-box access. However,
    since we can verify whether the output is a valid counterexample, we can just iterate through
    $(s, K) = (1, 1), (2, 2), \ldots, (n, n)$, stopping as soon as we obtain a valid counterexample. 
    This makes the refuter no longer be a polynomial-time algorithm as a general oracle-algorithm, but for
    any fixed 1-TM $M$ given as an oracle, it will be a polynomial-time algorithm, since for $n$ big enough
    the refuter will always stop at a fixed, constant value of $s$ and $K$. 
    \item The refuter depends on knowing the crossing sequences of the execution in order to choose a new palindrome
    which ``minimizes the overlap'' with the short crossing sequences obtained by previous palindromes. 
    However, as we hinted at before, this is not really necessary: indeed, if we just generate a fixed set 
    of palindromes in which the $O(\log n)$ first bits are all different, we still get that any two palindromes
    overlap in at most $O(\log n)$ prefixes, which is enough for our purposes. 
    \item The refuter also depends on knowing the crossing sequences in order to find a collision and to construct
    the counterexample. However, we do know that a collision among the polynomially-many generated palindromes will exist
    after all the iterations, so we can just check all pairs of palindromes and all indices $i = 1, \ldots, n/2$, in 
    polynomial time in total. 
\end{itemize}

Therefore, we have a polynomial-time refuter against $O(n \log n)$-time 1-TMs that only needs oracle access to $M$
(and in particular it does not need an $\NP$ oracle for 1-NTMs). And we do not even take advantage of much information about
the oracle calls to $M$: we simply use them to ``test candidates'' for counterexamples, stopping when we find a correct
counterexample but otherwise not being adaptive at all in the queries!

However, do note that, since we need to stop at constant $(s, K)$ in order to run in polynomial time, we do not have 
a polynomial-time explicit-obstructions refuter, like we did in the $o(n \log n)$ setting. So we do lose something
when going from $o(n \log n)$ to $\Theta(n \log n)$. Still, given the austerity of our refuter when it comes to using
information about the specific machine we are refuting, it is difficult to believe we can not do better. 

\begin{question}
Is it possible to obtain a better $\PTIME^\NP$-constructive separation, even if not good enough to obtain new circuit lower
bounds, by taking advantage of white-box access and the $\NP$ oracle?
\end{question}

It is of note that our refuters work for non-deterministic 1-TMs in exactly the same way as they do for deterministic ones,
and they work for all $O(n \log n)$ 1-TMs, not just those that accept only palindromes. 
\cref{thm:refutertm} might reduce our hopes of obtaining great progress against non-deterministic machines, 
since that would imply breakthrough circuit lower bounds, but progress in refuters against deterministic 1-TMs doesn't seem
easy either, and yet we do not have any explicit reason why. \cref{thm:nobppseparation} explains why it shouldn't be possible
to have a refuter for all $O(n^{1+\eps})$ deterministic 1-TMs, but in the setting of 1-TMs which only accept palindromes,
which is linked to ``plausible'' derandomization, we do not have any obstruction. 

\begin{question}
Can we get a better $\PTIME$-constructive separation against deterministic 1-TMs promised to accept only 
palindromes? If not, can we get any result saying that it should be as difficult as obtaining certain derandomization,
like we have for non-deterministic machines?
\end{question}



