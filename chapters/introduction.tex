\section{Background and Motivation}

Computational complexity is a field of mathematics and theoretical computer science 
which studies computational problems and the resources that are needed to solve them. 
A central goal of the field is to prove \emph{lower bounds} 
on the amount of resources necessary to solve certain problems. 

Despite significant efforts over the last 50 years, progress in central 
problems of the field, such as the $\PTIME$ versus $\NP$ question, has been very limited. 
However, we do have developed some understanding about the difficulty of those problems.
This understanding has come in the form of the so-called \emph{barriers},
most notably the \emph{relativization} barrier \cite{BakerGillSolovay75},
the \emph{natural proofs} barrier \cite{RazborovRudich97}, and the 
\emph{algebrization} barrier \cite{AaronsonWigderson09}. 
These barriers display certain aspects of the results we know how to prove
(such as $\PTIME \neq \EXP$, $\PTIME \not\subset \AC^0$, and $\IP = \PSPACE$)
which are implied by the techniques used to prove them, and which are not 
shared by other results we would like to prove 
(such as $\PTIME \neq \NP$ or $\NP \not\subset \Ppoly$).
For example, the relativization barrier shows the diagonalization technique
used to prove results like $\PTIME \neq \EXP$ proves statements 
which are true \emph{relative to any oracle}, while $\PTIME \neq \NP$ is not true relative 
to some oracles. This implies that new techniques other than diagonalization
need to be used to prove $\PTIME \neq \NP$. 

In this work, we study the ``constructivity'' aspect of lower bound proofs. 
Informally, a complexity lower bound for a problem
is \emph{constructive} if for any algorithm not satisfying the lower bounds, 
we can efficiently construct a counterexample showing that the algorithm
does not solve the problem, that is, an input in which the algorithm gives a
wrong answer. 

This kind of ``refuter algorithms'' were first studied by Kabanets \cite{Kabanets01}
in the context of derandomization, but the first important result in line with the
philosophy that we are describing here was by Gutfreund, Shaltiel and Ta Shma 
\cite{Gutfreund05}. They proved that the separation $\PTIME \neq \NP$ is constructive:
that is, assuming that $\PTIME \neq \NP$, for every polynomial-time algorithm purportedly
solving $\SAT$ there is an efficient refuter which generates counterexamples against it. 

The original motivation of the \cite{Gutfreund05} was about average-case complexity.
Their result has a nice interpretation from the perspective of Impagliazzo's five worlds 
\cite{Impagliazzo95}, a classification of possible ``worlds'' we could live in 
depending on the truth or falsity of conjectures about average-case complexity 
and the existence of cryptographic primitives. One possible world, \emph{Algorithmica},
is the world where $\PTIME = \NP$ and therefore all $\NP$ problems are easy in the worst case.
Another possible world, \emph{Heuristica}, is the world where $\PTIME \neq \NP$ but all
$\NP$ problems are easy in the average case, that is, for every efficiently samplable input
distribution there exists a polynomial-time algorithm which solves the problem on inputs from
that distribution.
One could imagine an intermediate world, \emph{Super-Heuristica}, where the quantifiers are reversed
and we have one algorithm which works for all samplable distributions, despite having $\PTIME \neq \NP$. 
The results of \cite{Gutfreund05} imply that this world can not exist.

Another motivation for the results of \cite{Gutfreund05} and of subsequent work \cite{Bogdanov10} 
is more practical: provide hard examples for SAT solvers. SAT solvers, that is, heuristic algorithms
for SAT, are widely used in scientific and industrial applications.
The result of \cite{Gutfreund05} provides an algorithm for generating hard instances 
for those solvers, which may be useful for the purposes of benchmarking their performance. 

A recent article by Chen, Jin, Santhanam and Williams \cite{ConstructiveSeparations} 
provides motivation for constructivity of lower bounds as a ``barrier-style'' result:
many lower bounds that we know how to prove are non-constructive, but the lower bounds
that we would like to prove are constructive. More concretely, in \cite{ConstructiveSeparations}
they show that:

\begin{enumerate}
    \item Many important conjectured separations are constructive, including not only $\PTIME \neq \NP$ but
    also $\PTIME \neq \PSPACE$, $\BPP \neq \NEXP$ and $\ZPP \neq \EXP$.
    \item Many lower bounds that we know how to prove via non-constructive methods
    are difficult to make constructive: if a refuter algorithm were found, it would
    imply breakthrough results in complexity theory that we currently do not know how 
    to prove, including $\PTIME \neq \NP$!
\end{enumerate}

In the view of \cite{ConstructiveSeparations}, this is evidence that in order to make progress
in proving lower bounds, new techniques which are inherently ``algorithmic'' 
and constructive need to be developed. A similar observation was made by Mulmuley 
\cite{Mulmuley10} in the context of the ``Geometric Complexity Theory'' program 
to prove complexity lower bounds using algebraic geometry. 

\section{Overview}

The starting point of this work is the article \cite{ConstructiveSeparations}. In this work,
we expose some of its results and also provide some extensions, focusing on the question of
when can refuters be constructed for problems with known lower bounds.

In \cref{chp:cs}, we expose the contents of \cite{ConstructiveSeparations}, especially those
results which we will discuss later in the work, providing the necessary technical background 
for them. 

In \cref{chp:qa}, we discuss refuters against query algorithms. 
Given an input $x \in \{0, 1\}^n$, we want to distinguish between the case where $x$ has at
least $(1/2 + \eps)n$ ones and the case where $x$ has at most $(1/2 - \eps)n$ ones. 
It is a standard exercise in probability to show that we can do so by randomly sampling $O(1/\eps^2)$
positions of the string $x$. It is also known, at least since \cite{Canetti95}, that such
random sampling is the best one can do under very general conditions. That is, there is no
algorithm that reliably distinguishes between the two cases using only $o(1/\eps^2)$ queries
to the string. In \cite{ConstructiveSeparations} it is shown that making this lower bound
constructive would imply $\PTIME \neq \NP$! We discuss this refuters for this
problem in the chapter, showing a 
randomized refuter, reproducing the proof of \cite{ConstructiveSeparations} and discussing
possible variations and strengthenings, and providing deterministic refuters for weaker algorithms. 

In \cref{chp:tm}, we discuss refuters against one-tape Turing machines. 
The original machine model defined by Turing \cite{Turing36} had only one tape, 
but in modern computational complexity the model of computation is usually taken
to be \emph{multi-tape} Turing machines, since one-tape TMs have some 
unrealistic limitations.
For example, they require $\Omega(n^2)$ time to regonize palindromes \cite{Hennie65}. 
In \cite{ConstructiveSeparations}, it is shown that making this lower bound constructive
would imply breakthrough circuit lower bounds. In the chapter, we discuss refuters
for the problem of palindromes, showing a randomized refuter, reproducing the proof of 
\cite{ConstructiveSeparations} with more generality, proving that in some refuters 
can not exist if we assume the existence of a certain cryptographic primitive, 
and providing deterministic refuters for weaker one-tape TMs. 


\section{Our Countributions}

This document contains both an exposition of the results of \cite{ConstructiveSeparations}
(and their background) and some original contributions. 

Our probabilistic refuters exposed in \cref{sec:lbqa} for query algorithms and in \cref{sec:lbtm}
for Turing Machines can be considered an adapted exposition of the results of 
\cite{Canetti95} and \cite{Hennie65}, respectively, since we just expose the proof of the lower
bound and then we make the (simple) observation that it can be turned into a probabilistic refuter.
This way of looking at these results relates the claims of obtaining breakthroughs via refuters
of \cite{ConstructiveSeparations} 
with the derandomization of these probabilistic refuters. 
Also, we do note that our adaptation of the results of \cite{Canetti95} has in some aspects more
generality than the original.

Our proof of the non-existence of refuters for one-tape TMs assuming the existence of 
cryptographic primitives follows directly from an argument in \cite{Oliveira24}. However,
the context of \cite{Oliveira24} is different (non-provability of lower bounds in
bounded arithmetic), and we believe the adaptation into the constructive separations context
is valuable since it shows that specific claims made in \cite{ConstructiveSeparations}
are vacuous assuming the existence of cryptographic primitives. We explain how to modify the 
claims to make them non-vacuous even if the relevant cryptographic primitive exists. 

The most novel part of this work is the construction of refuters for weak algorithms,
developed in \cref{sec:refuteragainstweakerqa} for query algorithms and in
\cref{sec:refuteragainstweakertm} for one-tape TMs. That is, even if \cite{ConstructiveSeparations}
shows that providing (deterministic) refuters against algorithms not satisfying the lower bound would
imply breakthrough results, we do exhibit refuters that work for weaker lower bounds 
(that is, for more stringent requirements on the number of queries or the time complexity
of the 1-TM for the algorithms being refuted). We develop a ``frontier'' between the scenarios 
where refuters can be constructed and the ones where proving their existence would imply 
a breakthrough result. Often, this frontier is quite tight in the sense that slightly 
relaxing one of the conditions stated in \cite{ConstructiveSeparations} in order to get 
a breakthrough result makes a refuter easily constructible. 
In other cases, though, some small gaps remain. 
We summarize those cases in the form of ``Open Questions'' that 
we believe are interesting avenues for further research.

