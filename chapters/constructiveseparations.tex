This chapter is a more technical introduction to the rest of this work. 
We lay out the definitions of fundamental concepts that we will be using
and state the theorems proven in \cite{ConstructiveSeparations} that 
motivate this work. 

\section{Preliminaries}

We presuppose a certain level of familiarity with computational complexity theory. However, 
for convenience in this section we will define some of the basic important concepts 
that we will use, as well as explain some of the notational choices we have made
in this work. We will not be overly formal and detailed with the more common definitions.
Anything not defined here can be found in standard references for the subject, 
such as \cite{AB09}.

As usual in theoretical computer science, we will work with formal languages.
Given a set $\Sigma$, the set $\Sigma^*$ is the set of \emph{words} over 
the alphabet $\Sigma$, that is, the set of finite strings of elements of $\Sigma$.
We denote the empty word by $\emptyword$. 
A language $L$ over $\Sigma$ is a subset of $\Sigma^*$. We will often identify
decisional problems with languages over $\{0, 1\}$, with the encodings of the positive instances of the 
problem belonging to the language. 

We sometimes identify a single character $x \in \Sigma$ with the word in $\Sigma^*$ consisting of that
single character. Given two words $w_1, w_2 \in \Sigma^*$, we denote by $w_1 \concat w_2$ their 
concatenation. Sometimes we will omit the symbol and just write $w_1w_2$. We denote by exponentiation
$w^n$ the repeated concatenation of $w$ $n$ times. For example, $1^n$ denotes a string of $n$ ones.
We denote by $w^R$ the reverse of the word $w$. We denote by $w_i$ the $i$-th character of $w$ 
($1$-indexed).
We denote by $w[i \ldots j]$ the subword of 
$w$ from the $i$-th character to the $j$-th character inclusive, and denote by 
$w(i \ldots j]$, $w[i \ldots j)$, $w(i \ldots j)$ the corresponding subwords with one or both of the
ends excluded.  
We denote by $|w|$ the length of a word $w$. We denote by $[n]$ the set $\{1, \ldots, n\}$.
If $S \subseteq [|w|]$, we denote by $w[S]$ the subsequence of $w$ with indices in $S$.
Given a nonnegative integer $n$, we let 
$\bitlength n = \lfloor \log_2 n \rfloor + 1$ be the length of its binary representation.  

A determinitic finite automaton (DFA) over an alphabet $\Sigma$ is a tuple $M = (Q, q_0, \delta, S)$, 
where $Q$ is a finite set of states, $q_0 \in Q$ is the initial state, 
$\delta \colon Q \times \Sigma \to Q$ is the transition function, and $S \subseteq Q$ is the 
set of accepting states. An automaton $M$ accepts a word $w \in \Sigma^*$ if there exists a sequence
of states $p_0, p_1, \ldots p_{|w|}$ with $p_0 = q_0$, $p_{i+1} = \delta(r_i, w_{i+1})$ , and 
$p_{|w|} \in S$. A language is said to be regular if it is the set of words accepted by a 
DFA. We denote by $\REG$ the set of all regular languages.
A nondeterministic finite automaton is similarly defined, but the transition function now 
can output \emph{sets} of states. We refer the reader to \cite{HopcroftUllman} to learn more about
automata and languages. 

A one-tape Turing machine over an input alphabet $\Sigma$ is a tuple $(Q, \Gamma, \blank, 
\delta, q_0, q_A, q_R)$, where $Q$ is the set of states, $\Gamma \supset \Sigma$ is the tape alphabet,
$\blank \in \Gamma \setminus \Sigma$ is the blank symbol, 
$\delta \colon Q \times \Gamma \to Q \times \Gamma \times \{L, R\}$ is the transition function, and
$q_0, q_A, q_R \in Q$ are the initial, accepting and rejecting states. The Turing machine 
has an infinite bidirectional tape subdivided into cells which we will index by integers, 
and a moving head which can read and write on one cell at a time. 
The execution of a machine on input $x \in \Sigma^*$ begins with all the 
cells of the tape containing the character $\blank$ except the cells from
$1$ to $|x|$, which contain $x$, and with the moving head positioned in cell $1$.
The execution progresses as follows: if the machine is in state $q$ and the 
moving head is reading character $c$ from is cell and $\delta(q, c) = (q', c', D)$,
in one step the machine will write character $c'$ on the current cell of the head,
will move the head one cell in direction $D$ and will switch to state $q'$. 
The execution stops when the machine enters either the accepting or rejecting 
states. The set of words so that the machine enters the accepting state with that 
word as input is the language recognized by the machine. We can also define non-deterministic
machines, in which the transition function can point to sets of \emph{two} states, and 
we say that a word is accepted by the machine if there is a choice of transitions to follow
so that the word is accepted. 

We defined one-tape Turing machines, in contrast with the 
``multitape'' model which is more common in complexity theory \cite{AB09}, because 
we will use this model of computation in \cref{chp:tm}. However, other than that, 
our ``default'' model of computation will be the multitape Turing machines, and 
``algorithms'' are understood to be multi-tape TMs unless otherwise specified. 
We will abbreviate 
``one-tape Turing machine'' to 1-TM, and we will use 1-DTM and 1-NTM to specifically 
refer to deterministic and non-deterministic one-tape Turing machines, respectively.
It is well-known that 1-TMs can simulate multitape TMs with a quadratic overhead. 
One of the techniques to do so is to set the tape alphabet of the 1-TM to be the product
of $k$ copies of the tape alphabet of the $k$-tape TM. This way, each cell in the tape is divided 
into $k$ ``tracks'', each representing one tape of the $k$-tape machine. We will use again
at some points in \cref{chp:tm}
this technique of extending the tape alphabet and dividing into tracks. 

Now we will see some definitions about query algorithms, the other limited 
model of computation we will work with in this thesis. 
The following definitions are not so common, so we isolate them for clarity.

\begin{definition}
	A function $f \colon \{0, 1\}^n \to \{0, 1\}$ is a \emph{$k$-junta} if there is a set of $k$ indices
	so that $f$ does not depend on the bits outside that set.

	A decision tree over $\{0, 1\}^n$ is a tree, where each internal node
is labeled with an index $i$ and has two children, the two edges leading from an internal node to its
children are labeled 0 and 1, and each leaf is labeled with an output value (0 or 1). A decision tree
computes a function $f \colon \{0, 1\}^n \to \{0, 1\}$ by moving from the root to a leaf according to the values of the
values queried.
\end{definition}

We also want to consider query algorithms that are \emph{uniform} (i.e. they are generated by 
a single algorithm for all $n$), and also that are \emph{probabilistic}.

\begin{definition}
	Let $k(n)$ be a function of $n$ with $k(n) \leq n$.

	A \emph{uniform distribution of $k(n)$-juntas} is a probabilistic algorithm $\A$ which, 
	being run on input $x$ with $|x| = n$, is given as input $n$ in binary together with 
	$O(k(n) \log n)$ bits describing a subset $S$ of size $k(n)$ of $[n]$, together with the values
	of $x$ on the indices of $S$.

	A \emph{(randomized) query algorithm using $k(n)$ queries} is a probabilistic algorithm $\A$
	which, being run on input $x$ with $|x| = n$, is given as input $n$ in binary and has oracle access
	to $x$ (i.e. has a special tape and a special state so that, when writing $i$ in binary in the tape,
	is given the value of $x_i$) which always does at most $k(n)$ oracle queries.
\end{definition}

The way inputs are given in the definition of query algorithm is relevant, since we will 
discuss query algorithms running in time sublinear in $n$. 

When executing a randomized or non-deterministic algorithm $\A$ on input $x$
with a fixed sequence of 
random or non-deterministic choices $r$, we will usually denote the result by $\A(x, r)$.
Another concept we will use for randomized algorithms is the \emph{probability gap}.

\begin{definition}
A randomized decisional algorithm $\A(x, r)$ has \emph{probability gap} $\delta$ if there exists a function
$\determine{\A} \colon \{0, 1\}^* \to \{0, 1\}$ such that, for all $x$, 
$\Probsub{r}{\A(x, r) = \determine{\A}(x)} \geq \frac{1 + \delta}{2}$. 
The function $\determine{\A}$ is the \emph{determination} of $\A$.

We say that a randomized algorithm $\A(x, r)$ has bounded probability gap if it has probability gap $\delta$
for some $\delta > 0$. 
\end{definition}

Finally, we define some terms related to boolean circuits. 

A boolean circuit is a directed acyclic graph where each of the non-source vertices is called a 
gate and has associated a type $\wedge$, $\vee$ or $\neg$, corresponding to the logical operations
\texttt{AND}, \texttt{OR} and \texttt{NOT}. The vertices of $\neg$ type have indegree $1$; we say
that the circuit is of \emph{unbounded fan-in} if we allow the vertices of type $\wedge$ and 
$\vee$ to have any indegree, otherwise the vertices of $\wedge$ and $\vee$ types should have
indegree $2$. The size $|C|$ of a circuit $C$ is the number of vertices in it.
A \emph{circuit family} is a sequence of circuits $\{C_n\}_{n}$ in which 
$C_i$ has $i$ sources. It has polynomial size (respectively, quasipolynomial size) if there exists
a polynomial $f(n)$ (resp. $f(n) = 2^{(\log n)^{O(1)}}$) so that $|C_n| \leq f(n)$.
A circuit with $n$ sources is evaluated on an input $x \in \{0, 1\}^n$ in the natural way. 

\begin{definition}
	The \emph{depth} of a circuit is the length of the longest path from a source to a sink.

	A family of unbounded fan-in circuits is $\AC^0$ (resp. $\qAC^0$) if it has polynomial
	(resp. quasipolynomial) size and all the circuits in the family
	have depth bounded by a constant. 

	A family $\{C_n\}_{n}$ of circuits is $\NC^1$  (resp. $\qNC^1$) if it has polynomial 
	(resp. quasipolynomial)
	size and circuit $C_n$
	has depth $O(\log n)$.
\end{definition}

\begin{definition}
	A family of circuits $\{C_n\}_{n}$ is \emph{polylogtime-uniform} if there is an 
	algorithm $B$ that, given $n$ in binary and additional input of length $2 \bitlength |C_n|$ 
	identifying two vertices of $C_n$, reports the types of gates of the two vertices and whether
	there is an edge between them in $\polylog(n)$ time.  
\end{definition}



\section{Refuters and Constructive Separations}

\subsection{Basic Definitions}

Here we define the concept of \emph{constructive separation} as defined in \cite{ConstructiveSeparations}.

\begin{definition}
	\label{def:refuter}
	For a function $f \colon \{0, 1\}^* \to \{0, 1\}$ and an algorithm $\A$, 
	a $\PTIME$-refuter for $f$ against $\A$ is a deterministic polynomial time
	algorithm $R$ that, given input $1^n$, prints a string $x \in \{0, 1\}^n$,
	such that for infinitely many $n$, $\A(x) \neq f(x)$.
	
	A $\BPP$-refuter for $f$ against $\A$ is a randomized polynomial time algorithm
	$R$ that, given input $1^n$, prints a string $x \in \{0, 1\}^n$, such that 
	for infinitely many $n$, $A(x) \neq f(x)$ with probability at least $2/3$.

	A $\ZPP$-refuter for $f$ against $\A$ is a randomized polynomial time algorithm $R$
	that, given input $1^n$, prints $x \in \{0, 1\}^n \cup \{\blank\}$, such that for 
	infinitely many $n$, either $x = \blank$ or $\A(x) \neq f(x)$, and $x \neq \blank$ with
	probability at least $2/3$. 
\end{definition}

We allow the algorithm $\A$ to be randomized, but we will only consider algorithms with bounded 
probability gap. Unless otherwise stated, when we say we are constructing a refuter against a 
class of randomized algorithms $\A$ we will impose that the algorithms $\A$ being refuted 
must have a probability gap of $1/3$, and the outputs of the refuter $R$ must be counterexamples
against $\determine{\A}$, the determination function of $\A$. 

\begin{definition}
	\label{def:constructiveseparation}
	For $\calD \in \{\PTIME, \BPP, \ZPP\}$ and a class of algorithms $\mathcal{C}$, we say there is a 
	$\calD$-constructive separation of $f \not\in \mathcal{C}$, if for every algorithm $\A$ computable
	in $\mathcal{C}$, there is a refuter for $f$ against $A$ that is computable in $\mathcal{D}$. 
\end{definition}

This is the definition as given in \cite[Definition 1.1]{ConstructiveSeparations}, but in the article
$\D$-refuters and $\D$-constructive separations for other classes of algorithms $\D$ are also considered.
For other classes of deterministic algorithms $\D$, the definition of $\D$-refuter and $\D$-constructive 
separation is as in the definition of $\PTIME$-refuter, but replacing ``polynomial time algorithm''
with ``algorithm from the class $\D$''.

\subsection{Additional Aspects of Refuters}

A possible relaxation that we can consider of \cref{def:refuter} is to allow the refuter 
to output not only one counterexample, but a list of possible candidates that contains
at least one counterexample.

\begin{definition}
	For a function $f \colon \{0, 1\}^* \to \{0, 1\}$, an algorithm $\A$,
	and a function $\ell(n) \colon \mathbb{N} \to \mathbb{N}$ 
	a $\ell(n)$-length $\PTIME$-list-refuter for $f$ against $\A$ is a deterministic polynomial time
	algorithm $R$ that, given input $1^n$, prints a list of length $n$ strings
	$x^{(1)}, \ldots, x^{(\ell(n))} \in \{0, 1\}^n$,
	such that for infinitely many $n$, there exists an $i \in [\ell(n)]$ so that 
	$\A(x^{(i)}) \neq f(x^{(i)})$.
\end{definition}

We can extend the definition to $\mathcal{D}$-list-refuters for other classes $\mathcal{D}$,
and we can refer to polynomial-length or quasipolynomial-length list-refuters to indicate that 
the function $\ell(n)$ is polynomial or quasipolynomial, respectively. 

Another possible variant concerns the dependency of the refuter with the algorithm. 
Note that in \cref{def:constructiveseparation}, the refuter is allowed to 
depend non-uniformly in the algorithm. In \cite{Gutfreund05}, the refuter is an uniform
algorithm which is given the source code and the running time of the algorithm being refuted. 
Atserias \cite{Atserias06} constructed a $\BPP$-refuter against polynomial-size circuits 
computing SAT which is \emph{almost} black-box: it only needs oracle access to the circuits
and a bound on the exponent of their size. A truly black-box refuter would only use oracle 
access to the algorithm being refuted. We can also consider refuters that do not depend on the 
algorithm being refuted at all: those were called \emph{explicit obstructions} in \cite{Chen20},
or also \emph{oblivious list-refuters}.

\begin{definition}
	For a function $f \colon \{0, 1\}^* \to \{0, 1\}$, a class of algorithms $\mathcal{C}$,
	and a function $\ell(n) \colon \mathbb{N} \to \mathbb{N}$ 
	a $\ell(n)$-length explicit obstructions $\PTIME$-refuter for $f$ against $\mathcal{C}$ is a deterministic polynomial time
	algorithm $R$ that, given input $1^n$, prints a list of length $n$ strings
	$x^{(1)}, \ldots, x^{(\ell(n))} \in \{0, 1\}^n$,
	such that for all $\A \in \mathcal{C}$, for infinitely many $n$, there exists an $i \in [\ell(n)]$ so that 
	$\A(x^{(i)}) \neq f(x^{(i)})$.
\end{definition}



\section{Constructive Separations for Known Results Imply Breakthroughs}

We proceed to state the 
theorems proven in \cite{ConstructiveSeparations} that are the foundations 
for this work. They say that the existence of constructive separations of problems 
with known lower bounds for restricted models of computation implies breakthrough results in complexity theory. 
This section is an introduction to these results; their proofs and further study of them will be 
done in the following chapters. 

In \cite{ConstructiveSeparations} there are four pairs of restricted models of computation
and computational problems studied: query algorithms and the promise majority problem,
one-tape Turing machines and the palindromes problem, streaming algorithms and the set disjointness problem,
and $\AC^0$ circuits and the minimum circuit size problem. In this work, we will only consider the first two. 

\subsection{Query Algorithms and $\MAJ$}

For a function $\eps(n)$ of $n$, consider the following basic problem $\MAJ_{n, \eps}$ (denoted 
by $\textrm{PromiseMAJORITY}_{n, \eps}$ in \cite{ConstructiveSeparations}):

\begin{itemize}
	\item[] $\MAJ_{n, \eps}$: given an input $x \in \{0, 1\}^n$, determine whether $w_1(x) \geq (1/2+\eps)n$
	or $w_1(x) \leq (1/2-\eps)n$, given that it is one of the two cases. 
\end{itemize}

It is proven in \cite{Canetti95} that query algorithms solving this problem must make $\Omega(1/\eps^2)$ 
queries; in \cite{ConstructiveSeparations} they show that making the lower bound sufficiently constructive
implies $\PTIME \neq \NP$! Here, ``sufficiently constructive'' means that the refuter needs to be 
polylogtime-uniform-$\AC^0$ (though, as we will see, polylogtime-uniform-$\qAC^0$ refuters suffice), but
it is still possible to get $\PTIME \neq \PSPACE$ using the less restrictive polylogtime-uniform-$\NC^1$ circuits.

\begin{restatable}[Theorem 1.6 from \cite{ConstructiveSeparations}]{theorem}{thmcsrefuterqa}\label{thm:csrefuterqa} 
	Let $\eps$ be a function of $n$ satisfying $\eps \leq 1/(\log n)^{\omega(1)}$, and $1/\eps$ is a positive integer
     computable in $\poly(1/\eps)$ time given $n$ in binary.
	\begin{itemize}
		\item If there is a polylogtime-uniform-$\AC^0$-constructive separation of $\MAJ_{n,\eps}$ from randomized
         query algorithms $A$ using $o(1/\eps^2)$ queries and $\poly(1/\eps)$ time, then $\NP \ne \PTIME$.
		\item If there is a polylogtime-uniform-$\NC^1$-constructive separation of $\MAJ_{n,\eps}$ from randomized
         query algorithms $A$ using $o(1/\eps^2)$ queries and $\poly(1/\eps)$ time, then $\PSPACE \ne \PTIME$.
	\end{itemize}
\end{restatable}

We will discuss this result and other aspects of refuters of the $\MAJ$ problem 
for query algorithms in \cref{chp:qa}.

\subsection{One-Tape Turing Machines and $\PAL$}

Consider the language of binary palindromes $\PAL = \{w \in \{0, 1\}^* : w = w^R\}$. One of the 
earliest results in complexity theory is a lower bound of $\Omega(n^2)$ time for recognizing 
this language in the model of 1-TMs. This bound works even for non-deterministic machines. 
In \cite{ConstructiveSeparations} they show that making the lower bound constructive
would imply breakthrough circuit lower bounds:

\begin{restatable}[Theorem 3.4 from \cite{ConstructiveSeparations}]{theorem}{thmcsrefutertm}\label{thm:csrefutertm} 
The following hold:
	\begin{itemize}
	    \item If there is a $\PTIME^{\NP}$-constructive separation of $\PAL$
		from nondeterministic $O(n^{1.1})$ time one-tape Turing machines, then
		$\E^{\NP} \not \subset \SIZE[2^{\delta n}]$ for some constant $\delta >0$.
	    
        \item If there is a $\PTIME$-constructive separation of $\PAL$ 
		from nondeterministic $O(n^{1.1})$ time one-tape Turing machines, then
		 $\E \not \subset \SIZE[2^{\delta n}]$ for some constant $\delta >0$.
        
        \item If there is a $\LOGSPACE$-constructive separation of $\PAL$ 
		from nondeterministic $O(n^{1.1})$ time one-tape Turing machines, then
		 $\PSPACE \not \subset \SIZE[2^{\delta n}]$ for some constant $\delta >0$.
	\end{itemize}
\end{restatable}

We will discuss this result and other aspects of refuters of the $\PAL$ problem 
for 1-TMs in \cref{chp:qa}.

\subsection{Lemmas About Circuit Complexity of Refuters}

%Explain and prove ``binary-to-unary'' lemmas, which are used in many of the proofs 
%and will be useful for next chapters. 

The proofs in \cite{ConstructiveSeparations} about refuters against limited models of computation 
implying breakthrough results all follow the same scheme:

\begin{itemize}
	\item Assuming the negation of the breakthrough result, the output $R(1^n)$ of the refuter
	can be generated by small circuits.
	\item The limited model of computation has enough power (possibly assuming the negation
	of the breakthrough result) to ``learn'' or ``guess'' the circuit generating the input. 
	\item Once the circuit generating the input is known, the problem can be solved in 
	the limited model of computation. Hence, there is an algorithm in that model which works 
	against all refuters, proving the contrapositive of the statement. 
\end{itemize}

The first step is done through the following simple lemmas:

\begin{lemma}
	\label{lem:circuitbinarytounary}
	Let $T(n), s(n)$ be non-decreasing functions such that $T(n) \geq n$ and $s(n) \geq n$. 
	If $\DTIME(T(2^{n/2})) \subseteq \SIZE(s(n))$,
	then for every algorithm $R$ running in time $T(n)$ and outputting $n$ bits,
	the output of $R(1^n)$ can be computed by a circuit of size $s(2 \bitlength n)$,
	which takes as input an index $0 \leq i < n$ and outputs the $i$-th bit of $R(1^n)$.
\end{lemma}
\begin{proof}
Consider the function $f_R(m, i)$ which, given $m$ and $i$ in binary, outputs the $i$-th bit of $R(1^m)$. The inputs to $f_R$ can 
be encoded in $N = 2 \bitlength n$ bits so that the input size is the same for each input $(m, i)$ with the same $m$.
Processing the input takes $O(2^{N/2}) = O(T(2^{N/2}))$ time and computing takes $T(n) \leq T(2^{N/2})$ time, so
$f_R \in \DTIME(T(2^{N/2}))$. 
Therefore, $f_R$ has circuits of size $s(N)$. 
We get the desired result by considering, for each $n$, the circuit for $f_R$ with input size $2 \bitlength n$
which we modify by fixing the first $\bitlength n$ input 
bits to the binary representation of $n$. 
\end{proof}

\begin{lemma}
	\label{lem:acbinarytounary}
	The following hold:
	\begin{enumerate}
		\item Assuming $\PTIME = \NP$, for every polylogtime-uniform $\qAC^0$ algorithm $R$
		such that $R(1^n)$ outputs $n$ bits, $R(1^n)$ can be computed by a $\polylog(n)$-size circuit
		which takes as input an index $0 \leq i < n$ and outputs the $i$-th bit of $R(1^n)$.
		\item Assuming $\PTIME = \PSPACE$, for every polylogtime-uniform $\qNC^1$ algorithm
		$R$ such that $R(1^n)$ outputs $n$ bits, $R(1^n)$ can be computed by a $\polylog(n)$-size circuit
		which takes as input an index $0 \leq i < n$ and outputs the $i$-th bit of $R(1^n)$.
	\end{enumerate}
\end{lemma}

\begin{proof}
	We prove the first item:

	Let $B$ be the polynomial-time algorithm that, given integer $n$
	in binary and $polylog(n)$-bit additional input, reports gate and wire information
	for a $\qAC^0$ circuit $R_n$. Consider the function $f_R(m, i)$ which, 
	given $m$ and $i$ in binary, outputs the $i$-th bit of $R(1^m) = R_m(1^m)$.

	We have that $f_R \in \PH$, since the evaluation of the circuit consists 
	of a finite alternation of existential and universal statements (corresponding to the 
	OR and AND gates of the circuit) which can be verified using the polynomial-time
	algorithm $B$. Since $\PTIME = \NP$, $f_R$ is computable in $\PTIME$, which means that it has 
	polynomial-size circuits. We conclude as in the previous lemma. 

	The same argument applies to the second item replacing $\qAC^0$ by $\qNC^1$ and $\PH$ by $\PSPACE$. 
\end{proof}

