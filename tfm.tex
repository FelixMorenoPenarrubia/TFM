\documentclass[12pt]{report}

\usepackage{amsthm,amsmath,amssymb,amsfonts,amscd}
\usepackage{graphicx}
\usepackage{enumerate}
\usepackage[all]{xy}
\usepackage{booktabs}
\usepackage{tikz}


%Dissertate-inspired
\usepackage{hyperref}
\usepackage[capitalise]{cleveref}
\usepackage{xcolor}
\hypersetup{
    colorlinks,
    linkcolor={red!60!black},
    citecolor={red!60!black},
    urlcolor={blue!80!black}
}
%\usepackage[tiny,md,sc]{titlesec}
\setlength{\headheight}{15pt}
\pagestyle{plain}
\usepackage{titling}
\usepackage[palatino]{quotchap}

\usepackage[width=6in, letterpaper]{geometry}
\usepackage{fancyhdr}
\usepackage{setspace}

%Thmtools
\usepackage{thmtools}
\usepackage{thm-restate}
%\declaretheorem[name=Theorem]{citetheorem}
\declaretheorem[within=chapter]{theorem}
\declaretheorem[sibling=theorem]{lemma}
\declaretheorem[sibling=theorem,style=definition]{definition}
\declaretheorem[sibling=theorem]{corollary}
\declaretheorem[style=definition, name=Open Question]{question} 

%\newtheorem{theorem}{Theorem}[section]
%\newtheorem{lemma}[theorem]{Lemma}
%\newtheorem{proposition}[theorem]{Proposition}
%\newtheorem{observation}[theorem]{Observation}
%\newtheorem{conjecture}[theorem]{Conjecture}


%\theoremstyle{definition}
%\newtheorem{definition}[theorem]{Definition}
%\newtheorem{example}[theorem]{Example}


\newcommand{\eps}{\epsilon}
\newcommand{\A}{\mathcal{A}}
\newcommand{\calD}{\mathcal{D}}
\newcommand{\D}{\calD}
\newcommand{\Prob}[1]{\text{Pr}\left[#1\right]}
\newcommand{\Probsub}[2]{\text{Pr}_{#1}\left[#2\right]}
\newcommand{\EV}[1]{\mathbb{E}\left[#1\right]}
\newcommand{\EVsub}[2]{\mathbb{E}_{#1}\left[#2\right]}
\newcommand{\samplefrom}{\leftarrow}
\newcommand{\F}{\mathbb{F}}
\newcommand{\determine}[1]{\bar{#1}}

\newenvironment{proofsketch}{%
  \renewcommand{\proofname}{Proof sketch}\proof}{\endproof}

\newcommand{\complexityclass}[1]{\mathsf{#1}}
\newcommand{\PTIME}{\complexityclass{P}}
\newcommand{\newcomplexityclass}[1]{\expandafter\newcommand\csname #1\endcsname{\complexityclass{#1}}}
\newcomplexityclass{EXP}
\newcomplexityclass{NEXP}
\newcomplexityclass{E}
\newcomplexityclass{BPP}
\newcomplexityclass{ZPP}
\newcomplexityclass{NP}
\newcomplexityclass{PSPACE}
\newcomplexityclass{AC}
\newcomplexityclass{NC}
\newcomplexityclass{qAC}
\newcomplexityclass{qNC}
\newcomplexityclass{REG}
\newcomplexityclass{LOGSPACE}
\newcomplexityclass{IP}


\newcomplexityclass{SIZE}
\newcomplexityclass{DTIME}


\newcommand{\poly}{\text{poly}}
\newcommand{\polylog}{\text{polylog}}
\newcommand{\Ppoly}{\PTIME\text{/poly}}

\newcommand{\computationalproblem}[1]{\mathrm{#1}}
\newcommand{\PAL}{\computationalproblem{PAL}}
\newcommand{\DISJ}{\computationalproblem{DISJ}}
\newcommand{\MAJ}{\computationalproblem{PromiseMAJORITY}}
\newcommand{\SAT}{\computationalproblem{SAT}}


%Crossing Sequences
\newcommand{\CS}{C}


\newcommand{\emptyword}{\lambda}
\newcommand{\concat}{||}
\newcommand{\blank}{\perp}
\newcommand{\complementary}[1]{\bar{#1}}

\usepackage{algorithm}
\usepackage{algorithmicx}
\usepackage{algpseudocode}
\algnewcommand\algorithmicinput{\textbf{Input:}}
\algnewcommand\Input{\item[\algorithmicinput]}
\algnewcommand\algorithmichardcode{\textbf{Hardcode:}}
\algnewcommand\Hardcode{\item[\algorithmichardcode]}
\algnewcommand\algorithmicprogram{\textbf{Program:}}
\algnewcommand\Program{\item[\algorithmicprogram]}
\algnewcommand\algorithmicoutput{\textbf{output} }
%\algnewcommand\Output{\item[\algorithmicoutput]}
\algnewcommand\Output{\State \algorithmicoutput}
%\algnewcommand\algorithmicreturn{\textbf{Return}}
%\algnewcommand\Return{\item[\algorithmicreturn]}
\algnewcommand\algorithmichalt{\textsc{Halt}}
%\algnewcommand\Halt{\item[\algorithmichalt]}
\algnewcommand\Halt{\State \algorithmichalt}

\algblockdefx[RepeatTimes]{RepeatTimes}{EndRepeatTimes}%
[1]{\textbf{repeat } #1 \textbf{ times:}}%
{\textbf{end}}

\algblockdefx[CustomBlock]{CustomBlock}{EndCustomBlock}%
[1]{#1}%
{\textbf{end}}

%Bibliography
\usepackage[style=alphabetic]{biblatex}
\addbibresource{references.bib}
% Suppress DOI
\DeclareFieldFormat{doi}{}
% Suppress URL
\DeclareFieldFormat{url}{}
% Suppress the URL date; comment out if you still want to see the access date
\DeclareFieldFormat{urldate}{}
\DeclareFieldFormat{issn}{}


%todonotes
\usepackage[colorinlistoftodos]{todonotes}
\definecolor{fadedviolet}{rgb}{1,0.8,1}
\definecolor{fadedgreen}{rgb}{0.8,1,0.8}
\newcommand{\todoidea}[1]{\todo[color=pink]{#1}}
\newcommand{\todoidealp}[1]{\todo[color=fadedviolet]{#1}}
\newcommand{\todoideahp}[1]{\todo[color=magenta]{#1}}
\newcommand{\todowrite}[1]{\todo[color=fadedgreen]{#1}}
\newcommand{\todowritehp}[1]{\todo[color=green]{#1}}
\newcommand{\todopolish}[1]{\todo[color=white]{#1}}
\newcommand{\todonit}[1]{\todo[color=gray]{#1}}


\begin{document}

\onehalfspacing

\listoftodos

\begin{titlepage}
    \begin{center}
        \vspace*{1cm}
            
        \Huge
        \textbf{Refuters for Algorithms with Known Lower Bounds}
            
        \vspace{0.5cm}
        \Large
        \textit{Félix Moreno Peñarrubia}
            
        
            
        \,Master's Thesis
            
        \vspace{0.8cm}
            
       
            
        \Large
        Master's Degree in Advanced Mathematics and Mathematical Engineering \\
        Facultat de Matemàtiques i Estadística, UPC 
        \vspace{1.2cm}
        

          
        \includegraphics[height=0.8cm]{figures/logo_FME_cropped.png}
 
        \vfill
        
      
        June 2024
        
		\vspace{2cm}
		\large        
        \textit{Director: Albert Atserias}
        
        
            
    \end{center}
\end{titlepage}

\section*{Abstract}

Proving lower bounds for algorithms on general models of computation is a central yet
largely unfulfilled goal in the field of computational complexity. 
However, for some restricted models of computation strong lower bounds are known.
A lower bound for a problem is said to be constructive if there is an efficient algorithm, 
known as a refuter, that produces inputs on which the algorithm fails. 
Recent results indicate that constructive lower bounds for restricted models are 
as difficult to obtain as some lower bounds for general models. 
The goal of this work is to explore the state of the art of the construction of refuters 
for problems with known lower bounds.

%\textbf{Keywords:} computational complexity, constructive separations,

\textbf{MSC2020 codes:} 68Q15, 68Q17, 68Q04




\newpage

\tableofcontents

\newpage


\chapter{Introduction}

\section{Background and Motivation}

Computational complexity is a field of mathematics and theoretical computer science 
which studies computational problems and the resources that are needed to solve them. 
A central goal of the field is to prove \emph{lower bounds} 
on the amount of resources necessary to solve certain problems. 

Despite significant efforts over the last 50 years, progress in central 
problems of the field, such as the $\PTIME$ versus $\NP$ question, has been very limited. 
However, we do have developed some understanding about the difficulty of those problems.
This understanding has come in the form of the so-called \emph{barriers},
most notably the \emph{relativization} barrier \cite{BakerGillSolovay75},
the \emph{natural proofs} barrier \cite{RazborovRudich97}, and the 
\emph{algebrization} barrier \cite{AaronsonWigderson09}. 
These barriers display certain aspects of the results we know how to prove
(such as $\PTIME \neq \EXP$, $\PTIME \not\subset \AC^0$, and $\IP = \PSPACE$)
which are implied by the techniques used to prove them, and which are not 
shared by other results we would like to prove 
(such as $\PTIME \neq \NP$ or $\NP \not\subset \Ppoly$).
For example, the relativization barrier shows the diagonalization technique
used to prove results like $\PTIME \neq \EXP$ proves statements 
which are true \emph{relative to any oracle}, while $\PTIME \neq \NP$ is not true relative 
to some oracles. This implies that new techniques other than diagonalization
need to be used to prove $\PTIME \neq \NP$. 

In this work, we study the ``constructivity'' aspect of lower bound proofs. 
Informally, a complexity lower bound for a problem
is \emph{constructive} if for any algorithm not satisfying the lower bounds, 
we can efficiently construct a counterexample showing that the algorithm
does not solve the problem, that is, an input in which the algorithm gives a
wrong answer. 

This kind of ``refuter algorithms'' were first studied by Kabanets \cite{Kabanets01}
in the context of derandomization, but the first important result in line with the
philosophy that we are describing here was by Gutfreund, Shaltiel and Ta Shma 
\cite{Gutfreund05}. They proved that the separation $\PTIME \neq \NP$ is constructive:
that is, assuming that $\PTIME \neq \NP$, for every polynomial-time algorithm purportedly
solving $\SAT$ there is an efficient refuter which generates counterexamples against it. 

The original motivation of the \cite{Gutfreund05} was about average-case complexity.
Their result has a nice interpretation from the perspective of Impagliazzo's five worlds 
\cite{Impagliazzo95}, a classification of possible ``worlds'' we could live in 
depending on the truth or falsity of conjectures about average-case complexity 
and the existence of cryptographic primitives. One possible world, \emph{Algorithmica},
is the world where $\PTIME = \NP$ and therefore all $\NP$ problems are easy in the worst case.
Another possible world, \emph{Heuristica}, is the world where $\PTIME \neq \NP$ but all
$\NP$ problems are easy in the average case, that is, for every efficiently samplable input
distribution there exists a polynomial-time algorithm which solves the problem on inputs from
that distribution.
One could imagine an intermediate world, \emph{Super-Heuristica}, where the quantifiers are reversed
and we have one algorithm which works for all samplable distributions, despite having $\PTIME \neq \NP$. 
The results of \cite{Gutfreund05} imply that this world can not exist.

Another motivation for the results of \cite{Gutfreund05} and of subsequent work \cite{Bogdanov10} 
is more practical: provide hard examples for SAT solvers. SAT solvers, that is, heuristic algorithms
for SAT, are widely used in scientific and industrial applications.
The result of \cite{Gutfreund05} provides an algorithm for generating hard instances 
for those solvers, which may be useful for the purposes of benchmarking their performance. 

A recent article by Chen, Jin, Santhanam and Williams \cite{ConstructiveSeparations} 
provides motivation for constructivity of lower bounds as a ``barrier-style'' result:
many lower bounds that we know how to prove are non-constructive, but the lower bounds
that we would like to prove are constructive. More concretely, in \cite{ConstructiveSeparations}
they show that:

\begin{enumerate}
    \item Many important conjectured separations are constructive, including not only $\PTIME \neq \NP$ but
    also $\PTIME \neq \PSPACE$, $\BPP \neq \NEXP$ and $\ZPP \neq \EXP$.
    \item Many lower bounds that we know how to prove via non-constructive methods
    are difficult to make constructive: if a refuter algorithm were found, it would
    imply breakthrough results in complexity theory that we currently do not know how 
    to prove, including $\PTIME \neq \NP$!
\end{enumerate}

In the view of \cite{ConstructiveSeparations}, this is evidence that in order to make progress
in proving lower bounds, new techniques which are inherently ``algorithmic'' 
and constructive need to be developed. A similar observation was made by Mulmuley 
\cite{Mulmuley10} in the context of the ``Geometric Complexity Theory'' program 
to prove complexity lower bounds using algebraic geometry. 

\section{Overview}

The starting point of this work is the article \cite{ConstructiveSeparations}. In this work,
we expose some of its results and also provide some extensions, focusing on the question of
when can refuters be constructed for problems with known lower bounds.

In \cref{chp:cs}, we expose the contents of \cite{ConstructiveSeparations}, especially those
results which we will discuss later in the work, providing the necessary technical background 
for them. \todo{short survey of other results of refuters?}

In \cref{chp:qa}, we discuss refuters against query algorithms. 
Given an input $x \in \{0, 1\}^n$, we want to distinguish between the case where $x$ has at
least $(1/2 + \eps)n$ ones and the case where $x$ has at most $(1/2 - \eps)n$ ones. 
It is a standard exercise in probability to show that we can do so by randomly sampling $O(1/\eps^2)$
positions of the string $x$. It is also known, at least since \cite{Canetti95}, that such
random sampling is the best one can do under very general conditions. That is, there is no
algorithm that reliably distinguishes between the two cases using only $o(1/\eps^2)$ queries
to the string. In \cite{ConstructiveSeparations} it is shown that making this lower bound
constructive would imply $\PTIME \neq \NP$! We discuss this refuters for this
problem in the chapter, showing a 
randomized refuter, reproducing the proof of \cite{ConstructiveSeparations} and discussing
possible variations and strengthenings, and providing deterministic refuters for weaker algorithms. 

In \cref{chp:tm}, we discuss refuters against one-tape Turing machines. 
The original machine model defined by Turing \cite{Turing36} had only one tape, 
but in modern computational complexity the model of computation is usually taken
to be \emph{multi-tape} Turing machines, since one-tape TMs have some 
unrealistic limitations.
For example, they require $\Omega(n^2)$ time to regonize palindromes \cite{Hennie65}. 
In \cite{ConstructiveSeparations}, it is shown that making this lower bound constructive
would imply breakthrough circuit lower bounds. In the chapter, we discuss refuters
for the problem of palindromes, showing a randomized refuter, reproducing the proof of 
\cite{ConstructiveSeparations} with more generality, proving that in some refuters 
can not exist if we assume the existence of a certain cryptographic primitive, 
and providing deterministic refuters for weaker one-tape TMs. 


\section{Our Countributions}

This document contains both an exposition of the results of \cite{ConstructiveSeparations}
(and their background) and some original contributions. 

Our probabilistic refuters exposed in \cref{sec:lbqa} for query algorithms and in \cref{sec:lbtm}
for Turing Machines can be considered an adapted exposition of the results of 
\cite{Canetti95} and \cite{Hennie65}, respectively, since we just expose the proof of the lower
bound and then we make the (simple) observation that it can be turned into a probabilistic refuter.
This way of looking at these results relates the claims of obtaining breakthroughs via refuters
of \cite{ConstructiveSeparations} 
with the derandomization of these probabilistic refuters. 
Also, we do note that our adaptation of the results of \cite{Canetti95} has in some aspects more
generality than the original.

Our proof of the non-existence of refuters for one-tape TMs assuming the existence of 
cryptographic primitives follows directly from an argument in \cite{Oliveira24}. However,
the context of \cite{Oliveira24} is different (non-provability of lower bounds in
bounded arithmetic), and we believe the adaptation into the constructive separations context
is valuable since it shows that specific claims made in \cite{ConstructiveSeparations}
are vacuous assuming the existence of cryptographic primitives. We explain how to modify the 
claims to make them non-vacuous even if the relevant cryptographic primitive exists. 

The most novel part of this work is the construction of refuters for weak algorithms,
developed in \cref{sec:refuteragainstweakerqa} for query algorithms and in
\cref{sec:refuteragainstweakertm} for one-tape TMs. That is, even if \cite{ConstructiveSeparations}
shows that providing (deterministic) refuters against algorithms not satisfying the lower bound would
imply breakthrough results, we do exhibit refuters that work for weaker lower bounds 
(that is, for more stringent requirements on the number of queries or the time complexity
of the 1-TM for the algorithms being refuted). We develop a ``frontier'' between the scenarios 
where refuters can be constructed and the ones where proving their existence would imply 
a breakthrough result. Often, this frontier is quite tight in the sense that slightly 
relaxing one of the conditions stated in \cite{ConstructiveSeparations} in order to get 
a breakthrough result makes a refuter easily constructible. 
In other cases, though, some small gaps remain. 
We summarize those cases in the form of ``Open Questions'' that 
we believe are interesting avenues for further research.



\chapter{Constructive Separations}
\label{chp:cs}

\todowrite{Write Chapter 2}

\section{Basic Definitions}

Basic definition of constructive separation as in \cite{ConstructiveSeparations}.

\begin{definition}
	\label{def:refuter}
	For a function $f \colon {0, 1}^* \to {0, 1}$ and an algorithm $\A$, 
	a $\PTIME$-refuter for $f$ against $\A$ is a deterministic polynomial time
	algorithm $R$ that, given input $1^n$, prints a string $x \in \{0, 1\}^n$,
	such that for infinitely many $n$, $A(x) \neq f(x)$.
	
	A $\BPP$-refuter for $f$ against $\A$ is a randomized polynomial time algorithm
	$R$ that, given input $1^n$, prints a string $x \in \{0, 1\}^n$, such that 
	for infinitely many $n$, $A(x) \neq f(x)$ with probability at least $2/3$.

	A $\ZPP$-refuter for $f$ against $\A$ is a randomized polynomial time algorithm $R$
	that, given input $1^n$, prints $x \in \{0, 1\}^n \cup \{\blank\}$, such that for 
	infinitely many $n$, either $x = \blank$ or $A(x) \neq f(x)$, and $x \neq \blank$ with
	probability at least $2/3$. 
\end{definition}

\begin{definition}
	\label{def:constructiveseparation}
	For $\calD \in \{\PTIME, \BPP, \ZPP\}$ and a class of algorithms $\mathcal{C}$, we say there is a 
	$\calD$-constructive separation of $f \not\in \mathcal{C}$, if for every algorithm $\A$ computable
	in $\mathcal{C}$, there is a refuter for $f$ against $A$ that is computable in $\mathcal{D}$. 
\end{definition}

\section{Main Results of \cite{ConstructiveSeparations}}

Briefly state the results of \cite{ConstructiveSeparations}, pointing to further discussion in the next chapters. 

\subsection{Conjectured Separations Can Be Made Constructive}

Results about constructive separations for conjectured uniform separations.

\subsection{Constructive Separations for Known Results Imply Breakthroughs}

\subsubsection{Results}

State the results we will discuss in this thesis.

\begin{restatable}[Theorem 1.6 from \cite{ConstructiveSeparations}]{theorem}{thmcsrefuterqa}\label{thm:csrefuterqa} 
	Let $\eps$ be a function of $n$ satisfying $\eps \leq 1/(\log n)^{\omega(1)}$, and $1/\eps$ is a positive integer
     computable in $\poly(1/\eps)$ time given $n$ in binary.
	\begin{itemize}
		\item If there is a polylogtime-uniform-$\AC^0$-constructive separation of $\MAJ_{n,\eps}$ from randomized
         query algorithms $A$ using $o(1/\eps^2)$ queries and $\poly(1/\eps)$ time, then $\NP \ne \PTIME$.
		\item If there is a polylogtime-uniform-$\NC^1$-constructive separation of $\MAJ_{n,\eps}$ from randomized
         query algorithms $A$ using $o(1/\eps^2)$ queries and $\poly(1/\eps)$ time, then $\PSPACE \ne \PTIME$.
	\end{itemize}
\end{restatable}

\begin{restatable}[Theorem 3.4 from \cite{ConstructiveSeparations}]{theorem}{thmcsrefutertm}\label{thm:csrefutertm} 
The following hold:
	\begin{itemize}
	    \item If there is a $\PTIME^{\NP}$-constructive separation of $\PAL$
		from nondeterministic $O(n^{1.1})$ time one-tape Turing machines, then
		$\E^{\NP} \not \subset \SIZE[2^{\delta n}]$ for some constant $\delta >0$.
	    
        \item If there is a $\PTIME$-constructive separation of $\PAL$ 
		from nondeterministic $O(n^{1.1})$ time one-tape Turing machines, then
		 $\E \not \subset \SIZE[2^{\delta n}]$ for some constant $\delta >0$.
        
        \item If there is a $\LOGSPACE$-constructive separation of $\PAL$ 
		from nondeterministic $O(n^{1.1})$ time one-tape Turing machines, then
		 $\PSPACE \not \subset \SIZE[2^{\delta n}]$ for some constant $\delta >0$.
	\end{itemize}
\end{restatable}


\subsubsection{Lemmas About Circuit Complexity of Refuters}

%Explain and prove ``binary-to-unary'' lemmas, which are used in many of the proofs 
%and will be useful for next chapters. 

\todo{REVIEW THIS - ATSERIAS}

\begin{lemma}
	\label{lem:circuitbinarytounary}
	Let $T(n), s(n) \geq n$ be functions and assume $\DTIME(T(2^n)) \subseteq \SIZE(s(n))$. 
	Then, for every algorithm $R$ running in time $T(n)$ and outputting $n$ bits,
	the output of $R(1^n)$ can be computed by a circuit of size $s(2 \lceil \log n \rceil)$ 
	(which takes as input an index $0 \leq i < n$ and outputs the $i$-th bit of $R(1^n)$).
\end{lemma}
\begin{proof}
Consider the function $f_R(m, i)$ which, given $m$ and $i$ in binary, outputs the $i$-th bit of $R(1^m)$. The inputs to $f_R$ can 
be encoded in $2 \lceil \log m \rceil$ bits so that the input size is the same for each input $(m, i)$ with the same $m$.
$f_R \in \DTIME(T(2^N))$, so $f_R$ has circuits of size $s(N)$, where $N = 2 \lceil \log m \rceil$ is the input size. 
We get the desired result by considering, for each $n$, the circuit for $f_R$ with input size $2 \lceil \log n \rceil$
which we modify by fixing the first $\lceil \log n \rceil$ input 
bits to the value of $n$. 
\end{proof}

\section{Additional Aspects of Refuters}

Explain additional aspects such as list refuters, black-boxness/gray-boxness/explicit constructions, certifiability, ``minimum refuting size vs. minimum counterexample size'', etc.
Mention results of \cite{Atserias06} and \cite{Bogdanov10}. 



\chapter{Refuters for Query Algorithms}
\label{chp:qa}

\todowritehp{Finish writing Chapter 3}

In this chapter, we study constructive separations in the query algorithm setting.
In particular, we study the problem $\MAJ$ of distinguishing determining whether
a binary string $x \in \{0, 1\}^n$ has at least $(\frac{1}{2}+\eps)n$ ones or
at most $(\frac{1}{2}-\eps)n$ ones, under the promise that it is one of the two
cases. 

We will prove the $\Omega(1/\eps^2)$ lower bound in the number of queries
for $\MAJ$, show a randomized constructive separation against algorithms
using fewer queries, and prove and discuss \cref{thm:csrefuterqa}, the result
from \cite{ConstructiveSeparations} showing that breakthrough lower
bounds follow from sufficiently constructive separations of $\MAJ$.  


\section{Lower Bounds Against Query Algorithms for $\MAJ$}
\label{sec:lbqa}

The obvious algorithm for $\MAJ$ consists in randomly sampling values and answering 
according to the majority of the values we have obtained. According to well-studied
bounds on the binomial distribution, $\Theta(1/\eps^2)$ samples
are necessary and sufficient to make the probability of error smaller than any 
fixed constant following this strategy.

In this section, we will prove the very intuitive result that this is essentially
the best that can be done, that is, that there is no query algorithm making $o(1/\eps^2)$
queries that gives the right answer with small probability of error. This was 
proved in \cite{Canetti95} in the more general context of \emph{samplers}: algorithms
that compute the average of a real-valued function $f \colon \{0, 1\}^n \to [0, 1]$
by sampling some of the inputs. 

The result that we will prove here is stronger than the one stated in \cite{Canetti95}:
we will prove not just that such query algorithms require $\Omega(1/\eps^2)$ in the 
worst case in order to have a high success probability \emph{for all} inputs; instead,
we will prove that $\Omega(1/\eps^2)$ queries are required for \emph{random} inputs,
where the probability of success is over the randomness of both the input and the algorithm.
This will allow us to obtain a constructive separation.
%However, the proof argument 
%we use is similar to the one used in \cite{Canetti95}.

\subsection{Proof of Lower Bounds for Random Inputs}



\begin{definition}
A probability distribution $\calD$ on $\{0, 1\}^n$ has \emph{homogeneous
symmetry} if for every $y \in \{0, 1\}^*$ and every pair of subsets $S, S' \subset [n]$
with $|S| = |S'| = |y|$, 
$\Probsub{x \samplefrom \calD}{x[S] = y} = \Probsub{x \samplefrom \calD}{x[S'] = y}$.
That is, the probability $\Probsub{x \samplefrom \calD}{x[S] = y}$ does not depend
on the set $S$. Denote by $F_{\calD}(y)$ the probability
$\Probsub{x \samplefrom \D}{x[S] = y}$ for any $S$ with $|S| = |y|$.

A probability distribution $\calD$ on $\{0, 1\}^n$ has \emph{complementary symmetry} if for every
$y in \{0, 1\}^n$, $\Probsub{x \samplefrom \D}{x = y} = \Probsub{x \samplefrom \D}{x = \complementary{y}}$,
where $\complementary{y}$ is the complementary vector of $y$.

A probability distribution $\D$ on $\{0, 1\}^n$ is \emph{symmetric} if it has homogeneous symmetry
and it has complementary symmetry. 
\end{definition}

Fix $n$ and $\eps$, and let $X^+_\eps = \{x \in \{0, 1\}^n | w_1(x) \geq (\frac{1}{2}+\eps )n \}$
and $X^-_\eps = \{x \in \{0, 1\}^n | w_1(x) \leq (\frac{1}{2}-\eps )n \}$.
For $\calD$ a probability distribution on $\{0, 1\}^n$ for which
the events $x \in X^+_\eps$ and $x \in X^-_\eps$ have nonzero probability (where $x$
is the string of $n$ bits sampled by the distribution), 
denote by $\calD^+$ the probability distribution conditioned on $x \in X^+_\eps$,
and by $\calD^-$ the probability distribution conditioned on $x \in X^-_\eps$.

\todowrite{define w1, etc in introduction}

\begin{lemma}
\label{lem:symproperties}
Let $\calD$ be a distribution on $\{0, 1\}^n$ which has homogeneous symmetry.

\begin{enumerate}
    \item The probability $F_{\calD}(y)$ only depends on the number of zeros and ones
    of $y$: $F_{\calD}(y) = F_{\D}(0^{w_0(y)}1^{w_1(y)})$. We will denote by $f_\D(a, b)$ the value of $F_{\D}(0^a 1^b)$.
    \item If the events $x \in X^+_\eps$ and $x \in X^-_\eps$ have nonzero probability
    (i.e. $\D^+$ and $\D^-$ are well-defined),
    \todopolish{Find a way to phrase this without the awkward variable $x$?}
    the distributions $\D^+$ and $\D^-$ have homogeneous symmetry. 
    \item If $\D^+$ and $\D^-$ are well-defined and $a \leq b$, then $f_{\D^+}(b, a) \leq f_{\D^+}(a, b)$.
    \item If $\D$ has complementary symmetry and $\D^+$ and $\D^-$ are well-defined, then  
    $f_{\D^+}(a, b) = f_{\D^-}(b, a)$ for all
    $a, b \geq 0$.
\end{enumerate}
\end{lemma}
\begin{proof}
   Let us prove the first item. For $|y| = 1$ it is clear that
   the value of $F_{\D}(y)$ only depends the number of zeros and ones of $y$, since
   $y = 0$ or $y = 1$.  Now we want to prove that for every $y$ with $|y| \geq 2$,
   $F_\D(y) = F_\D(0^{w_0(y)}1^{w_1(y)})$. To prove that, it is enough to prove that we
   can perform ``swaps'' for any substring $10$ in $y$, that is, that if $y = y_0 \concat 10 \concat y_1$,
   then $F_\D(y_0 \concat 10 \concat y_1) = F_\D(y_0 \concat 01 \concat y_1)$. By a sequence of such swaps, we can always reach the 
   string $0^{w_0(y)}1^{w_1(y)}$. This is proved as follows:  
   \begin{align*}
   F_\D(y_0 \concat 10 \concat y_1) & = \Prob{x[S] = y_0 \concat 10 \concat y_1} \\
                                    & = \Prob{x[S] = y_0 \concat 10 \concat y_1} 
                                    + \Prob{x[S] = y_0 \concat 00 \concat y_1}
                                    - \Prob{x[S] = y_0 \concat 00 \concat y_1} \\
                                    & = \Prob{x[S \setminus \{i\}] = y_0 \concat 0 \concat y_1}
                                    - \Prob{x[S] = y_0 \concat 00 \concat y_1} \\
                                    & = \Prob{x[S \setminus \{j\}] = y_0 \concat 0 \concat y_1}
                                    - \Prob{x[S] = y_0 \concat 00 \concat y_1} \\
                                    & = \Prob{x[S] = y_0 \concat 01 \concat y_1} 
                                    + \Prob{x[S] = y_0 \concat 00 \concat y_1}
                                    - \Prob{x[S] = y_0 \concat 00 \concat y_1} \\
                                    & = F_\D(y_0 \concat 01 \concat y_1),
   \end{align*}
   where $i$ and $j$ are the indices of $S$ corresponding to the $10$ subsequence. 

Second item:
\begin{align*}
    \Probsub{x \samplefrom \D^+}{x[S] = y} & = \Probsub{x \samplefrom \D}{x[S] = y | x \in X^+_\eps} \\
        & = \sum_{\substack{z\in X^+_\eps \\ z[S] = y}} \Probsub{x \samplefrom \D}{x = z} \\
        & = \sum_{\substack{z\in X^+_\eps \\ z[S] = y}} F_{\D}(0^{w_0(z)}1^{w_1(z)})
        && (\text{first item on set } S = [n])\\
        & = \sum_{\substack{z\in X^+_\eps \\ z[S'] = y}} F_{\D}(0^{w_0(z)} 1^{w_1(z)}) \\
        & = \Probsub{x \samplefrom \D^+}{x[S'] = y}.
\end{align*}

Third item:
    \begin{align*}
        f_{\D^+}(b, a) & = \Probsub{x \samplefrom \D^+}{x[[a+b]] = 0^b1^a} \\
        & = \frac{1}{\binom{n}{a+b}\binom{a+b}{b}} \sum_{S \in \binom{[n]}{a+b}} 
        \sum_{\substack{y \in \{0, 1\}^{a+b}\\w_0(y)=b}} 
        \Probsub{x \samplefrom \D^+}{x[S] = y}
        && (\text{homog. sym.}) \\
        & = \frac{1}{\binom{n}{a+b}\binom{a+b}{b}} 
        \sum_{S \in \binom{[n]}{a+b}} 
        \sum_{\substack{y \in \{0, 1\}^{a+b}\\w_0(y)=b}} 
        \sum_{\substack{z\in X^+_\eps\\z[S]=y}} \Probsub{x \samplefrom \D^+}{x = z} \\
        & = \frac{1}{\binom{n}{a+b}\binom{a+b}{b}}
        \sum_{z \in X^+_\eps} \binom{w_0(z)}{b} \binom{w_1(z)}{a} \Probsub{x \samplefrom \D^+}{x = z}
        && (\text{swap sums}) \\
        & \leq \frac{1}{\binom{n}{a+b}\binom{a+b}{b}}
        \sum_{z \in X^+_\eps} \binom{w_0(z)}{a} \binom{w_1(z)}{b} \Probsub{x \samplefrom \D^+}{x = z}
        && (*) \\
        & = f_{\D^+}(a,b),
    \end{align*}
    where the inequality at $(*)$ uses that if $x \leq y$ and $a \leq b$, then 
    $\binom{x}{a} \binom{y}{b} \geq \binom{x}{b} \binom{y}{a}$.

Fourth item:
\begin{align*}
    f_{\D^+}(a, b) & = \Probsub{x \samplefrom \D}{x[[a+b]] = 0^a1^b | x \in X^+_\eps} \\
    & = \frac{1}{\Probsub{x \samplefrom \D}{x \in X^+_\eps}}
    \sum_{\substack{y \in X^+_\eps\\y[[a+b]] = 0^a1^b}} \Probsub{x \samplefrom \D}{x = y} \\
    & = \frac{1}{\Probsub{x \samplefrom \D}{x \in X^-_\eps}}
    \sum_{\substack{y \in X^-_\eps\\y[[a+b]] = 1^a0^b}} \Probsub{x \samplefrom \D}{x = y} 
    && (\text{comp. sym.})\\
    & = \Probsub{x \samplefrom \D}{x[[a+b]] = 1^a0^b | x \in X^-_\eps} = f_{\D^-}(b, a).
\end{align*}


\end{proof}



%\begin{lemma}
%\label{lem:comsymproperties}
%Let $\D$ be a symmetric distribution on $\{0, 1\}^n$ for which the distributions
%$\D^+$ and $\D^-$ are well-defined.
%Then:
%\begin{enumerate}
%\item

%\end{enumerate}
%\end{lemma}

%\begin{proof}
%\end{proof}


\begin{lemma}
\label{lem:symmetricbound}
Let $\calD$ be a symmetric distribution supported on $X^+_\eps \cup X^-_\eps$,
 and let $\A$ be a probabilistic query algorithm
that always does $t$ queries and solves $\MAJ$ with error probability $\delta$ 
on inputs sampled from $\calD$. 
Then:
$$
\sum_{i=0}^{\lceil t/2 \rceil-1} \binom{t}{i} f_{\D^+}(i, t-i) \leq \delta.
$$
\end{lemma}
\begin{proof}
Denote by $x$ the input, and let $Y = y_1 \ldots y_t$ denote the random variable corresponding
to the values of the $t$ sampled points. Note that $Y$ depends on $x$ and also on the randomness
over the execution of $Y_r$ be the random variable corresponding to the sampled points where
the randomness of $\A$ has been fixed to be $r$. $Y_r$ only depends on the choice of $x$.

Let $P^+ = \Prob{\A(x) = 0 | x \in X^+_\eps}$ the probability that the algorithm errs
conditioned on $x$ having a majority of ones, and similarly let 
$P^- = \Prob{\A(x) = 1 | x \in X^-_\eps}$ be the probability that the algorithm errs conditioned
on $x$ having a majority of zeros.

We have:

$$
P^+ = \sum_{y\in \{0, 1\}^t} \Prob{Y = y | x \in X^+_\eps} \Prob{\A(x) = 0 | Y = y, x \in X^+_\eps}.
$$

Now note the following:

\begin{itemize}
    \item Once $y$ is fixed, the result of $\A$ does not depend on $x$, but only on $y$ and $r$.
    Hence $\Probsub{x,r}{A(x) = 0 | Y = y, x \in X^+_\eps} = \Probsub{r}{\A(x) = 0 | Y = y}$.
    We denote this probability by $G_\A(y)$.
    \item Since $\D$ is symmetric, the event $Y = y$ does not depend on the
    randomness $r$ of the algorithm and in fact $\Prob{Y = y | x \in X^+_\eps} = F_{\D^+}(y)$.
    This is because, fixing some randomness $r$ in the algorithm, the event that the sequence
    of $t$ sample points is equal to $y$ is equivalent to the event that a subsequence of
    $x[S]$ of $x$ of size $t$ is equal to a particular permutation of $\sigma(y)$ of $y$. 
    By \cref{lem:symproperties} (1), this probability does not depend on $S$ or $\sigma$.
\end{itemize}

Thus, 

$$
P^+ = \sum_{y \in \{0, 1\}^t} F_{\D^+}(y) \cdot G_\A(y).
$$

Similarly:

$$
P^- = \sum_{y \in \{0, 1\}^t} F_{\D^-}(y) \cdot (1-G_\A(y)).
$$

Adding them and applying \cref{lem:symproperties}, we get the desired result:
\begin{align*}
\delta & \geq \frac{1}{2}\left(P^+ + P^-\right) \\
        & = \frac{1}{2}\sum_{y \in \{0, 1\}^t} F_{\D^+}(y) \cdot G_\A(y) + F_{\D^-}(y) \cdot (1-G_\A(y)) \\ 
        & \geq \frac{1}{2}\sum_{y \in \{0, 1\}^t} \min \{F_{\D^+}(y), F_{\D^-}(y)\} \\
        & = \frac{1}{2}\sum_{i=0}^{t} \binom{t}{i} \min \{f_{\D^+}(i, t-i), f_{\D^+}(t-i, i)\} 
        && (\text{\cref{lem:symproperties} (4)})\\
        & \geq \sum_{i=0}^{\lceil t/2 \rceil-1} \binom{t}{i} f_{\D^+}(i, t-i).
        && (\text{\cref{lem:symproperties} (3)})
\end{align*}

\end{proof}

\begin{theorem}
\label{thm:querylb}
\todo{conditions on eps}
Let $\A$ be a query algorithm that solves the $\MAJ$ problem on inputs uniformly drawn from 
$S_n = \{x \in \{0, 1\}^n | w_1(x) \in \{\lfloor(1/2-\eps)n\rfloor, \lceil(1/2+\eps)n\rceil\}\}$ 
with failure probability $\delta(n)$ using always at most $t(n)$ queries, with $t(n) \leq \frac{1}{4}\sqrt{n}$. 
There exists an absolute constant $C$ so that: 
$$
t(n) \geq \frac{1}{4\eps^2} \log \left(\frac{C}{\delta}\right)
$$

for all sufficiently large $n$.
\end{theorem}

\begin{proof}

First, we can assume that the algorithm always samples exactly $t(n)$ distinct points. Otherwise,
we can create an algorithm $A'$ which samples additional points before answerting
until exactly $t(n)$ points have been sampled.

For each $n$, the uniform distribution $U_{S_n}$ on $S_n$ is a symmetric distribution. 
Therefore, by \cref{lem:symmetricbound}, we have:

$$
\sum_{i=0}^{\lceil t/2 \rceil-1} \binom{t}{i} f_{U^+_{S_n}}(i, t-i) \leq \delta(n).
$$

Let $k = \lceil(1/2+\eps)n\rceil$. We can compute $f_{U^+_{S_n}}$:


\begin{equation}
\label{eq:fcalc}
f_{U^+_{S_n}}(w, t-w) = \frac{\binom{n-t}{k-w}}{\binom{n}{k}} = \frac{(n-t)!k!(n-k)!}{n!(k-w)!(((n-k)-(t-w))!}
    = \frac{(k)_w}{(n)_w} \frac{(n-k)_{t-w}}{(n-w)_{t-w}}
\end{equation}

Where $(n)_h = \frac{n!}{h!}$ is the falling factorial. We use the following elementary inequalities 
for the falling factorial (which follow from the inequality $1+x \leq e^x$):

$$
n^h e^{-\frac{h^2}{2(n-h)}} \leq (n)_h \leq n^h e^{-\frac{h(h-1)}{2n}}
$$

So we have:

$$
\frac{(k)_w}{(n)_w} \geq \frac{k^we^{-\frac{w^2}{2(n-w)}}}{n^we^{-\frac{w(w-1)}{2(n-w)}}} = \left(\frac{k}{n}\right)^w e^{-w\left(\frac{w}{2(k-w)}-\frac{w-1}{2n}\right)}
$$

Recall that $w \leq t \leq \frac{1}{4}\sqrt{n}$, and that $k > \frac{1}{2}n$. Using this, we can see that for sufficiently large $n$, we have that $e^{-w\left(\frac{w}{2(k-w)}-\frac{w-1}{2n}\right)} \geq 1/\sqrt{2}$, so:

$$
\frac{(k)_w}{(n)_w} \geq \frac{1}{\sqrt{2}}  \left(\frac{k}{n}\right)^w
$$

Similarly, for sufficiently large $n$ we have:

$$
\frac{(n-k)_{t-w}}{(n-w)_{t-w}} \geq \frac{1}{\sqrt{2}} \left(\frac{n-k}{n-w} \right)^{t-w} > \frac{1}{\sqrt{2}} \left(\frac{n-k}{n} \right)^{t-w}
$$

Substituting in \eqref{eq:fcalc}, we have:

$$
f_{U^+_{S_n}}(w, t-w)  \geq \frac{1}{2} \left(\frac{k}{n}\right)^w \left(\frac{n-k}{n} \right)^{t-w}
$$
    
So that:

\begin{align*}
\delta & \geq \sum_{i=0}^{\lceil t/2 \rceil-1} \binom{t}{i} f_{U^+_{S_n}}(i, t-i) \\
       & \geq \sum_{i=0}^{\lceil t/2 \rceil-1} \frac{1}{2} \binom{t}{i} \left(\frac{k}{n}\right)^i \left(\frac{n-k}{n} \right)^{t-i} \\
       & = \frac{1}{2} \Probsub{X \sim \text{Bin}(t, k/n)}{X < \left\lceil \frac{t}{2} \right\rceil}
\end{align*}

By standard bounds on the tail of the binomial distribution \cite{Feller43},
if $n$ is large enough and $\eps$ is small enough there exists a constant $C$ so that 
$\Probsub{X \sim \text{Bin}(t, k/n)}{X < \left\lceil \frac{t}{2} \right\rceil} \geq C e^{-4\eps^2t}$.
Rearranging, we get our desired result. 

\end{proof}

\begin{corollary}
Let $\eps(n)$ be a function satisfying $\eps(n) = \omega(n^{-1/4})$ and $\eps(n) = o(1)$. Then, all query
algorithms $\A$ solving $\MAJ_{n,\eps}$ with success probability $2/3$ for all inputs do at least $\Omega(1/\eps(n)^2)$
queries in the worst case. 
\end{corollary}
\begin{proof}
Consider the query algorithm $\A^*$ consisting in repeating algorithm $\A$  multiple times and 
taking the majority answer, where the number of repetitions is a constant such
that the failure probability of $\A^*$ is strictly less than 
the constant $C$ from \cref{thm:querylb}. 
Algorithm $\A^*$ in particular has failure probability $\delta < C$ on inputs 
uniformly drawn from $S_n$, so we can apply \cref{thm:querylb} to show that
$\A^*$ does $\Omega(1/\eps^2)$ queries in the worst case. Note that we need $\eps(n) = \omega(n^{-1/4})$
for this.
Since $\A^*$ repeats $\A$ a constant number of times, $\A$ does $\Omega(1/\eps^2)$ queries in the worst case. 
\end{proof}

\subsection{Randomized Constructive Separation}

Explain how to make a randomized constructive separation from above (amplify error probability).

\begin{theorem}
\label{thm:qarandomizedcs}
\todowrite{randomized constructive separation for qas}
\end{theorem}



\section{Sufficiently Constructive Separations Imply Lower Bounds}

Recall that in \cite{ConstructiveSeparations}, the following theorem is proved: 

\thmcsrefuterqa*

From the proof we can get a stronger statement, with several improvements that are 
relevant for studying the best possible refuters in this setting:

\begin{itemize}
    \item The refuters can be quasipolynomial-size circuits (that is, $\qAC^0$ and 
    $\qNC^1$ rather than $\AC^0$ and $\NC^1$).
    \item The refuters can be quasipolynomial-length list-refuters. 
    \item It is only necessary to refute algorithms making $O(1/\eps^{1+\delta})$ queries,
    for some $\delta > 0$.
    \item It is only necessary to refute algorithms which are uniform distributions of
    $k$-juntas (not general randomized query algorithms).
\end{itemize}

\todo{Review this after definitions of refuters}

\begin{theorem}
    Let $\eps$ be a function of $n$ satisfying $\eps \leq 1/(\log n)^{\omega(1)}$, and $1/\eps$ is a positive integer
    computable in $\poly(1/\eps)$ time given $n$ in binary. Let $\delta > 0$, and let $t = 1/\eps^{1+\delta}$.
	\begin{itemize}
		\item If for every uniform distribution of $t$-juntas running in $\poly(1/\eps)$ time there exist
        a polylogtime-uniform-$\qAC^0$ quasipolynomial-length list-refuter $R$ of $\MAJ_{n, \eps}$ against 
        $\A$, then $\NP \neq \PTIME$. 
		\item If for every uniform distribution of $t$-juntas running in $\poly(1/\eps)$ time there exist
        a polylogtime-uniform-$\qNC^1$-uniform quasipolynomial-length list-refuter $R$ of $\MAJ_{n, \eps}$ against 
        $\A$, then $\PSPACE \neq \PTIME$. 
	\end{itemize}
\end{theorem}

\begin{proof}
    We will prove only the first item. The proof of the second item is analogous using the 
    corresponding second item from \cref{lem:acbinarytounary}.
    Assuming $\PTIME = \NP$, we will construct an algorithm $\A$ running in $\poly(1/\eps)$ time
    which samples $t$ points uniformly at random and solves the $\MAJ_{n, \eps}$ on the
    inputs generated by any polylogtime-uniform-$\qAC^0$ list-refuter.

    At the beginning, $\A$ computes $\eps$ in $\poly(1/\eps)$ time. 
    From \cref{lem:acbinarytounary} \todo{details about how list refuter works here}

    Therefore, there is a circuit of size $(c \log n)^c$ for some constant $c$ which computes
    the input given the index $i$ in binary. Since $\eps \leq 1/(\log n)^{\omega(1)}$,
    $(c \log n)^c \leq 1/\eps^{\delta/2}$ for sufficiently large $n$. 
    The number of circuits of size at most $1/\eps^{\delta/2}$ is $2^{O(\eps^{-\delta/2} \log \eps^{-1})}$.
    Therefore, this circuit can be PAC-learned with error $\eps/2$ and failure probability 
    $p = 1/6$ using $O(\eps^{-1} \cdot (\eps^{-\delta/2} \log \eps^{-1} + \log p^{-1})) \leq O(\eps^{-(1+\delta)})$
    random samples \cite[Theorem 2.5]{Mohri18}. Given that $\PTIME = \NP$, the learning of the circuit 
    $C$ can be done in $\poly(t, \log n) = \poly(1/\eps)$ time. Let $D$ be the resulting circuit.

    We decide $\MAJ_{n, \eps/2}$ on the output of $D$ by sampling $\Theta(1/\eps^2)$ uniformly random
    inputs from $D$, so that the probability of success is at least $5/6$, and return the result 
    as our answer. This takes $\poly(1/\eps)$ time. 
    Since the circuit $D$ only has error $\eps/2$ 
    compared with the original input string, the answer of $\MAJ_{n, \eps}$ on the original input
    string is the same as the answer of $\MAJ_{n, \eps/2}$ on the output of $D$.
    
    In total, our algorithm has success probability $2/3$, time complexity $\poly(1/\eps)$, and 
    sample complexity $O(1/\eps^{1+\delta})$. 
\end{proof}

\section{Constructive Separations Against Weaker Query Algorithms}
\label{sec:refuteragainstweakerqa}

In \cref{thm:csrefuterqa}, there are two important quantitative assumptions
that must hold in order to obtain a lower bound:

\begin{enumerate}
    \item We must have a refuter against all $t$-juntas with $t = O(1 / \eps^{1+\delta})$ for
    some $\delta > 0$.
    \item The function $\eps(n)$ must go to zero sufficiently fast: $\eps(n) = 1/(\log n)^{\omega(1)}$.
\end{enumerate}

We can wonder whether those conditions are \emph{tight}, in the sense that if we relax them slightly 
then we can have a constructive separation. We will show that the answer is positive for both of
those conditions. 

\subsection{Refuters for $t \leq C/\eps$}

In \cref{thm:csrefuterqa}, we need a refuter against $t$-juntas for $t = \Omega(1/\eps^{1+\delta})$ for any $\delta > 0$ in order 
to obtain a breakthrough. This condition is tight in the sense that we can easily construct list-refuters
for any query algorithms making less than
$C/\eps$ queries (for some $C$ depending on the probability gap of the algorithms).

\begin{theorem}
    \label{thm:refuterlessqueries}
Let $\eps(n)$ be a function so that $\eps(n) = o(1)$, $\eps(n) > 1/n$ and
 the integer $1/\eps$ is computable in polynomial time in $n$. 
There exists a polynomial-time 
algorithm $R$ that, on input $1^n$,
outputs a list of $O(1/\eps)$ strings of length $n$, 
so that every query algorithm using $\frac{1}{16\eps}$ queries
fails on one of the strings in the list with probability at least $1/3$.
\end{theorem}
\begin{proof}
The algorithm $R$ outputs $1 + \left\lfloor \frac{n}{4 \lceil \eps n \rceil}\right\rfloor$ strings.

The first string consist of $\lfloor (1/2 - \eps)n \rfloor$ ones, followed by zeros. 
The $i$-th next string ($1 \leq i \leq \left\lfloor\frac{n}{4 \lceil \eps n \rceil}\right\rfloor$)
consists of $\lfloor (1/2-\eps)n \rfloor$ ones followed by zeros in all positions except for 
the block from position
$\lfloor (1/2)n \rfloor +  (i-1) \cdot 2 \lceil \eps n \rceil$ to 
$\lfloor (1/2)n \rfloor +  i \cdot 2 \lceil \eps n \rceil - 1$.
That is, each string has a block of $2\lceil \eps n \rceil$ consecutive ones on its right half,
disjoint with the blocks of all other strings. 

Let $\A$ be any query algorithm as in the statement, and consider its execution on the first string.
With any fixed randomness $r$ for which $\A$ answers correctly on that string, 
it will answer incorrectly on at least half of the remaining 
$\left\lfloor \frac{n}{4 \lceil \eps n \rceil}\right\rfloor < \frac{1}{8\eps}$ strings,
since $\A$ can hit at most $\frac{1}{16\eps}$ distinct blocks with its queries.
Therefore, if $\A$ answers correctly on the first string with probability at least $2/3$, 
it will answer incorrectly with probability at least $1/3$ on at least one of the other strings.  

\end{proof}

Note that this is an ``explicit obstructions'' result: the list refuter does not depend on the algorithm being refuted. 
Note also that the refuter works for any query algorithm, not just juntas, and also
the refuter is quite simple and therefore can be implemented in restricted circuit models if
they allow computation of $1/\eps$ and simple arithmetic operations with it. 

\subsection{Derandomization of Query Algorithms}

One possible approach to obtaining deterministic constructive
separations is to try to derandomize 
the constructive separation of \cref{thm:qarandomizedcs}.
Pseudorandom generators fooling query algorithms are well-studied
\cite{Hatami2023}, and they will allow us to obtain a quasipolynomial
list-refuter when $\eps = 1/(\log(n))^{O(1)}$.

\begin{definition}
A set $X_1, \ldots, X_n$ of random variables is $k$-wise independent if for any $I \subseteq [n]$
with $|I| = k$ and any values $\{x_i\}_{i\in I}$, we have that the events $\{X_i = x_i\}_{i \in I}$ are independent.

We say that a probability distribution $\D$ on $\{0, 1\}^n$ is $k$-wise independent if the $n$ random variables 
corresponding to each of the bits are $k$-wise independent. 

We say that a probability distribution $\D$ on $\{0, 1\}^n$ is $k$-wise independent probability $p$ Bernoulli
if it is $k$-wise independent and $\Prob{x_i = 1} = p$ for all $i =1 , \dots, n$. 
\end{definition}

It is clear that $k$-juntas can not distinguish between $k$-wise independent Bernoulli distributions 
and $n$ independent Bernoulli random variables with the same probability. Interestingly, 
these distributions also fool any query algorithm using $k$ queries. 

\todowrite{Definitions of query algorithms in introduction}

\begin{theorem}
Let $\A$ be a query algorithm using at most $k$ queries on $\{0, 1\}^n$, 
let $\D$ be the distribution of $n$ independent 
Bernoulli variables with probability $p$ on $\{0, 1\}^n$, and let
$\D'$ be a $k$-wise independent probability $p$ Bernoulli distribution on $\{0, 1\}^n$. Then:
$$
\Probsub{x \samplefrom \D}{\A(x) = 1} = \Probsub{x \samplefrom \D'}{\A(x) = 1}.
$$
\end{theorem}
\begin{proof}
Note that it is sufficient to prove the result assuming $\A$ is a (deterministic) depth-$k$ decision tree,
since if $\A$ is probabilistic, the result for each of the decision trees resulting from each fixed randomness
implies the desired result for the probabilistic algorithm.

Let $\mathcal{L}$ be the set of leaves of the decision tree, and for each $u \in \mathcal{L}$,
let $\A_u(x)$ be equal
to $1$ if $\A(x) = 1$ and $\A$ ends at leaf $u$ of the tree on input $x$, and $0$ otherwise.
Note that $\A_u(x)$ is a $k$-junta
and that $\A(x) = \sum_{u\in \mathcal{L}} \A_u(x)$. Therefore:
$$
\EVsub{x \samplefrom \D}{\A(x)} = \sum_{u \in \mathcal{L}} \EVsub{x \samplefrom \D}{\A_u(x)} = \sum_{u \in \mathcal{L}} \EVsub{x \samplefrom \D'}{\A_u(x)} = \EVsub{x \samplefrom \D'}{\A(x)},
$$
and we have the desired result since  $\Prob{\A(x) = 1} = \EV{\A(x)}$,
because $\A(x)$ takes values $0$ and $1$.
\end{proof}

Now, we show that there is an efficient pseudorandom generator which samples from a $k$-wise
independent distribution with seed length $O(k \log n)$.

\begin{theorem}
\label{thm:kwiseindepgen}
There is a polynomial-time algorithm $\A$ which, given as input integers $n$, $k$ with $n \geq k \geq 1$,
an integer $p$ with $0 \leq p \leq 2^{\lceil \log_2 n \rceil}$, and $s = k \lceil \log_2 n \rceil$
random bits, outputs $n$ bits with a $k$-wise independent probability $p/2^{\lceil \log_2 n \rceil}$ 
Bernoulli distribution.
\end{theorem}
\begin{proof}
Let $q = 2^{\lceil \log_2 n \rceil}$, and let $\F_q$ be the finite field with $q$ elements. 

Let $\mathcal{P}_{k-1}$ be the set of univariate polynomials of degree at most $k-1$ in $\F_q$,
and let $z_1, \ldots, z_n \in \F_q$ be distinct elements of the field. Define the function 
$G \colon \mathcal{P}_{k-1} \to \F^n_q$ by
$$
G(p) = (p(z_1), \ldots, p(z_n)).
$$
We claim that, when $p$ is sampled uniformly at random from $\mathcal{P}_{k-1}$, the distribution
of $G(p)$ is $k$-wise independent. This is because, for all $I \subset [n]$ with $|I| \leq k$ and 
all sequences of values $\{x_i\}_{i\in I}$ in $\F_q$, the number of $p \in \mathcal{P}_{k-1}$ so that 
$p(z_i) = x_i$ for all $i \in I$ is exactly $q^{k-|I|}$, since each polynomial $p \in \mathcal{P}_{k-1}$
is uniquely determined by its evaluation at $k$ distinct points and conversely each possible sequence 
of values at $k$ distinct points yields a unique polynomial. Thus:
$$
\Prob{\bigwedge_{i\in I} p(z_i) = x_i} = \frac{1}{q^{|I|}} = \prod_{i\in I} \Prob{p(z_i) = x_i},
$$
and therefore the $|I|$ events $p(z_i) = x_i$ are independent, as desired. 

Now we describe the generator $\A$. We interpret the seed as the encoding of $k$ elements of $\F_q$,
determining a polynomial $p \in \mathcal{P}_{k-1}$, and we set the output bit to be $1$ if 
$f(p(z_i)) \leq p$ and to be $0$ otherwise, where $f$ is some bijection from $\F_q$ to $\{1, \ldots, q\}$.
Therefore, each bit has probability $p/q$ of being equal to $1$, and the $k$-wise independence of the
distribution follows from the $k$-wise independence of $G(p)$. 
\end{proof}

And finally, we can use this pseudorandom generator to derandomize \cref{thm:qarandomizedcs}.
We do not prove all of the details, since the argument is similar to that of \cref{thm:querylb}
but with a different probability distribution and with some additional complications. 

\begin{theorem}
    \label{thm:refutergreatereps}
Let $\eps = \eps(n) = 1/(\log n)^{O(1)}$ and let $\delta > 0$ be a sufficiently small constant.
\todopolish{Clarify value of delta like in previous results}
There exists a quasipolynomial-time quasipolynomial-list-refuter
$R$ for $\MAJ_{n,\eps}$ against all query algorithms using $t(n) = o(1/\eps^2)$ queries with probability gap bounded by $1-2\delta$.
\end{theorem}
\begin{proofsketch}
The refuter is as follows: 
On input $1^n$, set $k = 1/\eps^2$, $q = 2^{\lceil \log_2 n \rceil}$ and run the $k$-wise independent generator from \cref{thm:kwiseindepgen}
for all of the $2^{O(k \log n)} = 2^{\polylog(n)}$ seeds for each of 
$p = p_+ = \left\lceil q \cdot (1/2 + 2\eps) \right\rceil$ and
$p = p_- = \left\lfloor q \cdot (1/2 - 2\eps) \right\rfloor$. 
Output all the results which satisfy the at least $(1/2+\eps)n$ ones or at most $(1/2-\eps)n$ ones promise. 

Consider the uniform probability distribution on the outputs of the refuter $\D_1$. 
This distribution is statistically close to the uniform distribution $\D_2$ on all the outputs
of the generator, without filtering those that do not satisfy the promise, since there are 
very few that do not satisfy the promise 
(a fraction of around $\Probsub{X \sim \text{Bin}(n, p_+/q)}{X < (1/2+\eps)n}$).
In turn, distribution $\D_2$ is indistinguishable by algorithms making $t(n)$ queries
from the corresponding distribution $\D_3$ in which either all $n$ bits are sampled from
independent Bernoullis with probability $p_+/q$ or all $n$ bits are sampled from Bernoullis with
probability $p_-/q$, each of the two options with probability $1/2$. 
And finally, distribution $\D_3$ is statistically close to distribution $\D_4$, which is distribution 
$\D_3$ conditioned on the output satisfying the promise. We conclude that query algorithms using $t(n)$
queries can not distinguish between distributions $\D_1$ and $\D_4$ except with some small probability
(exponentially small in $n$, in fact). \todowrite{Verify this claim.}

Distribution $\D_4$ is a symmetric distribution on $X^+_\eps \cup X^-_\eps$, 
and with similar arguments as in \cref{thm:querylb} 
we can prove that query algorithms with error probability $2\delta$ with inputs 
over distribution $\D_4$ require $\Omega(1/\eps^2)$ queries. Therefore, algorithms
with probability gap bounded by $1-2\delta$ must fail at least on a fraction $\delta$ of
inputs from $\D_4$. Therefore, they must also fail at a constant fraction of inputs from $\D_1$
and in particular they will fail for at least one of the elements outputted by $R(1^n)$, for $n$ big enough.

\end{proofsketch}

\todowrite{Write conclusion relating it to observations of the previous section.}

\subsection{Limits of Explicit Obstructions}

One salient aspect of the list-refuters described 
in the two previous subsections is that they are explicit
obstructions: they do not depend on the algorithm being refuted. 
This is much more restrictive than the requirement to obtain
a breakthrough in \cref{thm:csrefuterqa}, in which the refuters are allowed to
depend on the algorithm being refuted. The following results displays
the limits of explicit obstructions for this problem:

\begin{theorem}
    \label{thm:explicitlimits}
Let $R$ be an algorithm that, on input $1^n$, outputs a list $L = L_n$ of strings of size $n$ in time $\poly(|L|)$.
There is a probabilistic query algorithm $\A$ running in time $\poly(n, |L|, 1/\eps)$ and making $O\left(\frac{\log (|L|) \cdot \log (\log (|L|))}{\eps}\right)$ queries
which correctly solves $\MAJ_{n, \eps}$ with probability at least $2/3$ on each of the outputs produced by $R$.
\end{theorem}
\begin{proof}

The algorithm $\A$ is described in \cref{alg:explicitlimits}. First, $\A$ computes $L$ and divides it into
$L^+$ and $L^-$, depending on whether the majority is $1$ or $0$. $L^+$ and $L^-$ will mantain at all times the elements
of the list that are consistent with the information obtained by the queries. If at any point
$L^+$ or $L^-$ becomes empty, the answer is determined and the algorithm can output it and stop. 

\begin{algorithm}
\caption{Algorithm for \cref{thm:explicitlimits}}\label{alg:explicitlimits}
\begin{algorithmic}
\Input $n$
\Program
\State $L \gets R(1^n)$
\State $(L^+, L^-) \gets \Call{Filter}{L}$ \Comment{$L^+$ are the elements of $L$ with $\geq (1/2+\eps)$ ones, $L^-$ the elements with $\leq (1/2-\eps)n$ ones}
\While{$L^+ \neq \emptyset$ and $L^- \neq \emptyset$}
    \State $m^+ \gets \Call{Majority}{L^+}$
    \State $m^- \gets \Call{Majority}{L^-}$
    \If{$m^+ \neq m^-$}
        \State $i \gets $ index such that $m^+_i \neq m^-_i$.
        \State $v \gets \Call{Query}{i}$
        \State $(L^+, L^-) \gets \Call{Update}{i, v, (L^+, L^-)}$ \Comment{Returns elements from the lists consistent with $x[i] = v$}
        %\State $L^- \gets \Call{Update}(i, v, L^-)$
    \Else
        \State $S_0 \gets \{i : m^+_i = 0\}$
        \State $S_1 \gets \{i : m^+_i = 1\}$
        \State $c \gets \operatorname*{argmax}_{i\in\{0, 1\}} |S_i|$
        \State $\texttt{found} \gets \texttt{false}$
        \RepeatTimes{$\lceil (\log (6 \log |L|)) \cdot 1/\eps \rceil$}
            \State $i \gets \Call{RandomElement}{S_c}$
            \State $v \gets \Call{Query}{i}$
            \State $(L^+, L^-) \gets \Call{Update}{i, v, (L^+, L^-)}$ 
            \If{$v \neq c$}
                %\State $L^- \gets \Call{Update}(i, v, L^-)$
                \State $\texttt{found} \gets \texttt{true}$
            \EndIf
        \EndRepeatTimes
        \If{not $\texttt{found}$} 
            \Output $c$
            \Halt
        \EndIf
    \EndIf
\EndWhile
\Output $0$ if $L^+ = \emptyset$, $1$ if $L^- = \emptyset$
\Halt
\end{algorithmic}
\end{algorithm}
    

$\A$ computes $m^+$ and $m^-$, the bitwise majority of all the elements of $L^+$ and $L^-$, respectively.
That is, for each $i = 1, \ldots, n$, $m^+_i$ is equal to $1$ if there are more elements of $L^+$ with $i$-th bit
equal to $1$ than elements with $i$-th bit equal to $0$, and is equal to $0$ otherwise. If $m^+ \neq m^-$, $\A$
queries one of the bits where $m^+$ and $m^-$ differ, discards the elements of $L^+$ and $L^-$ inconsistent with 
this information and starts over again. Note that this will result in halving the size of $L^+$ or $L^-$. 

If $m^+ = m^-$, then $\A$ determines the sets of indices where both $L^+$ and $L^-$ have majority $0$ 
and where they both have majority $1$ and samples $\lceil (\log (6 \log |L|)) \cdot 1/\eps \rceil$ random points
from the biggest set of those two. If all the sampled points coincide with the majority value on that 
set, $\A$ outputs the answer corresponding to that value. Otherwise, $\A$ starts over again; note that in
this case, the sizes of both $L^+$ and $L^-$ are halved.  

We proceed to the analysis of the algorithm. We have to argue that it performs $O(\log |L| \log \log |L| / \eps)$ queries, and
that it answers correctly with probability at least $2/3$. The bound on the number of queries is clear: 
the algorithm performs at most $O(\log \log |L| /\eps)$ queries in each iteration of the main loop
and it does at most $2 \cdot \log |L|$ iterations, since in each iteration the size 
of $L^+$ or $L^-$ is halved and it exits the loop when one of them becomes empty.

Suppose that we have an $x$ with $\geq (1/2 + \eps)$ ones, and we will bound the probability that $\A$ outputs $0$ (the
case where $x$ has majority $0$ is symmetric).
It is clear that if $\A$ exits the loop because $L^-$ has become empty, then it always gives the correct answer.
Then, $\A$ can only give a wrong answer when it outputs $0$ after sampling $\log (6 \log |L|)/\eps$ points in $S_0$
and finding that they are all equal to $0$. 
Note that, if $|S_0| \geq 1/2$, then $x[S_0]$ must have
at least $\eps n$ bits equal to $1$, and therefore the probability of sampling a bit equal to $1$ is at least $\eps$.
Therefore, the probability that all sampled bits are equal to $0$ is at most 
$(1-\eps)^{\log (6 \log |L|)/\eps} \leq \frac{1}{6 \log |L|}$. 
Since the algorithm performs at most $2 \log |L|$ iterations,
by the union bound the probability that $\A$ outputs $0$
is at most $2 \log |L| \cdot \frac{1}{6 \log |L|} = \frac{1}{3}$, as desired. 



\end{proof}

%Note that if $\eps = 1/(\log n)^{\omega(1)}$ and $|L|$ is quasipolynomial, then $O(1/\eps + \log |L|) = O(1/\eps)$,
%so the explicit obstructions refuter of \cref{thm:refuterlessqueries} is essentially optimal in this setting,
%and for algorithms that make $\Omega(1/\eps)$ queries running in time $\poly(n, |L|, 1/\eps)$ there are no explicit obstructions. 

\cref{thm:explicitlimits} shows that any constructive separation for a class of query algorithms including the algorithm described in 
the statement must necessarily have refuters that depend on the algorithm being refuted. 
Note that if $\eps = 1/(\log n)^{\omega(1)}$ and $|L|$ is quasipolynomial, then $O((\log |L| \log \log |L|)/\eps) = o(1/\eps^{1+\delta})$ for all 
$\delta > 0$. 
This suggests that that refuters that would achieve the breakthroughs of \cref{thm:csrefuterqa},
if they exist, should depend on the algorithm being refuted.
However, there are some details that could potentially make this not be the case: recall that
in \cref{thm:csrefuterqa} we only require refuters against uniform distributions of $k$-juntas
running in time $\poly(1/\eps)$,
while in \cref{thm:explicitlimits} the algorithm is an adaptive
query algorithm running in time $\poly(n, |L|, 1/\eps)$. 

There seems to be some room for improvement in \cref{thm:explicitlimits}.
There is a $\log (1/\eps) \log \log (1/\eps)$ gap between the number of 
queries in the algorithms refuted by the explicit obstructions of \cref{thm:refuterlessqueries}
and the number of queries in the 
algorithm $\A$ of \cref{thm:explicitlimits}.
The repeated sampling of $O(\log \log |L|/\eps)$ points seems to be somewhat wasteful, and 
perhaps with a different approach we could be able to get an $\A$ which makes e.g. $O(1/\eps + \log |L|)$ queries, 
which would close this gap.
It also seems feasible to eliminate the adaptivity of the queries by allowing $m^+$ and $m^-$
to differ in a small number of bits.  

\begin{question}
What is the precise frontier of explicit obstructions for $\MAJ_{n, \eps}$? Can we improve \cref{thm:explicitlimits} or \cref{thm:refuterlessqueries}?
\end{question}

\chapter{Refuters for One-Tape Turing Machines}
\label{chp:tm}


In this chapter, we will study constructive separations of the language of palindromes
$$\PAL = \{ww^R : w \in \{0, 1\}^*\} \cup \{wbw^R : w \in \{0, 1\}^*, b \in \{0, 1\}\}$$
from one-tape Turing machines running in time $o(n^2)$. In \cite{ConstructiveSeparations}
it was proved that constructive separations against nondeterministic machines yielded
breakthrough results in circuit lower bounds (recall \cref{thm:csrefutertm}). In this chapter
we will explore this setting more deeply, and explain when we can get circuit lower bounds, 
when we can actually get constructive separations, and even when we can prove that no 
constructive separations can exist. 

For simplicity, through the chapter we will often only consider the language 
of \emph{even} palindromes $\{ww^R : w \in \{0, 1\}^*\}$ in the proofs. 
Recall that in \cref{def:refuter} we only require
that the refuter works
for infinitely many values of $n$, so in particular a refuter that only works for
even palindromes and ignores the odd case is valid. However, the arguments that
we make for the even case in this chapter can always be modified slightly to 
work for the odd case too. We omit the explanations of which particular changes 
would be necessary for each of our results for brevity, since they are rarely
illuminating. 

%and, as we have said, it is often not necessary to consider the 
%odd case to reach the desired conclusion. 


\section{Lower Bounds Against 1-TMs for $\PAL$}
\label{sec:lbtm}

In this section, we prove the $\Omega(n^2)$ lower bound for 1-TMs computing $\PAL$,
due to Hennie \cite{Hennie65}. The proof of this fact lends itself to a simple
\emph{probabilistic} constructive separation that we also discuss. 

\subsection{Crossing Sequences and Lower Bounds}

\todowrite{explain indexing wrt to how input is written in the tape in the introduction}

\begin{definition}
Let $M$ be a 1-TM, and let $E = ((q_1, p_1), (q_2, p_2), \dots, (q_t, p_t))$, 
with $(q_i, p_i) \in Q \times \mathbb{Z}$, be a sequence describing an execution of
$M$, that is, $E$ is the sequence of states and positions of the tape head of $M$ when
it is executed on some input and (possibly) with some given non-deterministic choices.
The \emph{crossing sequence at position $i$}, denoted by $\CS_i(E)$,
is the sequence of states for 
which the tape head crosses from position $i$ to position $i+1$ or vice versa; that is, the
sequence of states $q_j$
such that $p_{j-1} = i$ and $p_{j} = i+1$ or $p_{j-1} = i+1$ and $p_j = i$. 

If $M$ is a deterministic 1-TM and $x \in \{0, 1\}^*$, we denote by $\CS_{M,i}(x)$ 
the crossing sequence at position $i$ of the execution of $M$ on input $x$. 

If $M$ is a non-deterministic 1-TM, $x \in \{0, 1\}^*$ and $b \in \{0, 1\}^*$ is a 
sequence of non-deterministic choices for the execution of $M$ with input $x$,
we denote by $\CS_{M,i}(x, b)$ the crossing sequence at position $i$ of the execution of $M$ on
input $x$ with non-deterministic choices $b$.

%If $M$ is a non-deterministic 1-TM and $x \in \{0, 1\}^*$ is a word accepted by $M$, we
%will denote by $\CS_i(x)$ the crossing sequence at position $i$ of the lexicographically smallest
%accepting execution of $M$ on $x$. 
\end{definition}

\begin{figure}
\label{fig:crossingsequences}
\missingfigure{crossing sequences}
\end{figure}

When the 1-TM $M$ is clear from context, we will usually write $\CS_i(x)$ instead of $\CS_{M,i}(x)$.
Also, for a non-deterministic 1-TM $M$ and an accepted word $x$, 
we will usually write just $\CS_i(x)$, omitting the description of 
the non-deterministic choices, to refer to the crossing sequence at position $i$ on input $x$
for \emph{some} accepting execution of $x$. That is, for each accepted word $x$, we arbitrarily choose
some accepting execution (e.g. the lexicographically smallest one), and use $\CS_i(x)$ to refer to the
crossing sequences of that execution. This is because, for the purposes of proving lower bounds for $\PAL$,
the fact that there might be multiple executions (and therefore multiple crossing sequences) for each accepted
word does not hurt, and we can prove the lower bounds only assuming that each accepted word has \emph{at least}
one accepting execution.

Intuitively, crossing sequences are a way to quantify the way a machine ``carries information'' from one
side of the boundary to the other one via its internal states. Note that, in a crossing sequence at position
$i \geq 1$, each odd crossing is from left to right
and each even crossing is from right to left. 
See \cref{fig:crossingsequences} for a pictorial depiction of 
crossing sequences.

\begin{lemma}
    \label{lem:pastingcs}
    Let $a = a_1\ldots a_i a_{i+1} \ldots a_n$ and $b = b_1 \ldots b_i b_{i+1} \ldots b_m$ 
    be two words accepted by a 1-TM $M$ so that $\CS_i(a) = \CS_i(b)$.
    Then the word  $c = a_1\ldots a_i b_{i+1} \ldots b_m$ is also accepted by $M$.
\end{lemma}
\begin{proof}
Let $E^a = ((q^a_1, p^a_1), \ldots (q^a_{t^a}, p^a_{t^a}))$ be the accepting execution for $a$ in $M$ and let
$E^b = ((q^b_1, p^b_1), \ldots (q^a_{t^b}, p'_{t^b}))$ be the accepting execution for $b$. 
Let $j^a_1, j^a_2, \ldots, j^a_{|\CS_i(a)|}$ and $j^b_1, j^b_2, \ldots, j^b_{|\CS_i(a)|}$ be the indices
corresponding to the states in $\CS_i(a)$ in $E^a$ and $E^b$, respectively.

We can define a new execution $E^c$ by reproducing the execution $E^a$ up to index $j^a_1$, then the execution
$E^b$ between indices $j^b_1$ and $j^b_2$, and so on:
\begin{multline*}
E^c = ((q^a_1, p^a_1), \ldots, (q^a_{j^a_1-1}, i), (q^b_{j^b_1}, i+1), (q^b_{j^b_1+1}, p^b_{j^b_1+1}), \ldots, \\
(q^b_{j^b_2-1}, i+1), (q^a_{j^a_2}, i), (q^a_{j^a_2+1}, p^a_{j^a_2+1}), \ldots ).
\end{multline*}
See \cref{fig:pastingcs} for an illustration of execution $E^c$. It is not difficult to see that $E^c$ is a 
valid accepting execution for $M$ on input $c$, since the machine behaves as if it were reading input $a$ to the left
of index $i$ and as if it were reading input $b$ to the right of index $i$. 

\end{proof}

\begin{figure}
    \label{fig:pastingcs}
    \missingfigure{pasting together different executions}
    \caption{Illustration of \cref{lem:pastingcs}.}
\end{figure}

\begin{corollary}
    \label{cor:crosscollision}
    Let $M$ be a 1-TM that accepts two palindromes $w_1 w_2 w_2^R w_1^R$ and
    $w_3 w_4 w_4^R w_3^R$ with $|w_1| = |w_3| > 0$, $|w_2| = |w_4|$, and $w_1 \neq w_3$. 
    If the $\CS_{|w_1|}(w_1 w_2 w_2^R w_1^R) = \CS_{|w_1|}(w_3 w_4 w_4^R w_3^R)$, 
    then $M$ also accepts a non-palidrome of length $2|w_1| + 2|w_2|$.
\end{corollary}
\begin{proof}
By \cref{lem:pastingcs}, the non-palindrome $w_1 w_4 w_4^R w_3^R$ is accepted. 
\end{proof}

Now we are ready to prove the $\Omega(n^2)$ lower bound for $\PAL$. We will prove a quantitative
bound, which will be useful for our randomized constructive separation:

\todopolish{check that formulas are correctly updated everywhere}
\begin{theorem}
\label{thm:palindromebound}
Let $M$ be a non-deterministic 1-TM running in time $T(n)$ with $s$ states
that rejects every non-palindrome.
Then, for every even $n$, the number of palindromes of length $n$ accepted by $M$ is
bounded by $2^{\frac{(4 \log s) T(n)}{n} + \frac{n+6}{4}}$.
\end{theorem}
\begin{proof}
Let $P_n$ be the set of palindromes of length $n$ accepted by $M$. 

Let $$\ell_i = \frac{1}{|P_n|} \sum_{p \in P_n} |\CS_i(p)|$$ be the average length of the
crossing sequence at position $i$ on the tape for all palindromes in $P_n$.
By Markov's inequality, there are at least $\frac{1}{2}|P_n|$ palindromes $p$
from $P_n$ for which $|C_i(p) \leq 2\ell_i$.
Now, fix $i \in [1, n/2]$ and note that:

\begin{itemize}
    \item There are $1 + s + \dots + s^{2\ell_i} < 2s^{2\ell_i}$ crossing sequences
    at position $i$
    of length at most $2\ell_i$, where $s$ is the number of states in $M$.
    \item There can be at most $2^{n/2-i}$ palindromes in $P_n$ with the same crossing sequence
    at position $i$,
    since in the decomposition $p = w_1 w_2 w_2^R w_1^R$ with $|w_1| = i$,
    the prefix
    $w_1$ must be the same for all such palindromes because of \cref{cor:crosscollision}
    and the assumption that $M$ rejects every non-palindrome. 
\end{itemize}

Therefore:

$$
\frac{1}{2}|P_n| < 2 s^{2\ell_i} \cdot 2^{n/2-i}.
$$

Rearranging: 

\begin{equation}
\ell_i > \frac{\log |P_n| + i - n/2 - 2}{2 \log s}.
\label{previneqpal}
\end{equation}

For each palindrome $p \in P_n$, we have that $\sum_{i=1}^{n/2} \CS_i(p) \leq T(n)$, so

$$
\sum_{i=1}^{n/2} \ell_i = \frac{1}{|P_n|} \sum_{i=1}^{n/2} \sum_{p \in P_n} |\CS_i(p)| = \frac{1}{|P_n|} \sum_{p \in P_n} \sum_{i=1}^{n/2} |\CS_i(p)|  \leq T(n)
$$

Hence, summing over all $i$ in \eqref{previneqpal}:

$$
T(n) > \sum_{i=1/}^{n/2} \frac{\log |P_n| + i - n/2 - 2}{2 \log s} = \frac{1}{2\log s} \left( \frac{n}{2} \log |P_n| - \frac{n^2}{8} -\frac{3n}{4} \right) 
$$

%So:

%$$
%\frac{(4 \log s) (T(n)+n/2)}{n} + \frac{n+6}{4} > \log |P_n|
%$$

And therefore:

$$
2^{\frac{(4 \log s) T(n)}{n} + \frac{n+6}{4}} > |P_n|
$$

As desired. 

    
\end{proof}

\begin{corollary}
\label{cor:palindromebound}
Every non-deterministic 1-TM recognizing $\PAL$ must run in $\Omega(n^2)$ time.
\end{corollary}
\begin{proof}
Such a machine $M$ rejects every non-palindrome, so we can apply \cref{thm:palindromebound} to get that
$2^{\frac{(4 \log s) T(n)}{n} + \frac{n+6}{4}} \geq 2^{n/2}$, so $T(n) \geq \frac{n(n-6)}{16 \log s} = \Omega(n^2)$.
%where $s$ is the number of states of the machine.
\end{proof}



\subsection{Randomized Constructive Separation}

We prove the following: 

\begin{theorem}
    \begin{itemize}
    \item There is a $\ZPP^\NP$-constructive separation of $\PAL$ from nondeterministic 1-TMs 
    running in time $o(n^2)$. 
    \item There is a $\BPP$-constructive separation of $\PAL$ from nondeterministic 1-TMs 
    running in time $o(n^2)$ which
    are promised to accept only palindromes. 
    \end{itemize}
\end{theorem}

Let $M$ be the non-deterministic Turing machine running in time $o(n^2)$ we are trying to refute, 
and let $M(x, b)$ be the result of the execution of $M$ on input $x$ and non-deterministic choices $b$.
Our refuter $R$ does the following on input $1^n$:

\begin{itemize}
	\item Uses the $\NP$ oracle to determine the truth of the statement 
	$$\exists x \in \{0, 1\}^n \exists b \in \{0, 1\}^{\leq n^2} : x \not\in \PAL \wedge M(x, b) \text{ accepts}.$$ 
	\begin{itemize}
	\item If it is true, it performs a standard search-to-decision 
	reduction with the $\NP$ oracle to determine such an $x$, and outputs $x$.
	\item Otherwise, it chooses a uniformly random palindrome of length $n$ and outputs it. 
    %(If $n$ is odd, it outputs anything and halts).
	\end{itemize}
\end{itemize}

In the first case, it is clear that $R$ outputs a counterexample for $M$. 
In the second case, we have that $M$ does not accept any non-palindrome of length $n$, so
by \cref{thm:palindromebound} it also rejects a fraction which goes to $1$ when $n \rightarrow \infty$
of palindromes of length $n$, 
so the probability of outputting a counterexample will be greater than $2/3$ for sufficiently large $n$.

Thus, we have a $\BPP^\NP$-constructive separation. In fact, it can be made $\ZPP^\NP$-constructive,
since we can verify the output to be a counterexample.
%but we will not insist in this distinction.
Also note that the refuter as stated (without performing verification of the output)
uses randomness and the $\NP$ oracle in a completely ``disjoint'' way:
in the case where the machine $M$ accepts a non-palindrome, it proceeds in a deterministic,
``$\PTIME^\NP$'' way, while in the case where the machine $M$ only accepts palindromes
it just outputs a random palindrome without using the $\NP$ oracle.
This differentiated behavior is of interest because, as we will see in the following sections,
there are different consequences for constructive separations from 1-TMs which are promised
to accept only palindromes and for constructive
separations from 1-TMs which are promised to accept all palindromes. 



\section{Constructive Separations Imply Circuit Lower Bounds}

Recall that in \cite{ConstructiveSeparations}, the following is proved:

\thmcsrefutertm*

We will reproduce the proof argument here, but with some modifications to suit our
purposes better. First, we prove the result with more generality, with the circuit lower 
bound we obtain being parametrized by the time complexity of the refuter, and not just
stated in two discrete cases for $\PTIME$ and $\PTIME^\NP$ (we will not discuss results about space complexity). 
This more granular statement will be 
useful to compare the result with the refuters we obtain in \cref{sec:refuteragainstweaker}.

Second, we emphasize in the statement that it is enough to have refuters against 1-TMs which
are promised to accept only palindromes, that is, that reject for every non-palindrome. 
This turns out to be an important aspect, because as we will see in \cref{sec:cryptotm}, no
$\PTIME$-constructive separation is possible even against deterministic 1-TMs running in time
$O(n^{1+\eps})$ assuming the existence of a cryptographic primitive. Thus, the second and third 
bullet points of \cref{thm:csrefutertm}, as stated in \cite{ConstructiveSeparations}, are 
vacuous under this cryptographic assumption. By introducing this strengthening, we can still
hope to find such constructive separations even if we believe that the cryptographic primitive exists. 

\begin{theorem}
    \label{thm:refutertm}
    Let $\{T_i(n)\}_{i \in \mathcal{I}}$ be a family of functions, let $\mathcal{O}$ be any language
    and let $\mathcal{C}^\mathcal{O}$ denote the class of algorithms
    running in time $T_i(n)$ for some $i \in \mathcal{I}$ with oracle access to $\mathcal{O}$. 
    Let $f(n)$ be a function with $f(n) = \Omega(\log n)$ and $f(n+1) = O(f(n))$.
    If there is a $\mathcal{C}^\mathcal{O}$-constructive separation of $\PAL$ from
    1-NTMs promised to accept only palindromes running in time $O(n \cdot f(n))$, then 
    $$\bigcup_{i\in \mathcal{I}} \DTIME[T_i(2^n)]^{\mathcal{O}} \not\subset \SIZE[f(2^{n/2})^\delta]$$
    for some universal constant $\delta > 0$.  
\end{theorem}
\begin{proof}
Assume that $\bigcup_{i\in \mathcal{I}} \DTIME[T_i(2^n)]^{\mathcal{O}} \subseteq \SIZE[f(2^{n/2})^\delta]$. 
By \cref{lem:circuitbinarytounary}, the output of any refuter $R \in \mathcal{C}^\mathcal{O}$ has circuit complexity
$O(f(n+1)^\delta) = O(f(n)^\delta)$. We will construct a 1-NTM $M$ running in time $O(n \cdot f(n))$
that works correctly on the output of any such refuter, proving the contrapositive of the desired statement. 
First, $M$ guesses a circuit $C$ of size $s = O(f(n)^\delta)$ and writes its description around the beginning of the tape.
Then, $M$ verifies that the circuit generates the input $x$ by moving the circuit and a counter through the tape
and checking that $C(i) = x_i$. Finally, $M$ checks that $C(i) = C(n-i+1)$ for all $1 \leq i \leq n/2$, verifying that
the input is indeed a palindrome. Moving the circuit through the input can be done in time $O(n \cdot (\log n + s))$ and each
of the $O(n)$ evaluations of the circuit can be done in time $O(s^{O(1)})$, so if $\delta$ is small enough the total
time complexity is $O(n \cdot f(n))$. 
\end{proof}

\begin{corollary}
The following hold:
\begin{itemize}
    \item A $\PTIME^{\NP}$-constructive separation of $\PAL$ from nondeterministic $O(n^{1.1})$ time
    one-tape Turing machines promised to accept only palindromes implies $\E^{\NP} \not \subset \SIZE[2^{\delta n}]$ for some constant $\delta >0$.
	    
    \item A $\PTIME$-constructive separation of $\PAL$ from nondeterministic $O(n^{1.1})$ time
    one-tape Turing machines promised to accept only palindromes implies $\E \not \subset \SIZE[2^{\delta n}]$ for some constant $\delta >0$.
\end{itemize}
\end{corollary}
\begin{proof}
Take $\mathcal{I} = \mathbb{N}$, $T_i(n) = n^i$, $f(n) = n^{0.1}$ and $\mathcal{O} = \SAT$ or $\mathcal{O} = \emptyset$ (respectively) in \cref{thm:refutertm}.
\end{proof}

Interestingly, those lower bounds for $\E^{\NP}$ and $\E$ are precisely the ones that are needed for derandomization of
algorithms:

\begin{theorem}[\cite{NW94, IW97}] 
    \label{thm:nisanwigderson}
    \begin{itemize}
    \item If $\E \not\subset \SIZE[2^{\delta n}]$, then there exists a pseudorandom generator $G : \{0, 1\}^s \to \{0, 1\}^n$,
    with $n = 2^{\Omega(s)}$ running in time $2^{O(s)}$ which fools circuits of size $n^3$.
    In particular, this implies $\BPP = \PTIME$. 
    \item If $\E^{\NP} \not\subset \SIZE[2^{\delta n}]$, then there exists a pseudorandom generator $G : \{0, 1\}^s \to \{0, 1\}^n$,
    with $n = 2^{\Omega(s)}$ running in time $2^{O(s)}$ with an $\NP$ oracle which fools circuits of size $n^3$.
    In particular, this implies $\BPP \subseteq \PTIME^\NP$. 
    \end{itemize}
\end{theorem}

Therefore, deterministic constructive separations of $\PAL$ from $O(n^{1+\eps})$ time 1-NTMs imply breakthrough results in
derandomization. At the same time, such deterministic constructive separations could be obtained by derandomizing the
probabilistic constructive separation of the previous section. The difference here between the two kinds of derandomization
is that in the second case we would need to have pseudorandom generators against \emph{nondeterministic} machines, which 
is a stronger requirement than the generators against deterministic circuits provided by \cref{thm:nisanwigderson}.

On the other hand, our generator only needs to fool nondeterministic one-tape machines running in $O(n^{1+\eps})$ time, 
which is a much weaker model than general non-deterministic polynomial-size circuits. In this setting, a pseudorandom
generator with seed length $\tilde{O}(\sqrt{T})$ against \emph{deterministic} 1-TMs running in time $O(T)$ is known 
\cite{Impagliazzo94}. Also, while the generators of \cref{thm:nisanwigderson} and \cite{Impagliazzo94} fool all 
$n^3$-size circuits and $O(T)$-time 1-DTMs respectively, our generator can be \emph{targeted}, that is, it can depend
on which TM $M$ we are trying to fool. And it does not need to be a full pseudorandom generator in the sense of 
indistinguishability from randomness; it can be a ``hitting set generator'' \cite{Andreev98} which generates at least
one element outside the square-root-sized set defined by \cref{thm:palindromebound}. All in all, it is not clear
that the requierements to derandomize our constructive separation are much stronger than the results of 
\cref{thm:nisanwigderson} by achieving it. 

\begin{question}
    What are the relationships between constructive separations of $\PAL$ from 1-NTMs, (targeted) derandomization/hitting sets
    of $O(n^{1+\eps})$ 1-NTMs, and derandomization of deterministic machines/circuits? 
    Can we find some sort of equivalence between them, or evidence than one of them is harder than the others? 
\end{question}




\section{Strongly Constructive Separations Do Not Exist Under Cryptographic Assumptions}
\label{sec:cryptotm}

In this section, we will see that $\BPP$-constructive separations of $\PAL$ against $O(n^{1+\eps})$ time 1-TMs do not exist if we assume
the existence of a certain cryptographic primitive. To show this, we will use an idea from \cite{Oliveira24}, where it is used 
to show that lower bounds for 1-TMs recognizing palindromes can not be proved in certain weak theories of arithmetic. Here we 
will reproduce their argument, adapting it to our setting.

The cryptographic primitives we will be using are \emph{keyless collision-resistant hash functions}.

\begin{definition}[Keyless Collision-Resistant Hash Function] A keyless collision-resistant hash
function with hash value length $m = m(n) < n$ is a polynomial-time computable
function $h \colon \{0, 1\}^* \to \{0, 1\}^*$ such that for every $x$ with $|x| = n$ we have $|h(x)| = m(n)$
and for every uniform probabilistic polynomial-time adversary
$\A$ and every $n$, we have
$$\Prob{\A(1^n)  \text{ outputs } \langle x_1, x_2 \rangle \text{ such that } x_1 \neq x_2, |x_1| = |x_2| = n \text{ and } h(x_1) = h(x_2)} \leq \eps(n)$$ 
where $\eps(n)$ is a \emph{negligible} function (that is, $\eps(n) = 1/n^{\omega(1)}$).
\end{definition}

The idea is that, if we have a TM that hashes the left half and the reverse of the right half of a palindrome and accepts if they are equal, a
refuter against this TM needs to find collisions in the hash function, which is supposed to be difficult. The main question here is whether
we can implement this in a one-tape TM running in time $O(n^{1+\eps})$, only assuming the existence of polynomial-time hash functions $h$.

The answer is yes: we can do it using a technique from cryptography known as the \emph{Merkle-Damgård construction} \cite{Merkle90, Damgard90}.

\begin{theorem}
\label{thm:hashonetape}
Suppose there exist keyless collision-resistant hash functions. Then, for all $0 < \delta, \eps < 1$, there exists a keyless collision resistant hash
function $H = H_{\delta, \epsilon}$ with hash value length $m(n) < n^\delta$ which can be computed in time $O(n^{1+\eps})$ in a one-tape
TM.
\end{theorem}

\begin{proof}
Let $h$ be a keyless collision-resistant hash function, computable in time $O(n^k)$ in a 1-TM for some $k$.
Without loss of generality, we can assume that $h$ has hash value length $n-1$ (we can always arbitrarily
pad the output values mantaining collision resistance). Given an input size $n$, 
let $m = m(n) = \min\{\lfloor n^\delta \rfloor,  \lfloor n^{\eps/k} \rfloor \}$ and define the following sequence of functions:

\todowrite{consistent notation for concatenation, also explain in introduction}

\begin{itemize}
    \item[] $H_{-1} \colon \{0, 1\}^{m-1} \to \{0, 1\}^{m-1}$: $H_{-1}(x) = x$
    \item[] $H_0 \colon \{0, 1\}^m \to \{0, 1\}^{m-1}$: $H_0(x) = h(x)$.
    \item[] $H_1 \colon \{0, 1\}^{m+1} \to \{0, 1\}^{m-1}$: $H_1(x \concat b) = h(H_0(x) \concat b)$.
    \item[] $\dots$
    \item[] $H_i \colon \{0, 1\}^{m+i} \to \{0, 1\}^{m-1}$: $H_i(x \concat b) = h(H_{i-1}(x) \concat b)$.
    \item[] $\dots$
    \item[] $H_{n-m} \colon \{0, 1\}^{n} \to \{0, 1\}^{m-1}$: $H_{n-m}(x \concat b) = h(H_{n-m-1}(x) \concat b)$.
\end{itemize}

%See \cref{fig:merkledamgard} for a pictorial depiction. 

\todowrite{figure? (maybe)}

Take $H = H_{n-m}$. We have to prove that $H$ is a collision-resistant hash function, and that it can be
computed in time $O(n^{1+\eps})$ in a 1-TM. 

We first prove that $H$ is a collision-resistant hash function.
Let $\A_H$ be a polynomial-time adversary outputting collisions for $H$. Consider the following algorithm
$\A_h$: on input $1^m$, run $\A(1^n)$ for all $n$ so that $m(n) = m$
(there are polynomially many such values of $n$). 
If $\A_H$ outputs a valid collision $\langle x_1, x_2 \rangle$ for $H$ for some such value of $n$,
do the following: compute the smallest value $i \in [0, n-m]$ such that $x_1[1 \ldots m+i] \neq x_2[1 \ldots m+i]$
but $H_i(x_1[1 \ldots m+i]) = H_i(x_2[1 \ldots m+i])$ and output
$\langle H_{i-1}(x_1[1 \ldots m+i-1]) || x_1[m+i], H_{i-1}(x_2[1 \ldots m+i-1]) || x_2[m+i] \rangle$.
We have that $h(H_{i-1}(x_1[1 \ldots m+i-1]) || x_1[m+i]) = h(H_{i-1}(x_2[1 \ldots m+i-1]) || x_2[m+i])$,
and by the assumption that $i$ was the smallest value with the specified property, at least one of
$H_{i-1}(x_1[1 \ldots m+i-1]) \neq H_{i-1}(x_2[1 \ldots m+i-1])$ or $x_1[m+i] \neq x_2[m+i]$
hold. In any case, $\A_h(1^m)$ outputs a valid collision of length $m$ for $h$. Hence:

$$
\Prob{\A_H(1^n)  \text{ outputs a collision for } H } \leq \Prob{\A_h(1^{m(n)}) \text{ outputs a collision for } h} \leq \eps_{\A_h}(m(n)),
$$

where $\eps_{\A_h}(n)$ is a negligible function since $h$ is collision-resistant, 
and therefore $\eps_{\A_h}(m(n))$ is also negligible and we get that the probability of success
of any adversary is bounded by a negligible function, as desired. 

Now we describe how to compute the function in a one-tape TM in time $O(n^{1+\eps})$:

First, we compute $n = |x|$, the length of the input. We iterate through the input from left to right,
keeping a counter near the head in a ``separate track'' from the input 
(using a larger tape alphabet).  
The counter has at most $\log n$ bits and so updating it and moving it to the right takes
$O(\log n)$ time, for a total complexity of $O(n \log n)$. 

Second, we compute $m$ and we put a mark on the $m$-th character of $x$ on the tape.
Computing $m$ given $n$ in binary can be done in $\polylog(n)$ time, since
$\eps$ and $\delta$ are fixed rational numbers, and mantaining a counter until the $m$-th character can
be done in $O(m \log m) = O(n)$, for a total complexity of $O(n)$ for this part. 

Third, we compute the hash value in a process with $n-m+1$ iterations. The $i$-th iteration ($0 \leq i \leq n-m$) is 
as follows:

\begin{enumerate}
    \item At the beginning, the tape contains $H_{i-1}(x[1 \ldots m-1+i])$, followed by the $m+i$-th 
    charcater of $x$ with a mark on it, followed by the remaining characters of $x$.
    %(For convenience, we let $H_{-1}$ be the identity function, so that this is also true for $i = 0$).
    \item We scan the tape, find the marked character, move the mark one cell to the right and copy the value 
    $H_{i-1}(x[1 \ldots m-1+i]) \concat x[m+i]$ into a separate track. 
    \item We compute $h(H_{i-1}(x[1 \ldots m-1+i]) \concat x[m+i]) = H_i(x[1 \ldots m+i])$ in a the separate
    track, cleaning it after the execution.
    \item We copy the value of $H_i(x[1 \ldots m+i])$ back into the main track, so that the tape now contains
    $H_{i}(x[1 \ldots m+i])$, followed by the $m+i+1$-th character with a mark, followed by the rest of characters
    of $x$.
\end{enumerate}

After $n-m+1$ iterations, we have the value of $H_{n-m}(x)$ in the tape, as desired. Copying the values from one
track to the other can be done in $O(m)$ time; the most expensive part of one iteration is computing $h$, which 
is done in $O(m^k)$ time. Therefore, the total time complexity is $O(n \cdot m^k) = O(n^{1+\eps})$. 



\end{proof}

%\begin{figure}
%    \label{fig:merkledamgard}
%    \missingfigure{merkle-damgard construction}
%    \caption{Merkle-Damgård construction.}
%\end{figure}

\begin{theorem}
    \label{thm:nobppseparation}
    Assuming the existence of keyless collision-resistant hash functions, there does not exist a 
    $\BPP$-constructive separation of $\PAL$ against 1-DTMs running in time $O(n^{1+\eps})$ for any $\eps > 0$.
\end{theorem}

\begin{proof}
Construct a machine $M$ that computes the hash function $H = H_{\eps, \eps}$ of \cref{thm:hashonetape} on the first
half on the input and on the reverse of the second half of the input, and accepts if they are equal. Note that dividing the 
input into two halves can be done in $O(n \log n)$ time by moving a counter, 
and that the computation of $H$ can be done on the reverse of
the second half without needing to explicitly write the reverse of the second half of the tape. Comparing the 
two hashes can be done in $O(mn) = O(n^{1+\eps})$, so the machine $M$ runs in time $O(n^{1+\eps})$. 

Any counterexample against $\PAL$ for $M$ must be of the form $x_1 x_2^R$ or $x_1 b x_2^R$,
where $x_1 \neq x_2$ and $\langle x_1, x_2 \rangle$ is a collision for $H$.
Thus, any polynomial-time refuter $R$ that outputs counterexamples against $M$ can be transformed into an 
efficient adversary that outputs collisions for $H$, contradicting that $H$ is collision-resistant. 
\end{proof}

Note that the machine $M$ we constructed in \cref{thm:nobppseparation} accepts all palindromes (and therefore fails only on non-palindromes), so
the result also applies to $\BPP$-constructive separations against 1-TMs promised to fail only on non-palindromes.
This offers an interesting contrast with \cref{thm:refutertm}: refuting 1-NTMs promised to fail only on palindromes proves
circuit lower bounds, while refuting 1-TMs promised to fail only on non-palindromes proves non-existence of cryptographic
primitives, which is an ``upper bound''.




\section{Constructive Separations Against Weaker 1-TMs}
\label{sec:refuteragainstweakertm}

One can wonder whether the results of \cref{thm:refutertm} are ``tight'', in the sense of requiring the minimal possible assumptions
on the hypotheses (such as the resources used by the one-tape TMs being refuted) in order to obtain a breakthrough result via a constructive
separation. It would be interesting to find refuters when each of the hypotheses are slightly weakened, in order to establish a ``threshold''
result which sets a dividing line between the situations where refuters can be found and the situations where proving the existence of
refuters is as hard as major open problems.

In this section, we show that indeed there exist such refuters for certain weakenings of the hypotheses of \cref{thm:refutertm}.

\subsection{Refuters Against $o(n \log n)$-time 1-TMs}

The most natural weakening of the hypotheses of \cref{thm:refutertm} is to consider deterministic 1-TMs rather than non-deterministic ones,
since they are a more natural model of computation, and to consider deterministic refuters for them. The results of the above 
section show that we can not hope to find such refuters against $O(n^{1+\eps})$-time machines, but we can wonder what happens with, say,
$O(n)$-time machines. 

It turns out that even nondeterministic $O(n)$-time (even $o(n \log n)$-time) one-tape machines are very limited in computational power,
which allows us to obtain simple deterministic refuters for nondeterministic machines. 

\begin{theorem}[\cite{Hartmanis68}]
Any language recognized by a 1-NTM running in $o(n \log n)$ time is regular.
\end{theorem}

\begin{theorem}
\label{thm:regularrefuter}
There is a $\PTIME$-constructive separation of $\PAL$ against $o(n \log n)$-time 1-NTMs. 
\end{theorem}

As usual, for simplicity we will focus only on refuting for even $n$. To construct the refuter, we first show that
``the first half'' of the counterexamples are generated by finite automata:

\todowrite{Write in introduction notation used for automata}

\begin{lemma}
\label{lem:regularlang}
Given $L \in \REG$ over any alphabet $\Sigma$, the following languages are also regular:
\begin{enumerate}
    \item $L_1 := \{w : ww^R \in L\}$.
    \item $L_2 := \{w : \exists \tilde{w} \in \Sigma^*, |w| = |\tilde{w}|, w \neq \tilde{w}, w\tilde{w}^R \in L\}$.
\end{enumerate}
\end{lemma}
\begin{proof}
Let $A(L) = (Q^0, q^0_0, \delta^0, S^0)$ be the DFA for $L$.

We construct the DFA $A(L_1) = (Q^1, q^1_0, \delta^1, S^1)$ for $L_1$ by setting
\begin{gather*}
Q^1 = Q^0 \times 2^{Q^0}, \\ q_0^1 = (q_0^0, S^0), \\ S^1 = \{(q, S) : q \in S\}, \\
\delta^1((q, S), x) = (\delta^0(q, x), \{q' : \delta^0(q', x) \in S\}).
\end{gather*}

Informally, we are simulating the DFA $A(L)$, but each time a character is read we are also ``moving'' the set of accepting states
in the inverse direction of the transitions. 

We can easily verify by induction that the following property is satisfied: $A(L_1)$ is in state $(q, S)$ after reading word $w$ if and only if
$A(L)$ is in state $q$ after reading $w$ and $S$ is the set of states $s$ that satisfy that, if $A(L)$ reads $w^R$ starting at state $s$, then it
ends in an accepting state in $S^0$. As a corollary, $A(L_1)$ accepts $w$ if and only if $A(L)$ accepts $ww^R$, as desired. 

We construct the NFA $A(L_2) = (Q^2, q^2_0, \delta^2, S^2)$ for $L_2$ by setting
\begin{gather*}
Q^2 = Q^0 \times 2^{Q^0} \times \{0, 1\}, \\ q_0^2 = (q_0^0, S^0, 0), \\ S^2 = \{(q, S, 1) : q \in S\}, \\
\delta^2((q, S, 0), x) = \{(\delta^0(q, x), \{q' : \delta^0(q', x) \in S\}, 0), (\delta^0(q, x), \{q' : \exists y \neq x \, \delta^0(q', y) \in S\}, 1)\}, \\
\delta^2((q, S, 1), x) = \{(\delta^0(q, x), \{q' : \exists y \, \delta^0(q', y) \in S\}, 1)\}.
\end{gather*}

Informally, we simulate the DFA $A(L)$ as before, but this time the word $\tilde{w}$ we are reading ``backwards'' (by moving the set of 
accepting states) needs to be different from the word $w$ we are reading forwards, so we divide the states in two stages depending on 
whether there is already at least one different character in the partial words $w$ and $\tilde{w}$ or not. 

The invariant property that is satisfied in this case is:

\begin{itemize}
    \item  $A(L_2)$ can reach state $(q, S, 0)$ after reading word $w$ if and only if
    $A(L)$ is in state $q$ after reading $w$ and $S$ is the set of states $s$ that satisfy that, if $A(L)$ reads $w^R$ starting at state $s$,
    then it ends in an accepting state in $S^0$, and
    \item $A(L_2)$ can reach state $(q, S, 1)$ after reading word $w$ if and only if
    $A(L)$ is in state $q$ after reading $w$ and $S$ is the set of states $s$ that satisfy that, 
    there exists $\tilde{w} \in \Sigma^*, |w| = |\tilde{w}|, w \neq \tilde{w}$ so that if $A(L)$ reads $\tilde{w}^R$ starting at state $s$,
    then it ends in an accepting state in $S^0$.
\end{itemize}

This implies that there is an accepting execution for $w$ in $A(L_2)$ if and only if there exists
$\tilde{w} \in \Sigma^*, |w| = |\tilde{w}|, w \neq \tilde{w}$ so that $A(L)$ accepts $w\tilde{w}^R$, as desired. 


\end{proof}


\begin{proof}[Proof of \cref{thm:regularrefuter}]
Let $L \in \REG$ be the language recognized by the 1-NTM. By \cref{lem:regularlang}, the languages 
$L_1 := \{w : ww^R \not\in L\}$ and $L_2 := \{w : \exists \tilde{w} \in \Sigma^*, |w| = |\tilde{w}|, w \neq \tilde{w}, w\tilde{w}^R \in L\}$
are also regular. We assume we have access to the automata $A_{L_i}$ recognizing those languages. Note that these automata are of size $O(1)$.

%On input $1^n$ with $n$ even, our refuter does the following:

%\begin{enumerate}
%\item Search for a string $w \in L_1$ or a string $w \in L_2$ of size $n/2$. This can be done in time
%$O(n \cdot \text{size of automaton}) = O(n)$ time by constructing the product automaton for $L_i \cap \{0, 1\}^{n/2}$. 
%    \begin{enumerate}
%        \item If a valid $w \in L_1$ is found, output $ww^R$. 
%        \item If a valid $w \in L_2$ is found, find the corresponding extension $w\tilde{w}^R$ and output it. 
%        In order to find the extension, find an accepting word with prefix $w$ in the product automaton for $L \cap \{0, 1\}^n$. 
%        Note that it is possible to find $ww^R$ as the accepting word, and then it would not be a valid counterexample.
%        In this case, we search for a second accepting word with prefix $w$ different from $ww^R$. This be done in $O(n)$ time
%        in total.   
%        \item Otherwise, halt without outputting anything.
%    \end{enumerate}
%\end{enumerate}

Our refuter is described in \cref{alg:regularrefuter}.

\begin{algorithm}
\caption{Refuter of \cref{thm:regularrefuter}}\label{alg:regularrefuter}
\begin{algorithmic}
\Input $1^n$
\Require $n$ even.
\Hardcode $A_{L_1} \gets$ DFA corresponding to $L_1$.
\Hardcode $A_{L_2} \gets$ DFA corresponding to $L_2$.
\Program
\State $w \gets \Call{FindWord}{A_{L_1}, n/2}$ \Comment{$\Call{FindWord}{A, \ell}$ finds a word of length $\ell$ accepted by automaton $A$.}
\If{Found $w$}
    \Output $ww^R$
    \Halt
\EndIf
\State $w \gets \Call{FindWord}{A_{L_2}, n/2}$
\If{Found $w$}
    \State $u \gets \Call{ExtendWord}{A_{L_2}, w, ww^R, n}$ \Comment{$\Call{ExtendWord}{A, p, v, \ell}$ finds a word of length $\ell$, with prefix $p$ and different from word $v$, accepted by automaton $A$.}
    \Output $u$
    \Halt
\EndIf
\Halt
\end{algorithmic}
\end{algorithm}

\textsc{FindWord} can be implemented in time $O(n)$ by constructing the product automaton for the language $L(A) \cap \{0, 1\}^\ell$. 
Similarly, \textsc{ExtendWord} can be implemented in time $O(n)$ by constructing the corresponding product automaton and finding a 
second accepting path if the first found one results in the undesired word (the graph of the product automaton is acyclic, so finding
a second path can be done simply by e.g. a DFS traversal). 

Thus, the refuter runs in polynomial (in fact, linear) time and that it outputs counterexamples for infinitely many values of $n$
(for any even value of $n$ for which there is a counterexample, it finds one, and for big enough $n$ there are always counterexamples by 
\cref{cor:palindromebound}).
\end{proof}

Note that the refuter can be made black-box by learning the regular language recognized by the TM using black-box queries. 
In fact, if we iterate over all automata of size (say) $\log \log n$ and output the counterexample found for each of the
automata, our algorithm still runs in polynomial time, so we have an explicit obstruction against all $o(n \log n)$ 1-TMs. 

\todoidea{can it be done in logspace??}

\subsection{Refuters Against $O(n \log n)$-time 1-TMs}

One-tape Turing machines running in $n \log n$ time have much more power than finite automata. And non-deterministic machines running in $n \log n$ time
are strictly more powerful than deterministic machines: for example, the language of non-palindromes can be recognized in $O(n \log n)$
non-deterministic time by guessing the index of the differing character and moving the $O(\log n)$ bits of the index around the tape,
while $\Omega(n^2)$ is required for deterministic machines to recognize this language as it is the complement of a language requiring $\Omega(n^2)$
time. 

Still, surprisingly, we can find a refuter against palindromes for these machines:

\begin{theorem}
    \label{thm:refuternlogn}

    \begin{itemize}
        \item There exists a $\PTIME^\NP$-constructive separation of $\PAL$ against one-tape nondeterministic Turing machines running in time $O(n \log n)$.
        \item There exists a $\PTIME$-constructive separation of $\PAL$ against one-tape deterministic Turing machines running in time $O(n \log n)$.
    \end{itemize}
\end{theorem}


We say that a crossing sequence is \emph{short} if it has length at most $4K \log n$. Our refuter works by repeatedly generating palindromes
and running the machine on them until either a non-accepting palindrome or a collision among short crossing sequences is found. The algorithmic
procedure is described in detail in \cref{alg:refuternlogn}; we proceed to explain how it works. 

\begin{algorithm}
    \caption{Refuter of \cref{thm:refuternlogn}}\label{alg:refuternlogn}
    \begin{algorithmic}
    \Input $1^n$
\Require $n$ even.
\Hardcode $M$ 1-TM to be refuted. 
\Hardcode $K \gets$ constant such that $M$ runs in $K n \log n$ steps.
\Hardcode $s \gets$ number of states of $M$. 
\Program
\State $L \gets 4K \log n$
\State \texttt{MarkedPrefixes} $\gets \{\}$
\State \texttt{SequenceByPrefix} $\gets []$
\RepeatTimes{$4 \cdot s^{L+1}$}
    \State $w \gets \Call{ChoosePalindrome}{n, \texttt{MarkedPrefixes}}$
    %\State Run $M$ on input $ww^R$
    \CustomBlock{\textbf{run} $M$ on input $ww^R$, and set:}
        \State $\texttt{Accepted} \gets$ result (whether $M$ accepts or not).
        \State $\CS \gets$ crossing sequences of the execution. 
    \EndCustomBlock
    \If{$\texttt{Accepted}$} 
        \For{$i \gets 1\ldots n/2$}
            \State $p \gets w[1\ldots i]$
            \If{$|\CS_i| \leq L$ and $p \not\in \texttt{MarkedPrefixes}$}
                \For{$\tilde{p} \in \texttt{MarkedPrefixes}$}
                    \If{$|\tilde{p}| = i$ and $\texttt{SequenceByPrefix}[\tilde{p}] = \CS_i$}
                        \State $v \gets w(i\ldots n/2]$
                        \Output $\tilde{p}vw^R$
                        \Halt
                    \EndIf
                \EndFor
                \State $\texttt{MarkedPrefixes} \gets \texttt{MarkedPrefixes} \cup \{p\}$
                \State $\texttt{SequenceByPrefix}[p] \gets \CS_i$
            \EndIf
        \EndFor
    \Else
        \Output $ww^R$
        \Halt
    \EndIf
\EndRepeatTimes
\Halt
    \end{algorithmic}
\end{algorithm}
\begin{algorithm}
    \caption{Procedure $\textsc{ChoosePalindrome}$ for \cref{alg:refuternlogn}}\label{alg:choosepalindrome}
    \begin{algorithmic}
    \Function{ChoosePalindrome}{n, $S$}
    \State $\texttt{Weight} \gets []$ \Comment{Associative array of integers. If $\texttt{Weight}$ is accessed for a key that has not been defined yet, it is initialized with default value $0$.}
    \State $\texttt{VerticesByLength} \gets \Call{Array}{n/2}$ \Comment{Array of sets.}
    \For{$v \in S$}
        \State $\texttt{VerticesByLength}[|v|] \gets \texttt{VerticesByLength}[|v|] \cup \{v\}$
    \EndFor
    \For{$\ell \gets n/2 \ldots 1$}
        \For {$v \in \texttt{VerticesByLength}[\ell]$}
            \If{$v \in S$}
                \State $\texttt{Weight}[v] \gets \texttt{Weight}[v]+1$
            \EndIf
            \State $p \gets \Call{Pop}{v}$
            \If{$\texttt{Weight}[p] = 0$}
                \State $\texttt{Weight}[p] \gets \texttt{Weight}[v]$
                \State $\texttt{VerticesByLength}[\ell-1] \gets \texttt{VerticesByLength}[\ell-1] \cup \{p\}$
            \Else
                \State $\texttt{Weight}[p] \gets \min\{\texttt{Weight}[p], \texttt{Weight}[v]\}$
            \EndIf
        \EndFor
    \EndFor
    \State $w \gets \emptyword$
    \RepeatTimes{$n/2$}
        \If{$\texttt{Weight}[w \concat 0] \leq \texttt{Weight}[w \concat 1]$}
            \State $w \gets w \concat 0$
        \Else
            \State $w \gets w \concat 1$
        \EndIf
    \EndRepeatTimes
    \State \Return $w$
\EndFunction
    \end{algorithmic}
\end{algorithm}

\todowrite{Explain empty word notation in introduction}

The program starts by initializing $L$ to the maximum length of short sequences (this will simply act as a shorthand
for the expression $4K \log n$ later, it is not a variable), and by creating two empty containers, 
$\texttt{MarkedPrefixes}$ and $\texttt{SequenceByPrefix}$. $\texttt{MarkedPrefixes}$ will be a set
containing prefixes of palindromes accepted by $M$ for which the $M$ generated a short crossing
sequence at the right end of the prefix in an accepting execution. $\texttt{SequenceByPrefix}$ will be
an associative array mapping each prefix in $\texttt{MarkedPrefixes}$ to the corresponding 
short crossing sequence.

Next, we have the main loop, in which in each iteration we choose a palindrome $ww^R$ and run $M$ on it.  
We choose the palindrome using the procedure \textsc{ChoosePalindrome}, described in \cref{alg:choosepalindrome}. 
This procedure returns a word $w$ of length $n/2$, and the corresponding palindrome will be $ww^R$. 
What this procedure does is to choose a $w$ minimizing the number of prefixes of $w$ which are
in \texttt{MarkedPrefixes}. It is easier to visualize it by thinking of a binary tree of height $n/2$ whose
vertices correspond to all binary words of length up to $n/2$, the root being the empty word and the two children
of each vertex corresponding to the word of the parent concatenated with each of the two characters in the alphabet.

\begin{figure}
    \missingfigure{Tree of words}
\end{figure}

In this setting, \texttt{MarkedPrefixes} would correspond to marked vertices in the tree, and we want to find a path
from the root to a leaf visiting the minimum number of marked vertices. We can compute the minimum number of marked
vertices in such a path by an easy recursive formula: 
$$\texttt{Weight}[v] = \texttt{Weight}[\text{child}_1(v)] + \texttt{Weight}[\text{child}_2(v)] + [1 \text{if } v \text{ is marked}],$$
and once we have computed it for all vertices, find a path with minimum weight by repeatedly descending
to the child with smaller weight. However, we can not explore the whole tree since it is of exponential size.
\textsc{ChoosePalindrome} does this in a ``bottom-up'' way so that only vertices with weight greater than $0$ are
explored. See the pseudocode description at \cref{alg:choosepalindrome} for details.

After choosing the palindrome and running it, if $M$ does not accept we can output it as a counterexample.
If $M$ does accept it, we analyze the crossing sequences at the first half of the input. If there is a prefix
with a short crossing sequence that we have not marked yet, we iterate over all other marked prefixes of the same
size. If we had marked a different prefix with the same short crossing sequence, we can replace the current prefix
by that one and we obtain a non-palindrome that is accepted, which we can output as a counterexample. If not, we
add it to the \texttt{MarkedPrefixes} and \texttt{SequenceByPrefix} containers.

We claim that, for all big enough $n$, we will always find a counterexample repeating this $4 \cdot s^{L+1}$ times,
and that the whole algorithm always runs in polynomial time. Let us prove some simple observations first:



\begin{lemma}
\label{lem:averageseqlength}
    Let $M$ be a 1-TM running in time $K n \log n$, and let $w$ be an accepted word of even length $n$. Then, for at least 
    $n/4$ of the indices $1 \leq i \leq n/2$, $|\CS_i(x)| \leq 4K \log n$. 
\end{lemma}
\begin{proof}
\begin{multline*}
K n \log n \geq \sum_{i=1}^{n/2} |\CS_i(x)| \geq 4K \log n \cdot \left(n/2 - |\{i : |\CS_i(x)| \leq 4K \log n\}|\right) \\ \implies  |\{i : |\CS_i(x)| \leq 4K \log n\}| \geq n/4.
\end{multline*}
\end{proof}

\begin{lemma}
\label{lem:choosepalindromeshortseq}
Let $M$ be a 1-TM. For $n$ large enough (depending on $M$), for every call to \textsc{ChoosePalindrome}
during the execution of the refuter
of \cref{alg:refuternlogn} against $M$, the word $w$ returned by \textsc{ChoosePalindrome} satisfies the following:
there exist at least $n/8$
indices $1 \leq i \leq n/2$ so that $|\CS_i(ww^R)| \leq 4K log n$ and the prefix $w[1 \ldots i]$ does not
already belong to \texttt{MarkedPrefixes}. (In other words, at least $n/8$ prefixes are added to \texttt{MarkedPrefixes}
in each iteration, if no counterexample is found).
\end{lemma}
\begin{proof}
There are $4 \cdot s^{L+1}$ iterations and in each iteration at most $n/2$ prefixes are added to \texttt{MarkedPrefixes}.
Therefore, at any time, there are at most $2n \cdot s^{L+1}$ prefixes in \texttt{MarkedPrefixes}. We claim that the 
word returned by $\textsc{ChoosePalindrome}(n, S)$ has at most $1+ \log |S|$  
prefixes in $S$, which for our calls with $S = \texttt{MarkedPrefixes}$ is at most 
$1 + \log(2n \cdot s^{L+1}) = O(\log n)$.

To prove so, consider the following greedy algorithm to find a path from the root of the tree to a leaf
passing through not too many marked vertices 
which could have been an alternative implementation for 
\textsc{ChoosePalindrome}: at each vertex, descend to the child with the least amount of marked vertices 
in its subtree. It is clear that after each step the number of marked vertices in the current subtree is halved,
so after $1 + \log |S|$ steps there are no marked vertices in the subtree. Therefore, there exists a path with
at most $1 + \log |S|$ marked vertices, and since \textsc{ChoosePalindrome} chooses the path with the minimum number of
marked vertices, its result must have at most that many marked vertices.

To finish, by \cref{lem:averageseqlength} the result $w$ has at least $n/4$ indices $0 \leq i \leq n/2$ with 
$|\CS_i(ww^R)| \leq 4K \log n$, so at least $n/4 - O(\log n) > n/8$ (for big enough $n$) of them are not in 
\texttt{MarkedPrefixes}. 
\end{proof}

\begin{lemma}
\label{lem:numiterations}
Let $M$ be a 1-TM. For $n$ large enough (depending on $M$), the execution of the refuter
of \cref{alg:refuternlogn} against $M$ outputs a word $x$ such that $M(x) \neq \PAL(x)$.
\end{lemma}
\begin{proof}
It is clear that whenever the refuter outputs a word, it is a correct counterexample. Therefore,
we have to prove that for $n$ large enough the refuter always outputs a counterexample. 
Suppose not, that for some $n$ large enough to apply \cref{lem:choosepalindromeshortseq}
the refuter does not output anything after the $4 \cdot s^{L+1}$ iterations.
Then, by \cref{lem:choosepalindromeshortseq}, \texttt{MarkedPrefixes} has at least $n/8 \cdot 4 \cdot s^{L+1}$
elements at the end of the execution. 
That means that for some $1 \leq i \leq n/2$, there must be at least $s^{L+1}$ of length $i$. 
There are $1 + s + \ldots + s^L < s^{L+1}$ crossing sequences of length at most $L$, so there are two 
prefixes of length $i$ with the same short crossing sequence. But that would have been detected 
during the execution of the program and a counterexample would have been printed, contradiction. 
\end{proof}

\begin{proof}[Proof of \cref{thm:refuternlogn}]
The correctness of the refuter is established by \cref{lem:numiterations}. We now need to prove that it runs in time 
polynomial in $n$.  The number of iterations is $4 \cdot s^{L+1} = 4 \cdot n^{4K \log s}$, polynomial in $n$.
Inside each iteration, $\textsc{ChoosePalindrome}(n, S)$ runs in $O(n \cdot |S|)$, which is polynomial (by the previous
argument that \texttt{MarkedPrefixes} always has an at most polynomial number of elements), 
running $M$ is $O(n \log n)$ (with possibly a call to an $\NP$ oracle),
and the subsequent iteration
over $i \gets 1 \ldots n/2$ and all elements of \texttt{MarkedPrefixes} is also polynomial.

Therefore, the refuter runs in polynomial time and we have a $\PTIME$-constructive (or $\PTIME^\NP$-constructive) 
separation. 
\end{proof}

\subsection{Additional Aspects and Consequences of the Refuters}

We can modify our refuter to work against machines that run in more than $n \log n$ time; we just have to modify the
definition of ``short sequence'' to ones that are shorter than $4T(n)/n$, where $T(n)$ is the runtime of the machine.
Our refuter then no longer runs in polynomial time in that case, though: 

\begin{theorem}
    For each 1-NTM $M$ running in time $O(n \cdot f(n))$ with $f(n) = o(n)$, there is a refuter of $\PAL$ against $M$ running in time 
    $n \cdot 2^{O(f(n))}$ with a $\NP$ oracle.
\end{theorem}

\begin{corollary}
    $\DTIME[2^{n+O(f(2^n))}] \not\subset \SIZE[f(2^{n/2})^\delta]$.
\end{corollary}

Note that by taking $f(n) = (\log n)^k$, the $\DTIME$-class bound we get is similar to the bound implicitly obtained
by the classic diagonalization used to show $\Sigma^{\PTIME}_4 \not\subset \SIZE(n^k)$ \cite{Kannan82}. This refuter
is too weak to give new circuit lower bounds, but on in some sense it seems to be ``close''. For example, if we could 
refute $O(n \polylog(n))$ 1-NTMs still using polynomial time, then we would prove $\E^\NP \not\subset \SIZE(n^k)$, 
which is not known. We now have a refuter against all $O(n^{1+o(1)})$ 1-TMs using subexponential ($2^{n^o(1)}$) time;
if we could refute $O(n^{1+\eps})$ in that time, we would get $\DTIME[2^{2^{o(n)}}] \not\subset \SIZE[2^{\delta n}]$.   

Therefore, it seems difficult to improve this refuter. However, there is some evidence that it does admit improvements.
First, for the ease of exposition here we have explained a version of the refuter which is white-box on the machine 
being refuted (indeed, it is a constructive separation in the sense of \cite{ConstructiveSeparations} which has access
to uncomputable information about the machine $M$, such as the constant $K$ in its runtime), but it can be improved 
to be black-box (with oracle access to $M$). Here we briefly explain which aspects need to be modified:

\begin{itemize}
    \item The refuter depends on the values of $s$ and $K$, which can not be obtained by black-box access. However,
    since we can verify whether the output is a valid counterexample, we can just iterate through
    $(s, K) = (1, 1), (2, 2), \ldots, (n, n)$, stopping as soon as we obtain a valid counterexample. 
    This makes the refuter no longer be a polynomial-time algorithm as a general oracle-algorithm, but for
    any fixed 1-TM $M$ given as an oracle, it will be a polynomial-time algorithm, since for $n$ big enough
    the refuter will always stop at a fixed, constant value of $s$ and $K$. 
    \item The refuter depends on knowing the crossing sequences of the execution in order to choose a new palindrome
    which ``minimizes the overlap'' with the short crossing sequences obtained by previous palindromes. 
    However, as we hinted at before, this is not really necessary: indeed, if we just generate a fixed set 
    of palindromes in which the $O(\log n)$ first bits are all different, we still get that any two palindromes
    overlap in at most $O(\log n)$ prefixes, which is enough for our purposes. 
    \item The refuter also depends on knowing the crossing sequences in order to find a collision and to construct
    the counterexample. However, we do know that a collision among the polynomially-many generated palindromes will exist
    after all the iterations, so we can just check all pairs of palindromes and all indices $i = 1, \ldots, n/2$, in 
    polynomial time in total. 
\end{itemize}

Therefore, we have a polynomial-time refuter against $O(n \log n)$-time 1-TMs that only needs oracle access to $M$
(and in particular it does not need an $\NP$ oracle for 1-NTMs). And we do not even take advantage of much information about
the oracle calls to $M$: we simply use them to ``test candidates'' for counterexamples, stopping when we find a correct
counterexample but otherwise not being adaptive at all in the queries!

However, do note that, since we need to stop at constant $(s, K)$ in order to run in polynomial time, we do not have 
a polynomial-time explicit-obstructions refuter, like we did in the $o(n \log n)$ setting. So we do lose something
when going from $o(n \log n)$ to $\Theta(n \log n)$. Still, given the austerity of our refuter when it comes to using
information about the specific machine we are refuting, it is difficult to believe we can not do better. 

\begin{question}
Is it possible to obtain a better $\PTIME^\NP$-constructive separation, even if not good enough to obtain new circuit lower
bounds, by taking advantage of white-box access and the $\NP$ oracle?
\end{question}

It is of note that our refuters work for non-deterministic 1-TMs in exactly the same way as they do for deterministic ones,
and they work for all $O(n \log n)$ 1-TMs, not just those that accept only palindromes. 
\cref{thm:refutertm} might reduce our hopes of obtaining great progress against non-deterministic machines, 
since that would imply breakthrough circuit lower bounds, but progress in refuters against deterministic 1-TMs doesn't seem
easy either, and yet we do not have any explicit reason why. \cref{thm:nobppseparation} explains why it shouldn't be possible
to have a refuter for all $O(n^{1+\eps})$ deterministic 1-TMs, but in the setting of 1-TMs which only accept palindromes,
which is linked to ``plausible'' derandomization, we do not have any obstruction. 

\begin{question}
Can we get a better $\PTIME$-constructive separation against deterministic 1-TMs promised to accept only 
palindromes? If not, can we get any result saying that it should be as difficult as obtaining certain derandomization,
like we have for non-deterministic machines?
\end{question}





%\chapter{Refuters for Streaming Algorithms}

%\todowrite{Write Chapter 5}
\todoideahp{Think about what to write here lol}
\todowritehp{Write the draft to send to Atserias}



\newpage

\printbibliography
\end{document}


